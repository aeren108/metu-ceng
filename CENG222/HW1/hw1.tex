\documentclass[12pt]{article}
\usepackage[utf8]{inputenc}
\usepackage{float}
\usepackage{amsmath}
\usepackage[hmargin=3cm,vmargin=6.0cm]{geometry}
\topmargin=-2cm
\addtolength{\textheight}{6.5cm}
\addtolength{\textwidth}{2.0cm}
\setlength{\oddsidemargin}{0.0cm}
\setlength{\evensidemargin}{0.0cm}
\usepackage{indentfirst}
\usepackage{amsfonts}

\begin{document}

\section*{Student Information}

Name : Ahmet Eren Çolak\\

ID : 2587921\\

\section*{Answer 1}
\subsection*{a)}
Event of all drawn balls are being black is the complement event of at least one of the balls is white. Let $P(B)$ be the probability of all drawn balls are black. Then probability of at least one of the balls is white, equals $1-P(B)$.
    \begin{align*}
        P(B) &= \frac{8}{10} \cdot \frac{11}{15} \cdot \frac{9}{12} = \frac{22}{50}\\
        1 - P(B) &= 1 - \frac{22}{50} = \frac{28}{50} = 56\%
    \end{align*}
\subsection*{b)}
Possibility of a ball chosen from a certain box being white equals to $\frac{\text{number of white balls}}{\text{number of all balls}}$. Since drawing a single ball from each box is independent from each other, probabilities can be directly multiplied to calculate their intersection.
    \begin{equation*}
        \frac{2}{10} \cdot \frac{4}{15} \cdot \frac{3}{12} = \frac{1}{75}
    \end{equation*}
\subsection*{c)}
Let $w$ be the number of white balls and $n$ be the total number of balls in a box. For each box, probability of two drawn balls being white can be calculated as: \[ \frac{{w \choose 2}}{{n \choose 2}} \]
Let $P(B1), P(B2), P(B3)$ be the probabilities of two drawn balls being white for first, second and third box respectively. When probabilities for each box is considered, second box has the highest possibility of two drawn balls are white.
\begin{align*}
    P(B1) &= \frac{{2 \choose 2}}{{10 \choose 2}} \approx 0.022\\
    P(B2) &= \frac{{4 \choose 2}}{{15 \choose 2}} \approx 0.057\\
    P(B3) &= \frac{{3 \choose 2}}{{12 \choose 2}} \approx 0.045
\end{align*}
\subsection*{d)}
Let $P(B1), P(B2), P(B3)$ be the probabilities of a single drawn ball being white for first, second and third box respectively.
\begin{align*}
    P(B1) &= \frac{2}{10} = 0.2\\
    P(B2) &= \frac{4}{15} \approx 0.27\\
    P(B3) &= \frac{3}{12} = 0.25
\end{align*}
Therefore for the first draw, it is best to draw from second box. For second draw let's calculate $P(B1), P(B2), P(B3)$ again. It is important to denote that second box missing a white ball now.
\begin{align*}
    P(B1) &= \frac{2}{10} = 0.2\\
    P(B2) &= \frac{3}{14} \approx 0.21\\
    P(B3) &= \frac{3}{12} = 0.25
\end{align*}
For the second draw it is best to draw from third box. 
\subsection*{e)}
Let $X$ be a random variable representing the \emph{number of white balls}. $X$ can take values between $0$ and $3$. Let $W$ represent the event of a ball drawn from a box is white and $B$ represent the event of a ball drawn from a box is black. Then values of $P(X=x)$ are:
\begin{align*}
    P(X=0) &= P(BBB) = \frac{8}{10} \cdot \frac{11}{15} \cdot \frac{9}{12} = 0.44\\
    P(X=1) &= P(WBB) + P(BWB) + P(BBW) = \frac{2}{10} \cdot \frac{11}{15} \cdot \frac{9}{12} +
    \frac{8}{10} \cdot \frac{4}{15} \cdot \frac{9}{12}
    \frac{8}{10} \cdot \frac{11}{15} \cdot \frac{3}{12} \approx 0.417\\
    P(X=2) &= P(WWB) + P(BWW) + P(WBW) = \frac{2}{10} \cdot \frac{4}{15} \cdot \frac{9}{12} +
    \frac{8}{10} \cdot \frac{4}{15} \cdot \frac{3}{12} +
    \frac{2}{10} \cdot \frac{11}{15} \cdot \frac{3}{12} = 0.13\\
    P(X=3) &= P(WWW) = \frac{2}{10} \cdot \frac{4}{15} \cdot \frac{3}{12} \approx 0.013
\end{align*}
To calculate the expected value of $X$, apply the formula:
\begin{align*}
    E(X) &= \sum_{x}^{} x \cdot P(X=x)\\
    E(X) &= 0\cdot 0.44 + 1\cdot 0.417 + 2\cdot 0.13 + 3\cdot 0.013 = 0.716
\end{align*}
\subsection*{f)}
Let $W$ be the event of a drawn ball from a random box being white and $B1$ be the event of a ball is drawn from the first box.
\begin{align*}
    P(W) &= \frac{\text{number of all white balls}}{\text{number of all balls}} = \frac{9}{37}\\
    P(B1) &= \frac{\text{number of balls in first box}}{\text{number of all ballss}} = \frac{10}{37}
\end{align*}
By the bayes rule $P(B1 \mid W)$ can be calculated as:
\begin{align*}
    P(B1 \mid W) &= \frac{P(W \mid B1) \cdot P(B1)}{P(W)}\\
    P(B1 \mid W) &= \frac{\cfrac{2}{10} \cdot \cfrac{10}{37}}{\cfrac{9}{37}} = \frac{2}{9}\\
\end{align*}
\section*{Answer 2}
\subsection*{a)}
Let $R$ be the event of destroying the ring and $S$ be the event of Sam is corrupted. It is known that $P(S) = 0.1$, \ $P(R \mid S) = 0.5$, \ $P (R \mid S') = 0.9$. By the bayes rule $P(S \mid R)$ can be expressed as:
\begin{equation*}
    P(S \mid R) = \frac{P(R \mid S) \cdot P(S)}{P(R)}
\end{equation*}
The unknown $P(R)$ can be calculated by applying the law of total probability. Since event of Sam is corrupted and not corrupted are exhaustive and mutually exclusive, these events can be used in calculation.
\begin{align*}
    P(R) &= P(R \cap S) + P(R \cap S') = P(R \mid S)P(S) + P(R \mid S')P(S')\\
    P(R) &= 0.5 \cdot 0.1 + 0.9 \cdot (1-0.1) = 0.86
\end{align*} 
Now $P(S \mid R)$ can be calculated:
\begin{equation*}
    P(S \mid R) = \frac{0.5 \cdot 0.1}{0.86} \approx 0.0581
\end{equation*}
\subsection*{b)}
Let $F$ be the event of Frodo is corrupted. Addition to the previous question it is known that $P(F) = 0.25$, $P(R \mid F) = 0.2$, $P(F \mid F' \cap S') = 0.9$, $P(R \mid F \cap S) = 0.05$. Then by the bayes rule $P(F \cap S \mid R)$ can be written as:
\begin{equation*}
    P(F \cap S \mid R) = \frac{P(R \mid F \cap S) \cdot P(F \cap S)}{P(R)}
\end{equation*} 
The unknown $P(R)$ can be calculated by the law of total probability. When events $F$, $S$ and $F' \cap S'$ considered, they are not exhaustive because both $F$ and $S$ events include the event $F \cap S$. Therefore if probability of event $F \cap S$ is subtracted from the summation, then the total law of probability can be applied.
\begin{align*}
    P(R) &= P(R \cap F' \cap S') + P(R \cap F) + P(R \cap S) - P(R \cap F \cap S)\\
    P(R) &= P(R \mid F' \cap S')P(F' \cap S') + P(R \mid F)P(F) + P(R \mid S)P(S) - P(R \mid F \cap S)P(F \cap S)\\
    P(R) &= 0.9 \cdot (1-0.1) \cdot (1-0.25) + 0.2 \cdot 0.25+0.5 \cdot 0.1-0.05 \cdot 0.1 \cdot 0.25 \approx 0.00177
\end{align*}
\section*{Answer 3}
\subsection*{a)}
Let $A$ and $I$ be random variables and representing the \emph{number of snowy days} in Ankara and İstanbul respectively. Probability of snowy days in total being equal to $4$ is $P(A+I=4)$. $P(A+I=4)$ equals to sum of joint distributions $P(2,2) + P(3,2)$ which is equal to $0.2 + 0.12 = 0.32$.
\subsection*{b)}
Two events are independent if and only if probability of their intersection equals to product of their individual probabilities. Two random variables are independent if and only if all possible pairs have this property.

By the law of total probability individual probabilities of random variables are:
\begin{align*}
    P(A=1) &= P(A=1 \cap I=1) + P(A=1 \cap I=2) = P(1,1) + P(1,2) = 0.3\\
    P(A=2) &= P(A=2 \cap I=1) + P(A=2 \cap I=2) = P(2,1) + P(2,2) = 0.5\\
    P(A=3) &= P(A=3 \cap I=1) + P(A=3 \cap I=2) = P(3,1) + P(3,2) = 0,2\\
    P(I=1) &= P(I=1 \cap A=1) + P(I=1 \cap A=2) + P(I=1 \cap A=3) =\\ 
    & \hspace{5.70cm} P(1,1) + P(2,1) + P(3,1) = 0.6\\
    P(I=2) &= P(I=2 \cap A=1) + P(I=2 \cap A=2) + P(I=2 \cap A=3) =\\ 
    & \hspace{5.70cm} P(1,2) + P(2,2) + P(3,2) = 0.4\\
\end{align*}
Let's check whether probability of their intersection equals to their products.
\begin{align*}
    P(A=1 \cap I=1) &= P(1,1) = 0.18 && P(A=1) \cdot P(I=1) = 0.3 \cdot 0.6 = 0.18\\
    P(A=1 \cap I=2) &= P(1,2) = 0.12 && P(A=1) \cdot P(I=2) = 0.3 \cdot 0.4 = 0.12\\
    P(A=2 \cap I=1) &= P(2,1) = 0.3 && P(A=2) \cdot P(I=1) = 0.5 \cdot 0.6 = 0.3\\
    P(A=2 \cap I=2) &= P(2,2) = 0.2 && P(A=2) \cdot P(I=2) = 0.5 \cdot 0.4 = 0.2\\
    P(A=3 \cap I=1) &= P(3,1) = 0.12 && P(A=3) \cdot P(I=1) = 0.2 \cdot 0.6 = 0.12\\
    P(A=3 \cap I=2) &= P(3,2) = 0.08 && P(A=3) \cdot P(I=2) = 0.2 \cdot 0.4 = 0.08\\
\end{align*}
Since all values hold, snowy days in Ankara and İstanbul are independent.
\end{document}

