\documentclass[11pt]{article}
\usepackage[utf8]{inputenc}
\usepackage[dvips]{graphicx}
\usepackage{fancybox}
\usepackage{verbatim}
\usepackage{array}
\usepackage{latexsym}
\usepackage{alltt}
\usepackage{hyperref}
\usepackage{textcomp}
\usepackage{color}
\usepackage{amsmath}
\usepackage{amsfonts}
\usepackage{tikz}
\usepackage{float}
\usepackage[hmargin=3cm,vmargin=5.0cm]{geometry}
%\topmargin=0cm
\topmargin=-2cm
\addtolength{\textheight}{6.5cm}
\addtolength{\textwidth}{2.0cm}
%\setlength{\leftmargin}{-5cm}
\setlength{\oddsidemargin}{0.0cm}
\setlength{\evensidemargin}{0.0cm}


\begin{document}

\section*{Student Information } 
%Write your full name and id number between the colon and newline
%Put one empty space character after colon and before newline
Full Name :  Ahmet Eren Çolak \newline
Id Number :  2587921


\section*{Q1}
Assume that there are positive integers smaller than $1$ and let $n$ be the least of them. It can expressed as:
	\begin{equation*}
		0 < n < 1
	\end{equation*}
Multiply both sides of the inequality $n < 1$ by $n$:
	\begin{equation*}
		n^2 < n
	\end{equation*} 
Since $n$ is a positive integer, it is known that $n^2$ is also a positive integer. This contradicts with our assumption that $n$ is the least positive integer between $0$ and $1$. Thus, there is no positive integer less than $1$.

\section*{Q2}
Base case: $S(1,1)=\frac{(1+1-1)!}{1!0!}=1$ and $x_1=1$ is the only solution. Therefore $S(1,1)$ is true. \newline
Induction step for $m$:
Assume $S(m,n)$ is true, then $S(m+1, n)$ should be true.
\begin{equation*}
	x_1+x_2+\dots+x_m+x_{m+1} = n
\end{equation*}
\begin{align*}
	&\text{For $x_{m+1}=0$, number of possibilities are $S(m,n)$} \\
	&\text{For $x_{m+1}=1$, number of possibilities are $S(m,n-1)$} \\
	&\vdots \\
	&\text{For $x_{m+1}=n-1$, number of possibilities are $S(m,1)$} \\
	&\text{For $x_{m+1}=n$, number of possibilities are $1$}
\end{align*}

So, $S(m+1,n)$ can be represented as:
\begin{equation*}
	S(m+1,n)=S(m,n)+S(m,n-1)+S(m,n-2)+\dots+S(m,1)+1
\end{equation*}

\section*{Q3}
\paragraph{\textbf{a.}}
A unit square contains $4$ congruent triangles. There are $21$ unit squares in total. From squares, there are $21 \cdot 4 = 84$ congruent triangles.
Apart from unit squares, dots in the figure that are on the hypotenuse, do not form a square. From the dots on hypotenuse, there are $7$ congruent triangles.
In total there are $84 + 7 = 91$ triangles.
\paragraph{\textbf{b.}}
Amount of onto functions can be found by subtracting non onto functions from all functions. There are $4^6$ functions in all possible ways. To find amount of non onto functions, a recursive function can be defined such that $F_n$ denotes the amount of onto functions from $6$ elements to $n$ elements.

There must be at least one element in the codomain of a function which is not image of any element of domain, so that function is not onto. If an element is eliminated from codomain, number of onto functions will be ${n \choose 1} F_{n-1}$. If each elimination considered a recursive function is obtained:
\begin{equation*}
	F_n = n^6 - {n \choose 1}F_{n-1} - {n \choose 2}F_{n-2} - \dots - {n \choose n-1}F_1
\end{equation*}

It is obvious that $F_1=1$. Then $F_4$ can be calculated:
\begin{align*}
	F_2 &= 2^6 - {2 \choose 1}\cdot 1 = 62 \\
	F_3 &= 3^6 - {3 \choose 1}\cdot 62 - {3 \choose 2}\cdot 1 = 540 \\
	F_4 &= 4^6 - {4 \choose 1}\cdot 540 - {4 \choose 2}\cdot 62 - {4 \choose 3}\cdot 1 = 1560
\end{align*}

\section*{Q4}
\paragraph{\textbf{a.}}
Let $a_n$ denote the number of possible strings. Then there are two disjoint possibilities for $a_{n}$:\newline
1) Valid string with $n-1$ length + any number (0, 1, 2) \newline
2) Invalid string with $n-1$ length + last digit of invalid string

For $n > 1$, $a_n = 3a_{n-1}+3^{n-1}-a_{n-1}=2a_{n-1}+3^{n-1}$
\paragraph{\textbf{b.}}
\begin{align*}
	a_1 &= 0 \\
\end{align*}
\paragraph{\textbf{c.}}
Homogeneous solution:
\begin{equation*}
	a_n^{(h)} = 2a_{n-1}=2^n
\end{equation*}
Particular solution:
\begin{equation*}
	a_n^{(p)} = 2a_{n-1}+3^{n-1}
\end{equation*}
Solution must be in form of $A\cdot3^n$.
\begin{align*}
	A\cdot3^n&=2A\cdot3^{n-1}+3^{n-1}\\
	A\cdot3^{n-1}&=3^{n-1} \\
	A&=1 \\
\end{align*}
General solution: $a_n=c\cdot2^n+3^n$, for $n=1, a_1=0$
\begin{align*}
	c\cdot2^1+3^1&=0\\
	2c+3&=0 \\
	c&=-3/2 \\
	a_n&=-3\cdot2^{n-1}+3^n
\end{align*}
\end{document}

