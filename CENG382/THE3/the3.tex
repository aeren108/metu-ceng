\documentclass[12pt,a4paper, margin=1in]{article}
\usepackage{fullpage}
\usepackage{amsfonts, amsmath, pifont}
\usepackage{amsthm}
\usepackage{amsmath}

\usepackage{geometry}
 \geometry{
 a4paper,
 % total={210mm,297mm},
 left=10mm,
 right=10mm,
 top=5mm,
 bottom=10mm,
 }
 \author{
  Çolak, Ahmet Eren\\
  \texttt{e2587921@ceng.metu.edu.tr}
}
\title{CENG 382 - Analysis of Dynamic Systems \\
20221\\
Take Home Exam 3 Solutions}
\begin{document}
\maketitle

\noindent\rule{19cm}{1.2pt}
\newcommand\ddfrac[2]{\frac{\displaystyle #1}{\displaystyle #2}}

\begin{enumerate}
% Write your solutions in the following items.

    \item % Q1
        \begin{enumerate}
            \item 
            Jacobian matrix of the system $F$ is:

            \begin{equation*}
                F(x_1, x_2) = \begin{bmatrix}
                    \ddfrac{\partial f_1}{\partial x_1} & \ddfrac{\partial f_1}{\partial x_2} \\ \\
                    \ddfrac{\partial f_2}{\partial x_1} & \ddfrac{\partial f_2}{\partial x_2} 
                \end{bmatrix} = \begin{bmatrix}
                    -1 & 2x_2 \\
                    6x_1 & -2
                \end{bmatrix}
            \end{equation*}

            \begin{equation*}
                F(0,0) = \begin{bmatrix}
                    -1 & 0 \\
                    0 & -2
                \end{bmatrix}
            \end{equation*}

            Characteristic equation of the Jacobian matrix at fixed point $x = (0,0)$ is:

            \begin{equation*}
                (\lambda + 1)(\lambda+2) = 0
            \end{equation*}

            Eigen values of the matrix are $\lambda_1 = -1$ and $\lambda_2 = -2$. Since all eigen values
            are less than $0$, fixed point is stable.
            \item 
            If a Lyapunov function exists for the fixed point $x=(0,0)$, then the fixed point is stable.
            Since Lyapunov function $V = x_1^2/2 + x_2^2/4$ is polynomial it is continous on $\mathbb{R}^2$.
            First condition of Lyapunov function is satisfied.\newline

            To find the minimum value of the Lyapunov function, let's check its partial derivatives.

            \begin{equation*}
                \ddfrac{\partial V(x_1,x_2)}{\partial x_1} = x_1 = 0 
            \end{equation*}

            \begin{equation*}
                \ddfrac{\partial V(x_1,x_2)}{\partial x_2} = \ddfrac{x_2}{2} = 0
            \end{equation*}

            Minimum value of the Lyapunov functions is at $x_1 = 0, x_2=0$. Therefore the second condition is satisfied.
            \newline

            Lastly, if the Lyapunov function $V(x_1, x_2)$ is decreasing on a spherical region centered at the fixed point for
            every trajectory of the system, then fixed point $x=(0,0)$ is stable.

            \begin{equation*}
                \ddfrac{dV(x_1, x_2)}{dt} = \ddfrac{\partial V(x_1, x_2)}{\partial x_1} \ddfrac{dx_1}{dt} + 
                \ddfrac{\partial V(x_1, x_2)}{\partial x_2} \ddfrac{dx_2}{dt}
            \end{equation*}

            \begin{equation*}
                \ddfrac{dV(x_1, x_2)}{dt} = x_1(x_2^2 - x_1) + \ddfrac{x_2}{2}(3x_1^2-2x_2)
            \end{equation*}

            \begin{equation*}
                \ddfrac{dV(x_1, x_2)}{dt} = x_2^2(x_1 - 1) + x_1^2(3x_2/2 - 1)
            \end{equation*}

            When $x_1 < 1$ and $x_2 < 2/3$ condition is satisfied. Spherical region is centered at $x=(0,0)$ and its radius is $2/3$.
            Since all conditions are satisfied, Lyapunov function $V$ exists and fixed point is stable.
        \end{enumerate}
        
        
    \item % Q2
        Let's choose Lyapunov function as the $V(x_1, x_2, x_3) = x_1^2 + x_2^2 + x_3^2$. Since the function is polynomial
        it is continous on $\mathbb{R}^3$.

        Let's check where the minimum of the function is.

        \begin{equation*}
            \ddfrac{\partial V}{\partial x_1} = 2x_1 = 0, \ \ x_1 = 0
        \end{equation*}
        \begin{equation*}
            \ddfrac{\partial V}{\partial x_2} = 2x_2 = 0,  \ \ x_2 = 0
        \end{equation*}
        \begin{equation*}
            \ddfrac{\partial V}{\partial x_3} = 2x_3 = 0,  \ \ x_3 = 0
        \end{equation*}
        
        Minimum of the function is at $(0,0,0)$, so the second condition is satisfied. 

        Let's check $V(x(k+1))) - V(x(k))$ to determine $V(x(k))$ is non-increasing.

        \begin{equation*}
            V(x(k+1)) = \ddfrac{x_1^2}{2} + \ddfrac{x_2^2}{2} + \ddfrac{x_3^2}{4}
        \end{equation*}

        \begin{equation*}
            V(x(k+1)) - V(x(k)) = \ddfrac{-x_1^2}{2} - \ddfrac{x_2^2}{2} - \ddfrac{3x_3^2}{4}
        \end{equation*}

        Since $x_1^2$, $x_2^2$ and $x_3^2$ are always non negative, it can be concluded that
        $V(x(k+1)) - V(x(k)) \leq 0$. Since Lyapunov function is always non increasing fixed point is stable.
    \item % Q3
        Let's first determine the fixed points of the system. 

        \begin{equation*}
            \ddfrac{dx}{dt} = 0
        \end{equation*}

        \begin{equation*}
            x_1 + x_2 - 4x-1(x_1^2 + x_2^2) = 0
        \end{equation*}
        \begin{equation*}
            -x_1 + x_2 - 4x-2(x_1^2 + x_2^2) = 0
        \end{equation*}
        \begin{equation*}
            \ddfrac{x_1+x_2}{x_1} = 4(x_1^2+x_2^2)
        \end{equation*}
        \begin{equation*}
            \ddfrac{x_2-x_2}{x_2} = 4(x_1^2+x_2^2)
        \end{equation*}
        \begin{equation*}
            1+\ddfrac{x_2}{x_1} = 1 - \ddfrac{x_1}{x_2}
        \end{equation*}
        \begin{equation*}
            x_2^2 = -x_1^2
        \end{equation*}

        The only values that satisfy the equation $x_2^2 = -x_1^2$ are $x_1=0$ and $x_2=0$.
         $(0,0)$ is the only fixed point of the system.
        Then we check the stability of the fixed point via linearization. Jacobian matrix of the system is:

        \begin{equation*}
            F(x_1, x_2) = \begin{bmatrix}
                \ddfrac{\partial f_1}{\partial x_1} & \ddfrac{\partial f_1}{\partial x_2} \\ \\
                \ddfrac{\partial f_2}{\partial x_1} & \ddfrac{\partial f_2}{\partial x_2} 
            \end{bmatrix} = \begin{bmatrix} 
                1-4x_2^2 - 12x_1^2 & -8x_1x_2 + 1 \\
                -1-8x_2x_1 & 1-12x_2-4x_1^2
            \end{bmatrix}
        \end{equation*}

        \begin{equation*}
            F(0,0)=\begin{bmatrix}
                1 & 1 \\
                -1 & 1
            \end{bmatrix}
        \end{equation*}

        Characteristic equation of the matrix above is:

        \begin{equation*}
            \lambda^2 - 2\lambda +2 = 0
        \end{equation*}

        Eigen values of the matrix are $\lambda_{1,2} = 1 \pm i$. Since real part of one of the eigen values
        is positive, fixed point is unstable.
        
        To check if the system has a periodic limit cycle, we should find where the Lyapunov function is decreasing and increasing.

        \begin{equation*}
            \ddfrac{dV(x_1,x_2)}{dt} = x_1(x_1+x_2-4x_1(x_1^2+x_2^2)) + 
            x_2(-x_1+x_2-4x_2(x_1^2+x_2^2))
        \end{equation*}
        \begin{equation*}
            \ddfrac{dV(x_1,x_2)}{dt} = x_1^2+x_2^2 - 4x_1^4 - 8x_1^2x_2^ - 4x_2^4
        \end{equation*}
        \begin{equation*}
            \ddfrac{dV(x_1,x_2)}{dt} = x_1^2+x_2^2 - 4(x_1^2+x_2^2)^2 = (x_1^2+x_2^2)(1-4(x_1^2+x_2^2))
        \end{equation*}
        
        When $0 < x_1^2+x_2^2 < 4$, $V(x_1, x_2)$ is increasing and when $x_1^2+x_2^2 > 4$, $V(x_1, x_2)$ is decreasing.
        Therefore the system has a periodic limit cycle on the circle $x_1^2 + x_2^2 = 4$.
        
    \item % Q4 
        \begin{enumerate}
            \item 
                Fixed points satisfy the equation in discrete systems $x(k+1)=x(k)$.

                \begin{equation*}
                    x(k) = 3-x^2(k) 
                \end{equation*}
                \begin{equation*}
                    x^2(k) + x(k) - 3 = 0 
                \end{equation*}
                \begin{equation*}
                    \tilde{x}_{1,2} = \ddfrac{-1 \pm \sqrt{13}}{2}
                \end{equation*}
            \item 
                Periodic points of prime period 2, satisfy the equation $f^2(x) = x$ where $f(x) = 3 - x^2$.

                \begin{equation*}
                    f^2(x) = f(3-x^2) = -4x^4 + 6x^2 - 6 = x
                \end{equation*}
                \begin{equation*}
                    x^4-6x^2+x+6=0
                \end{equation*}

                Roots of the equation $x^2+x-3=0$ are also the roots of the equation $x^4-6x^2+x+6=0$. In other words
                if the prime period of a value is $1$ (which means it is a fixed point), then $2$ is also its period.
                Because if a value is repeated in every step, it means it also repeats in every $2$ step. Therefore 
                we can divide $x^4-6x^2+x+6$ by $x^2+x-3$ without any remainders. When polynomial division is applied,
                $x^4-6x^2+x+6 = (x^2+x-3)(x^2-x-2)$ is obtained. Roots of the $x^2-x-2$ are the values of prime period 2. This 
                property of fixed points helps with finding roots.

                \begin{equation*}
                    x^2-x-2 = (x-2)(x+1) = 0
                \end{equation*}
                \begin{equation*}
                    x_1 = 2, \ \ x_2 = -1
                \end{equation*}
                
                Therefore periodic points of prime period 2 are $-1$ and $2$.

            \item 
                We can test the stability of periodic points via linearization. If the coefficient 
                of $x$ in the linearized form of $f^2(x)$ is negative, then periodic point is stable.
                \newline \newline
                Linearization for periodic point $x=2$:

                \begin{equation*}
                    \ddfrac{df^2(x)}{dx} = -4x^3+12x, \ \ \ddfrac{df^2}{dx}(2) = -8
                \end{equation*}
                \begin{equation*}
                    f^2(x) \cong f^2(2) + (-8)(x-2)
                \end{equation*}

                Since the coefficient of x ($-8$), is negative, periodic point is stable.
                \newline \newline
                Linearization for periodic point $x=-1$:
                \begin{equation*}
                    \ddfrac{df^2(x)}{dx} = -4x^3+12x, \ \ \ddfrac{df^2}{dx}(-1) = -8
                \end{equation*}
                \begin{equation*}
                    f^2(x) \cong f^(-1) + (-8)(x+1)
                \end{equation*}
                Since the coefficient of x ($-8$), is negative, this periodic point is also stable.
        \end{enumerate}


\end{enumerate}

\end{document}
