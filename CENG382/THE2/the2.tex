\documentclass[12pt,a4paper, margin=1in]{article}
\usepackage{fullpage}
\usepackage{amsfonts, amsmath, pifont}
\usepackage{amsthm}
\usepackage{graphicx}
\usepackage{tkz-euclide}
\usepackage{amsmath}
\usepackage{tikz}
\usepackage{pgfplots}
\usepackage{enumitem}
\usepackage{geometry}
 \geometry{
 a4paper,
 total={210mm,297mm},
 left=10mm,
 right=10mm,
 top=5mm,
 bottom=10mm,
 }
 \author{
  Çolak, Ahmet Eren\\
  \texttt{e2587921@ceng.metu.edu.tr}
}
\title{CENG 382 - Analysis of Dynamic Systems \\
20221\\
Take Home Exam 2 Solutions}

\makeatletter
\renewcommand*\env@matrix[1][*\c@MaxMatrixCols c]{%
  \hskip -\arraycolsep
  \let\@ifnextchar\new@ifnextchar
  \array{#1}}
\makeatother
\begin{document}
\maketitle

\newcommand\ddfrac[2]{\frac{\displaystyle #1}{\displaystyle #2}}

\noindent\rule{19cm}{1.2pt}

\begin{enumerate}[label=\textbf{\arabic*.}]
% Write your solutions in the following items.

    \item % Q1
        \begin{enumerate}
            \item
                \begin{equation*}
                    P = \begin{bmatrix}
                        0.7 & 0.2 & 0.1 \\
                        0.2 & 0.6 & 0.2 \\
                        0.1 & 0.4 & 0.5
                    \end{bmatrix}
                \end{equation*}
            
            \item 
                Probability that a randomly chosen grandson of an unskilled laborer is a professional man
                can be found in the entries of the second power of the state transition matrix, $P$.

                \begin{equation*}
                    P^2 = \begin{bmatrix}
                        0.54 & 0.3 & 0.16 \\
                        0.28 & 0.48 & 0.24 \\
                        0.2 & 0.46 & 0.34
                    \end{bmatrix}
                \end{equation*}

                Unskilled labors are the third state and the professionals are the first state. Therefore probability of a 
                randomly chosen grandson of an unskilled labor being a professional is the $0.2$.
            \item 
                Probability of a randomly chosen grandson of a professional being a professional 
                is $0.54$ as it can be found in the first entry of second power of the state transition matrix, $P^2$.
        
            \item 
                After 100 generations Markov chain is going to converge to a fixed state. If $p(k)$ is the state of the Markov chain after $k$ steps, $p(k+1)$ can be expressed as:
                \begin{equation*}
                    p(k+1) = p(k) \cdot P
                \end{equation*}

                As $k \rightarrow \infty$, system satisfies the equation below.

                \begin{equation*}
                    p(k+1) = p(k), \ \ p(k) = p(k) \cdot P
                \end{equation*}

                \begin{equation*}
                    \begin{bmatrix}
                        p_1 & p_2 & p_3
                    \end{bmatrix} = \begin{bmatrix}
                        p_1 & p_2 & p_3
                    \end{bmatrix} \begin{bmatrix}
                        0.7 & 0.2 & 0.1 \\
                        0.2 & 0.6 & 0.2 \\
                        0.1 & 0.4 & 0.5
                    \end{bmatrix}
                \end{equation*}

                \begin{align*}
                    -0.3p_1 + 0.2p_2 + 0.1p_3 &= 0 \\
                     0.2p_1 - 0.4p_2 + 0.4p_3 &= 0 \\
                     0.1p_1 + 0.2p_2 - 0.5p_3 &= 0
                \end{align*}

                When system of equations above is solved $p_1 = 1.5p_3$, $p_2 = 1.75p_3$ and $p_3$ is found free. 
                Sum of $p_1$, $p_2$ and $p_3$ must be equal to $1$.

                \begin{equation*}
                    1.5\cdot p_3 + 1.75 \cdot p_3 + p_3 = 4.25 \cdot p_3 = 1 
                \end{equation*}

                \begin{equation*}
                    p_3 = \frac{100}{425}
                \end{equation*}
                \begin{equation*}
                    p_1 = 1.5 \cdot p_3 = \frac{150}{425}
                \end{equation*}
                \begin{equation*}
                    p_2 = 1.75 \cdot p_3 = \frac{175}{425} 
                \end{equation*}

                \bigskip

                \begin{equation*}
                    p(100) \approx \begin{bmatrix}
                        \dfrac{150}{425} & \dfrac{175}{425} & \dfrac{100}{425}
                    \end{bmatrix}
                \end{equation*}
        \end{enumerate}
        
        
    \item % Q2
        \begin{enumerate}
            \item 
                If controllability matrix $M$, has rank 3, then system is controllable.

                \begin{align*}
                    M &= \begin{bmatrix}
                        B & AB & A^2B
                    \end{bmatrix} \\
                    M &= \begin{bmatrix}
                        1 & 0 & 0 \\
                        0 & -2 & 0 \\
                        0 & 0 & -2
                    \end{bmatrix}
                \end{align*}
                
                Since $rank(M) = 3$, system is controllable.
           \item 
                \begin{equation*}
                    x(1) = \begin{bmatrix}
                        0 & 0 & 1 \\
                        -2 & 0 & -1 \\
                        0 & 1 & 0
                    \end{bmatrix} \begin{bmatrix}
                        1 \\ 1 \\ 1
                    \end{bmatrix} + \begin{bmatrix}
                        u_0 \\ 0 \\ 0
                    \end{bmatrix} = \begin{bmatrix}
                        1 + u_0 \\ -3 \\ 1
                    \end{bmatrix}
                \end{equation*}

                \begin{equation*}
                    x(2) = \begin{bmatrix}
                        0 & 0 & 1 \\
                        -2 & 0 & -1 \\
                        0 & 1 & 0
                    \end{bmatrix} \begin{bmatrix}
                        1 + u_0 \\ -3 \\ 1
                    \end{bmatrix} + \begin{bmatrix}
                        u_1 \\ 0 \\ 0
                    \end{bmatrix} = \begin{bmatrix}
                        1 + u_1 \\ -3 -2u_0 \\ -3
                    \end{bmatrix}
                \end{equation*}

                \begin{equation*}
                    x(3) = \begin{bmatrix}
                        0 & 0 & 1 \\
                        -2 & 0 & -1 \\
                        0 & 1 & 0
                    \end{bmatrix} \begin{bmatrix}
                        1 + u_1 \\ -3 - 2u_0 \\ -3
                    \end{bmatrix} + \begin{bmatrix}
                        u_2 \\ 0 \\ 0
                    \end{bmatrix} = \begin{bmatrix}
                        -3 + u_2 \\ 1-2u_1 \\ -3-2u_0
                    \end{bmatrix}
                \end{equation*}

                \begin{equation*}
                    x(3) = \begin{bmatrix}
                        -3 + u_2 \\ 1-2u_1 \\ -3-2u_0
                    \end{bmatrix} = \begin{bmatrix}
                        4 \\ 4 \\ 4
                    \end{bmatrix}
                \end{equation*}

                \begin{align*}
                    u_0 &= -7/2 \\
                    u_1 &= -3/2 \\
                    u_2 &= 7
                \end{align*}
                
        \end{enumerate}
        
        
    \item % Q3
        \begin{enumerate}
            \item 
                Let's assume that the initial state of the system is $x(k) = \begin{bmatrix}
                    x_1(k) \\ x_2(k) \\ x_3(k)
                \end{bmatrix}$. Then sequence of outputs and following states are:
                

                \begin{equation*}
                    x(k+1) = \begin{bmatrix}
                        0 & 0 & 1 \\
                        -2 & 0 & 1 \\
                        0 & 1 & 0
                    \end{bmatrix} x(k) = \begin{bmatrix}
                        x_3(k) \\ -2x_1(k) - x_3(k) \\ x_2(k)
                    \end{bmatrix}
                \end{equation*}

                \begin{equation*}
                    x(k+2) = \begin{bmatrix}
                        0 & 0 & 1 \\
                        -2 & 0 & 1 \\
                        0 & 1 & 0
                    \end{bmatrix} x(k+1) = \begin{bmatrix}
                        x_2(k) \\ -2x_3(k) - x_2(k) \\ -2x_1(k) - x_3(k)
                    \end{bmatrix}
                \end{equation*}

                \begin{equation*}
                    y(k) = \begin{bmatrix}
                        0 & -2 & -4
                    \end{bmatrix} x(k) = -2x_2(k) - 4x_3(k)
                \end{equation*}

                \begin{equation*}
                    y(k+1) = \begin{bmatrix}
                        0 & -2 & -4
                    \end{bmatrix} x(k+1) = 4x_1(k) - 4x_2(k) + 2x_3(k)
                \end{equation*}

                \begin{equation*}
                    y(k+2) = \begin{bmatrix}
                        0 & -2 & -4
                    \end{bmatrix} x(k+2) = 8x_1(k) - 2x_2(k) + 8x_3(k)
                \end{equation*}

                The systems of output equations can represented in matrix form:

                \begin{equation*}
                    \begin{bmatrix}[ccc|c]
                        0 & 2 & -4 & y(k)\\
                        4 & -4 & 2 & y(k+1)\\
                        8 & 2 & 8 & y(k+2)
                    \end{bmatrix}
                \end{equation*}

                Since matrix above is non-singular there is a solution for $x_1(k)$, $x_2(k)$ and $x_3(k)$.
                It is possible the determine the initial state by looking at most $3$ consecutive outputs. Therefore system is observable.
            \item 
                Observability matrix $M$ equals to $\begin{bmatrix}
                    C \\ CA \\ CA^2
                \end{bmatrix}$.

                \begin{equation*}
                    M = \begin{bmatrix}
                        0 & 2 & -4\\
                        4 & -4 & 2\\
                        8 & 2 & 8
                    \end{bmatrix}
                \end{equation*}

                Since $rank(M) = 3$, system is observable.
        \end{enumerate}
    \item %Q4 
        \begin{equation*}
            \dot{x} = f(x) = 3x^2 - 3x^3
        \end{equation*}
        Fixed points are the values of $x$ which make $dx/dt$ zero.
        \begin{align*}
            \dot{x} = 3x^2 - 3x^3 &= 0 \\
            3x^2\cdot (1-x) &= 0 \\
        \end{align*}
        \begin{equation*}
            x_1 = 0, \ x_2 = 1
        \end{equation*}

        Linearization of $f(x)$ around the first fixed point $x_1 = 0$, can be written in the equation below.

        \begin{equation*}
            f(x) \cong \frac{d}{dx}f(x_1)(x-x_1) + f(x_1)
        \end{equation*}
        \begin{equation*}
            \frac{df}{dx} = 6x - 9x^2, \ \frac{d}{dx}f(0) = 0
        \end{equation*}
        \begin{equation*}
            f(x) \cong 0\cdot(x-0) + f(0) = f(0) = 0
        \end{equation*}

        Stability test fails because coefficient of the $x$ is zero. Problem requires further analysis.
        When $x$ is slightly larger than $x_1 = 0$, $f'(x) = 6x-9x^2 > 0$. Therefore $f(x)$ will move away from $x_1=0$
        which shows that $x_1=0$ is unstable.

        Linearization of $f(x)$ around the second fixed point $x_2 = 1$, can be written in the equation below.

        \begin{equation*}
            f(x) \cong \frac{d}{dx}f(x_2)(x-x_2) + f(x_2)
        \end{equation*}
        \begin{equation*}
            \frac{d}{dx}f(1) = -3
        \end{equation*}
        \begin{equation*}
            f(x) \cong -3\cdot(x-1) + f(1) = -3\cdot(x-1) + 0
        \end{equation*}

        Since coefficient of $x$ is negative, fixed point $x_2 = 1$ is stable.
\end{enumerate}

\end{document}
