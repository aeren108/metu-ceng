\documentclass[10pt,a4paper, margin=1in]{article}
\usepackage{fullpage}
\usepackage{amsfonts, amsmath, pifont}
\usepackage{amsthm}
\usepackage{graphicx}
\usepackage{flafter}
\usepackage{float}
\usepackage{pgf}
\usepackage{minted}

\usepackage{tkz-euclide}
\usepackage{tikz}
\usepackage{pgfplots}
\pgfplotsset{compat=1.13}

\usepackage{geometry}
 \geometry{
 a4paper,
 total={210mm,297mm},
 left=10mm,
 right=10mm,
 top=10mm,
 bottom=10mm,
 }
 % Write both of your names here. Fill exxxxxxx with your ceng mail address.
 \author{
  Çavuşoğlu, Arda\\
  \texttt{e2448249@ceng.metu.edu.tr}
  \and
  Çolak, Ahmet Eren\\
  \texttt{e2587921@ceng.metu.edu.tr}
}

\title{CENG 384 - Signals and Systems for Computer Engineers \\
Spring 2023 \\
Homework 2}
\begin{document}
\maketitle



\noindent\rule{19cm}{1.2pt}

\begin{enumerate}

\item %write the solution of q1
    \begin{enumerate}
    % Write your solutions in the following items.
    \item %write the solution of q1a
    $\Dot{y}(t) = x(t) - 5y(t) \longrightarrow \Dot{y}(t) + 5y(t) = x(t)$ \vspace{0.3cm}
    \item %write the solution of q1b
    $x(t) = (e^{-t} + e^{-3t})u(t), y(t) = y_p(t) + y_h(t)$ \vspace{0.2cm}\\
    $y_p(t) = Kx(t) = K(e^{-t} + e^{-3t})u(t) \longrightarrow \frac{d(K(e^{-t} + e^{-3t})u(t)}{dt} + 5K(e^{-t} + e^{-3t})u(t) = (e^{-t} + e^{-3t})u(t)$ \vspace{0.1cm}\\
    $\frac{d(K(e^{-t} + e^{-3t})u(t)}{dt} = K(-e^{-t}-3e^{-3t})u(t)$ \vspace{0.1cm} \\
    $Ku(t)(-e^{-t}-3e^{-3t}+5e^{-t}+5e^{-3t}) = (e^{-t} + e^{-3t})u(t)$ \vspace{0.1cm} \\
    $K(4e^{-t}+2e^{-3t}) = e^{-t} + e^{-3t}, K = \frac{e^{-t}+e^{-3t}}{4e^{-t}+2e^{-3t}} \longrightarrow y_p(t) = \frac{e^{-2t}+2e^{-4t}+e^{-6t}}{4e^{-t}+2e^{-3t}}$ \vspace{0.2cm} \\

    $y_h(t) = Ce^{\alpha t} \longrightarrow C\alpha e^{\alpha t} + 5Ce^{\alpha t} = 0$ \vspace{0.1cm} \\
    $C(\alpha + 5)e^{\alpha t} = 0 \longrightarrow \alpha = -5, y_h(t) = Ce^{-5t}$ \vspace{0.2cm} \\

    $y(t) = \frac{e^{-2t}+2e^{-4t}+e^{-6t}}{4e^{-t}+2e^{-3t}}u(t) + Ce^{-5t}$ \vspace{0.1cm} \\
    $y(0) = 0 \longrightarrow \frac{4}{6} + C = 0, C = -\frac{2}{3}$ \vspace{0.1cm} \\
    $y(t) =  \frac{e^{-2t}+2e^{-4t}+e^{-6t}}{4e^{-t}+2e^{-3t}}u(t)-\frac{2}{3}e^{-5t}$ \vspace{0.3cm}\\
    \end{enumerate}

\item %write the solution of q2  
	\begin{enumerate}
    % Write your solutions in the following items.
    \item %write the solution of q2a
    $x[n] = 2\delta[n] + \delta[n+1], h[n] = \delta[n-1] + 2\delta[n+1]$ \vspace{0.1cm} \\
    $h_1[n] = \delta[n-1], h_2[n] = \delta[n+1] \longrightarrow h[n] = h_1[n] + h_2[n] + h_2[n]$ \vspace{0.1cm} \\
    $x[n]*h[n] = x[n](h_1[n] + h_2[n] + h_2[n]) = x[n]*h_1[n]+x[n]*h_2[n]+x[n]*h_2[n]$ from distributivity. \vspace{0.2cm} \\
    $x[n]*h_1[n] = x[n-1] = 2\delta[n-1]+\delta[n]$ \vspace{0.1cm} \\
    $x[n]*h_2[n] = x[n+1] = 2\delta[n+1] + \delta[n+2]$ \vspace{0.1cm} \\
    $y[n] = 2\delta[n-1] + \delta[n] + 2\delta[n+1] + \delta[n+2]$ \vspace{0.2cm} \\

    
    \begin{filecontents}{data.dat}
     n   y[n] 
    -2   1.0
    -1   2.0
     0   1.0  
     1   2.0  
    \end{filecontents}


    \begin{tikzpicture}
      \begin{axis}[
          axis lines=middle,
          xlabel={$n$},
          ylabel={$\boldsymbol{y[n]}$},
          xtick={-3,-2,-1, ..., 3},
          xmin=-3, xmax=3,
          ymin=-1, ymax=3,
          xticklabel style={
            anchor=south,
            inner sep=5pt,
          }
        ]
        \addplot [ycomb, black, thick, mark=*] table [x={n}, y={y[n]}] {data.dat};
      \end{axis}
    \end{tikzpicture} 
    \vspace{0.8cm} \\
    
    \item %write the solution of q2b
    $x(t)=u(t-1)+u(t+1), h(t)=e^{-t}sin(t)u(t), y(t)=\frac{dx(t)}{dt}*h(t)$ \vspace{0.1cm} \\
    $\Dot{x}(t)=\delta(t-1) + \delta(t+1)$ \vspace{0.1cm} \\
    $x_1(t) = \delta(t-1), x_2(t)=\delta(t+1) \longrightarrow \Dot{x}(t) = x_1(t) + x_2(t)$ \vspace{0.2cm} \\
    $\Dot{x}(t)*h(t)=h(t)*\Dot{x}$ (from commutativity), $h(t)*\Dot{x}(t)=h(t)*x_1(t)+h(t)*x_2(t)$ (from distributivity) \vspace{0.2cm} \\
    $h(t)*x_1(t) = h(t-1), h(t)*x_2(t) = h(t+1)$ \vspace{0.1cm} \\
    $y(t) = h(t-1)+h(t+1) = e^{1-t}sin(t-1)u(t-1)+e^{-1-t}sin(t+1)u(t+1)$
    \end{enumerate}
    \vspace{0.3cm} 

\item %write the solution of q3
    \begin{enumerate}
    % Write your solutions in the following items.
    \item %write the solution of q3a
    $h(t) = e^{-2t}u(t), x(t) = e^{-t} y(t)=x(t)*h(t)$ \vspace{0.1cm} \\
    $x(t)*h(t) = \int_{-\infty}^{\infty} x(\tau)h(t-\tau)d\tau = \int_{-\infty}^{\infty}x(t-\tau)h(\tau)d\tau$ \vspace{0.1cm} \\
    $= \int_{0}^{t}e^{\tau-t}.e^{-2\tau}d\tau$ \vspace{0.1cm} \\
    $= e^{-t}\int_{0}^{t}e^{-\tau}d\tau$ \vspace{0.1cm} \\
    $= e^{-t}(-e^{-T})|^t_0$ \vspace{0.1cm} \\
    $e^{-t}(1-e^{-t})u(t)$ \vspace{0.3cm} \\
    \item %write the solution of q3b
    $h(t)=e^{3t}u(t), x(t) = u(t)-u(t-1) y(t) = x(t)*h(t)$ \vspace{0.1cm} \\
    $x(t) = u(t) - u(t-1) = \delta(t)$ \vspace{0.1cm} \\
    $h(t)*x(t) = h(t)*\delta(t) = h(t) = e^{3t}u(t)$ \vspace{0.3cm} \\
    \end{enumerate}

\item %write the solution of q4
    \begin{enumerate}   
    % Write your solutions in the following items.
    \item %write the solution of q4a
        Guess $y[n] = Cz^n$:
        \begin{equation*}
            Cz^n-Cz^{n-1}-Cz^{n-2} = 0
        \end{equation*}
        \begin{equation*}
            Kz^{n-2}(z^2-z-1) = 0
        \end{equation*}
        \begin{equation*}
            (z^2-z-1) = 0
        \end{equation*}
        \begin{equation*}
            z_{1,2} = \frac{1 \pm \sqrt{5}}{2}
        \end{equation*}
            
        $y[n]$ is the linear combination of solutions for $z_1$ and $z_2$.

        \begin{equation*}
            y[n] = C_1 \left ( \frac{1 + \sqrt{5}}{2} \right )^n + C_2 \left (\frac{1 - \sqrt{5}}{2} \right )^n
        \end{equation*}

        By the initial conditions $y[0] = 1$ and $y[1] = 1$, coefficients $C_1$ and $C_2$ can be calculated.

        \begin{equation*}
            C_1 = \frac{1 + \sqrt{5}}{10}, \ \ C_2 = \frac{9-\sqrt{5}}{2}
        \end{equation*}
        \begin{equation*}
            y[n] = \frac{1 + \sqrt{5}}{10}\left ( \frac{1 + \sqrt{5}}{2} \right )^n +\frac{9-\sqrt{5}}{2}\left (\frac{1 - \sqrt{5}}{2} \right )^n
        \end{equation*}
    \item %write the solution of q4b
        Assume $y(t) = Ce^{\alpha t}$, then characteristic equation will be:
        \begin{equation*}
            \alpha^3-6\alpha^2+13\alpha-10 = 0
        \end{equation*}
        \begin{equation*}
            \alpha_1 = 2, \ \ \alpha_{2,3} = 2 \pm j
        \end{equation*}

        For the imaginary part, there are 2 unique solutions.

        \begin{equation*}
            y'_1(t) = e^{2t}e^{jt} = cos(t) + jsin(t)
        \end{equation*}
        \begin{equation*}
            y'_2(t) = e^{2t}e^{-jt} = cos(t) - jsin(t)
        \end{equation*}

        \begin{equation*}
            y_1(t) = \frac{1}{2} y'_1(t)+\frac{1}{2}y'_2(t) = e^{2t}cos(t)
        \end{equation*}
        \begin{equation*}
            y_2(t)= \frac{1}{2j} y'_1(t)-\frac{1}{2j}y'_2(t) = e^{2t}sin(t)
        \end{equation*}

        Solution for the imaginary part is the linear combination of these 2 unique solutions.

        \begin{equation*}
            C_2y_1(t)+C_3y_2(t)
        \end{equation*}

        \begin{equation*}
            y(t) = C_1e^{2t} + C_2e^{2t}cos(t) + C_3e^{2t}sin(t)
        \end{equation*}

        By the initial conditions $y''(0) = 3$, $y'(0) = 1.5$, $y(0) = 1$, solution is:

        \begin{equation*}
            y(t) = 2e^{2t} - 1e^{2t}cos(t) - 0.5e^{2t}sin(t)
        \end{equation*}

    \end{enumerate}

\item %write the solution of q5
    \begin{enumerate}
    % Write your solutions in the following items.
    \item %write the solution of q5a
        Guess $y_p(t) = H(\lambda)cos(5t) = H(\lambda)(0.5e^{5jt} +0.5e^{-5jt})$. Using the linearity property solutions is in the form of:
        \begin{equation*}
            y_p(t) = 0.5 H(5j)e^{5jt} + 0.5 H(-5j)e^{-5jt}
        \end{equation*}

        where $H(\lambda)$ corresponds to the transfer function. To find the transfer function assume, input is $x(t) = e^{\lambda t}$. 
        Then output will be in the form of $H(\lambda)e^{\lambda t}$

        \begin{equation*}
            H(\lambda)e^{\lambda t} (\lambda^2+5\lambda+6)=e^{\lambda t}
        \end{equation*}
        \begin{equation*}
            H(\lambda) = \frac{1}{\lambda^2+5\lambda+6}
        \end{equation*}

        \begin{equation*}
            H(5j) = \frac{1}{-19+25j}, \ \ H(-5j) = \frac{1}{-19-25j}
        \end{equation*}

        \begin{equation*}
            y_p(t) = \frac{1}{-38+50j}e^{5jt} + \frac{1}{-38-50j}e^{-5jt}
        \end{equation*}
    \item %write the solution of q5b
        Characteristic equation of the systems is:

        \begin{equation*}
            \alpha^2+5\alpha+6 = 0
        \end{equation*}
        \begin{equation*}
            \alpha_1 = -3, \ \ \alpha_2=-2
        \end{equation*}

        Homogenious solution is $y_h(t) = C_1e^{-3t} + C_2e^{-2t}$
	\item %write the solution of q5c
        Since the system is rest initially, the homogenious solution is $0$. Then the general solution equals to:

        \begin{equation*}
            y(t) = \frac{1}{-38+50j}e^{5jt} + \frac{1}{-38-50j}e^{-5jt}
        \end{equation*}
        
    \end{enumerate}
    
\item %write the solution of q6
    \begin{enumerate}
    % Write your solutions in the following items.
    \item %write the solution of q6a
        $x[n] = \delta [n]$

        \begin{equation*}
            w[n] = \delta [n] + \frac{1}{2} w[n-1]
        \end{equation*}
        \begin{equation*}
            w[0] = \delta [0] + \frac{1}{2} w[-1] = 1
        \end{equation*}
        \begin{equation*}
            w[1] = \delta [1] + \frac{1}{2} w[0] = \frac{1}{2}
        \end{equation*}
        \begin{equation*}
            w[2] = \delta [2] + \frac{1}{2} w[1] = \frac{1}{4}
        \end{equation*}
        \begin{equation*}
            h_0[n] = w[n] = \left ( \frac{1}{2} \right )^n u[n]
        \end{equation*}
    \item %write the solution of q6b
        \begin{equation*}
            h[n] = h_0[n] * h_0[n]
        \end{equation*}
        \begin{equation*}
            h[n] = \sum_{k=-\infty}^{\infty} \left ( \frac{1}{2} \right )^k u[k] \left ( \frac{1}{2} \right )^{n-k} u[n-k]
        \end{equation*}
        \begin{equation*}
            h[n] = \sum_{k=0}^{n} \left ( \frac{1}{2} \right )^k \left ( \frac{1}{2} \right )^{n-k}
        \end{equation*}
        \begin{equation*}
            h[n] = \left ( \frac{1}{2} \right )^{n} \sum_{k=0}^{n} \left ( \frac{1}{2} \right )^k \left ( \frac{1}{2} \right )^{-k}
        \end{equation*}
        \begin{equation*}
            h[n] = \left ( \frac{1}{2} \right )^{n} \sum_{k=0}^{n} 1 = \left ( \frac{1}{2} \right )^{n}u[n]n
        \end{equation*}
	\item %write the solution of q6c
        \begin{equation*}
            y[n] = h[n] * x[n]
        \end{equation*}
        \begin{equation*}
            y[n] = \sum_{k=-\infty}^{\infty} \left ( \frac{1}{2} \right )^{k}ku[k]x[n-k]
        \end{equation*}
        \begin{equation*}
            y[n] = \sum_{k=0}^{\infty} \left ( \frac{1}{2} \right )^{k}kx[n-k]
        \end{equation*}
    \end{enumerate}
    
\item %write the solution of q7
    \begin{enumerate}
    % Write your solutions in the following items.
    \item %write the solution of q7a
        \begin{figure}[H]
            %% Creator: Matplotlib, PGF backend
%%
%% To include the figure in your LaTeX document, write
%%   \input{<filename>.pgf}
%%
%% Make sure the required packages are loaded in your preamble
%%   \usepackage{pgf}
%%
%% Also ensure that all the required font packages are loaded; for instance,
%% the lmodern package is sometimes necessary when using math font.
%%   \usepackage{lmodern}
%%
%% Figures using additional raster images can only be included by \input if
%% they are in the same directory as the main LaTeX file. For loading figures
%% from other directories you can use the `import` package
%%   \usepackage{import}
%%
%% and then include the figures with
%%   \import{<path to file>}{<filename>.pgf}
%%
%% Matplotlib used the following preamble
%%   
%%   \usepackage{fontspec}
%%   \setmainfont{DejaVuSerif.ttf}[Path=\detokenize{/home/aeren/.local/lib/python3.10/site-packages/matplotlib/mpl-data/fonts/ttf/}]
%%   \setsansfont{DejaVuSans.ttf}[Path=\detokenize{/home/aeren/.local/lib/python3.10/site-packages/matplotlib/mpl-data/fonts/ttf/}]
%%   \setmonofont{DejaVuSansMono.ttf}[Path=\detokenize{/home/aeren/.local/lib/python3.10/site-packages/matplotlib/mpl-data/fonts/ttf/}]
%%   \makeatletter\@ifpackageloaded{underscore}{}{\usepackage[strings]{underscore}}\makeatother
%%
\begingroup%
\makeatletter%
\begin{pgfpicture}%
\pgfpathrectangle{\pgfpointorigin}{\pgfqpoint{6.400000in}{5.160000in}}%
\pgfusepath{use as bounding box, clip}%
\begin{pgfscope}%
\pgfsetbuttcap%
\pgfsetmiterjoin%
\definecolor{currentfill}{rgb}{1.000000,1.000000,1.000000}%
\pgfsetfillcolor{currentfill}%
\pgfsetlinewidth{0.000000pt}%
\definecolor{currentstroke}{rgb}{1.000000,1.000000,1.000000}%
\pgfsetstrokecolor{currentstroke}%
\pgfsetdash{}{0pt}%
\pgfpathmoveto{\pgfqpoint{0.000000in}{0.000000in}}%
\pgfpathlineto{\pgfqpoint{6.400000in}{0.000000in}}%
\pgfpathlineto{\pgfqpoint{6.400000in}{5.160000in}}%
\pgfpathlineto{\pgfqpoint{0.000000in}{5.160000in}}%
\pgfpathlineto{\pgfqpoint{0.000000in}{0.000000in}}%
\pgfpathclose%
\pgfusepath{fill}%
\end{pgfscope}%
\begin{pgfscope}%
\pgfsetbuttcap%
\pgfsetmiterjoin%
\definecolor{currentfill}{rgb}{1.000000,1.000000,1.000000}%
\pgfsetfillcolor{currentfill}%
\pgfsetlinewidth{0.000000pt}%
\definecolor{currentstroke}{rgb}{0.000000,0.000000,0.000000}%
\pgfsetstrokecolor{currentstroke}%
\pgfsetstrokeopacity{0.000000}%
\pgfsetdash{}{0pt}%
\pgfpathmoveto{\pgfqpoint{0.800000in}{0.567600in}}%
\pgfpathlineto{\pgfqpoint{5.760000in}{0.567600in}}%
\pgfpathlineto{\pgfqpoint{5.760000in}{4.540800in}}%
\pgfpathlineto{\pgfqpoint{0.800000in}{4.540800in}}%
\pgfpathlineto{\pgfqpoint{0.800000in}{0.567600in}}%
\pgfpathclose%
\pgfusepath{fill}%
\end{pgfscope}%
\begin{pgfscope}%
\pgfsetbuttcap%
\pgfsetroundjoin%
\definecolor{currentfill}{rgb}{0.000000,0.000000,0.000000}%
\pgfsetfillcolor{currentfill}%
\pgfsetlinewidth{0.803000pt}%
\definecolor{currentstroke}{rgb}{0.000000,0.000000,0.000000}%
\pgfsetstrokecolor{currentstroke}%
\pgfsetdash{}{0pt}%
\pgfsys@defobject{currentmarker}{\pgfqpoint{0.000000in}{-0.048611in}}{\pgfqpoint{0.000000in}{0.000000in}}{%
\pgfpathmoveto{\pgfqpoint{0.000000in}{0.000000in}}%
\pgfpathlineto{\pgfqpoint{0.000000in}{-0.048611in}}%
\pgfusepath{stroke,fill}%
}%
\begin{pgfscope}%
\pgfsys@transformshift{1.025455in}{0.567600in}%
\pgfsys@useobject{currentmarker}{}%
\end{pgfscope}%
\end{pgfscope}%
\begin{pgfscope}%
\definecolor{textcolor}{rgb}{0.000000,0.000000,0.000000}%
\pgfsetstrokecolor{textcolor}%
\pgfsetfillcolor{textcolor}%
\pgftext[x=1.025455in,y=0.470378in,,top]{\color{textcolor}\sffamily\fontsize{10.000000}{12.000000}\selectfont \ensuremath{-}20}%
\end{pgfscope}%
\begin{pgfscope}%
\pgfsetbuttcap%
\pgfsetroundjoin%
\definecolor{currentfill}{rgb}{0.000000,0.000000,0.000000}%
\pgfsetfillcolor{currentfill}%
\pgfsetlinewidth{0.803000pt}%
\definecolor{currentstroke}{rgb}{0.000000,0.000000,0.000000}%
\pgfsetstrokecolor{currentstroke}%
\pgfsetdash{}{0pt}%
\pgfsys@defobject{currentmarker}{\pgfqpoint{0.000000in}{-0.048611in}}{\pgfqpoint{0.000000in}{0.000000in}}{%
\pgfpathmoveto{\pgfqpoint{0.000000in}{0.000000in}}%
\pgfpathlineto{\pgfqpoint{0.000000in}{-0.048611in}}%
\pgfusepath{stroke,fill}%
}%
\begin{pgfscope}%
\pgfsys@transformshift{1.783285in}{0.567600in}%
\pgfsys@useobject{currentmarker}{}%
\end{pgfscope}%
\end{pgfscope}%
\begin{pgfscope}%
\definecolor{textcolor}{rgb}{0.000000,0.000000,0.000000}%
\pgfsetstrokecolor{textcolor}%
\pgfsetfillcolor{textcolor}%
\pgftext[x=1.783285in,y=0.470378in,,top]{\color{textcolor}\sffamily\fontsize{10.000000}{12.000000}\selectfont 0}%
\end{pgfscope}%
\begin{pgfscope}%
\pgfsetbuttcap%
\pgfsetroundjoin%
\definecolor{currentfill}{rgb}{0.000000,0.000000,0.000000}%
\pgfsetfillcolor{currentfill}%
\pgfsetlinewidth{0.803000pt}%
\definecolor{currentstroke}{rgb}{0.000000,0.000000,0.000000}%
\pgfsetstrokecolor{currentstroke}%
\pgfsetdash{}{0pt}%
\pgfsys@defobject{currentmarker}{\pgfqpoint{0.000000in}{-0.048611in}}{\pgfqpoint{0.000000in}{0.000000in}}{%
\pgfpathmoveto{\pgfqpoint{0.000000in}{0.000000in}}%
\pgfpathlineto{\pgfqpoint{0.000000in}{-0.048611in}}%
\pgfusepath{stroke,fill}%
}%
\begin{pgfscope}%
\pgfsys@transformshift{2.541115in}{0.567600in}%
\pgfsys@useobject{currentmarker}{}%
\end{pgfscope}%
\end{pgfscope}%
\begin{pgfscope}%
\definecolor{textcolor}{rgb}{0.000000,0.000000,0.000000}%
\pgfsetstrokecolor{textcolor}%
\pgfsetfillcolor{textcolor}%
\pgftext[x=2.541115in,y=0.470378in,,top]{\color{textcolor}\sffamily\fontsize{10.000000}{12.000000}\selectfont 20}%
\end{pgfscope}%
\begin{pgfscope}%
\pgfsetbuttcap%
\pgfsetroundjoin%
\definecolor{currentfill}{rgb}{0.000000,0.000000,0.000000}%
\pgfsetfillcolor{currentfill}%
\pgfsetlinewidth{0.803000pt}%
\definecolor{currentstroke}{rgb}{0.000000,0.000000,0.000000}%
\pgfsetstrokecolor{currentstroke}%
\pgfsetdash{}{0pt}%
\pgfsys@defobject{currentmarker}{\pgfqpoint{0.000000in}{-0.048611in}}{\pgfqpoint{0.000000in}{0.000000in}}{%
\pgfpathmoveto{\pgfqpoint{0.000000in}{0.000000in}}%
\pgfpathlineto{\pgfqpoint{0.000000in}{-0.048611in}}%
\pgfusepath{stroke,fill}%
}%
\begin{pgfscope}%
\pgfsys@transformshift{3.298946in}{0.567600in}%
\pgfsys@useobject{currentmarker}{}%
\end{pgfscope}%
\end{pgfscope}%
\begin{pgfscope}%
\definecolor{textcolor}{rgb}{0.000000,0.000000,0.000000}%
\pgfsetstrokecolor{textcolor}%
\pgfsetfillcolor{textcolor}%
\pgftext[x=3.298946in,y=0.470378in,,top]{\color{textcolor}\sffamily\fontsize{10.000000}{12.000000}\selectfont 40}%
\end{pgfscope}%
\begin{pgfscope}%
\pgfsetbuttcap%
\pgfsetroundjoin%
\definecolor{currentfill}{rgb}{0.000000,0.000000,0.000000}%
\pgfsetfillcolor{currentfill}%
\pgfsetlinewidth{0.803000pt}%
\definecolor{currentstroke}{rgb}{0.000000,0.000000,0.000000}%
\pgfsetstrokecolor{currentstroke}%
\pgfsetdash{}{0pt}%
\pgfsys@defobject{currentmarker}{\pgfqpoint{0.000000in}{-0.048611in}}{\pgfqpoint{0.000000in}{0.000000in}}{%
\pgfpathmoveto{\pgfqpoint{0.000000in}{0.000000in}}%
\pgfpathlineto{\pgfqpoint{0.000000in}{-0.048611in}}%
\pgfusepath{stroke,fill}%
}%
\begin{pgfscope}%
\pgfsys@transformshift{4.056776in}{0.567600in}%
\pgfsys@useobject{currentmarker}{}%
\end{pgfscope}%
\end{pgfscope}%
\begin{pgfscope}%
\definecolor{textcolor}{rgb}{0.000000,0.000000,0.000000}%
\pgfsetstrokecolor{textcolor}%
\pgfsetfillcolor{textcolor}%
\pgftext[x=4.056776in,y=0.470378in,,top]{\color{textcolor}\sffamily\fontsize{10.000000}{12.000000}\selectfont 60}%
\end{pgfscope}%
\begin{pgfscope}%
\pgfsetbuttcap%
\pgfsetroundjoin%
\definecolor{currentfill}{rgb}{0.000000,0.000000,0.000000}%
\pgfsetfillcolor{currentfill}%
\pgfsetlinewidth{0.803000pt}%
\definecolor{currentstroke}{rgb}{0.000000,0.000000,0.000000}%
\pgfsetstrokecolor{currentstroke}%
\pgfsetdash{}{0pt}%
\pgfsys@defobject{currentmarker}{\pgfqpoint{0.000000in}{-0.048611in}}{\pgfqpoint{0.000000in}{0.000000in}}{%
\pgfpathmoveto{\pgfqpoint{0.000000in}{0.000000in}}%
\pgfpathlineto{\pgfqpoint{0.000000in}{-0.048611in}}%
\pgfusepath{stroke,fill}%
}%
\begin{pgfscope}%
\pgfsys@transformshift{4.814607in}{0.567600in}%
\pgfsys@useobject{currentmarker}{}%
\end{pgfscope}%
\end{pgfscope}%
\begin{pgfscope}%
\definecolor{textcolor}{rgb}{0.000000,0.000000,0.000000}%
\pgfsetstrokecolor{textcolor}%
\pgfsetfillcolor{textcolor}%
\pgftext[x=4.814607in,y=0.470378in,,top]{\color{textcolor}\sffamily\fontsize{10.000000}{12.000000}\selectfont 80}%
\end{pgfscope}%
\begin{pgfscope}%
\pgfsetbuttcap%
\pgfsetroundjoin%
\definecolor{currentfill}{rgb}{0.000000,0.000000,0.000000}%
\pgfsetfillcolor{currentfill}%
\pgfsetlinewidth{0.803000pt}%
\definecolor{currentstroke}{rgb}{0.000000,0.000000,0.000000}%
\pgfsetstrokecolor{currentstroke}%
\pgfsetdash{}{0pt}%
\pgfsys@defobject{currentmarker}{\pgfqpoint{0.000000in}{-0.048611in}}{\pgfqpoint{0.000000in}{0.000000in}}{%
\pgfpathmoveto{\pgfqpoint{0.000000in}{0.000000in}}%
\pgfpathlineto{\pgfqpoint{0.000000in}{-0.048611in}}%
\pgfusepath{stroke,fill}%
}%
\begin{pgfscope}%
\pgfsys@transformshift{5.572437in}{0.567600in}%
\pgfsys@useobject{currentmarker}{}%
\end{pgfscope}%
\end{pgfscope}%
\begin{pgfscope}%
\definecolor{textcolor}{rgb}{0.000000,0.000000,0.000000}%
\pgfsetstrokecolor{textcolor}%
\pgfsetfillcolor{textcolor}%
\pgftext[x=5.572437in,y=0.470378in,,top]{\color{textcolor}\sffamily\fontsize{10.000000}{12.000000}\selectfont 100}%
\end{pgfscope}%
\begin{pgfscope}%
\definecolor{textcolor}{rgb}{0.000000,0.000000,0.000000}%
\pgfsetstrokecolor{textcolor}%
\pgfsetfillcolor{textcolor}%
\pgftext[x=3.280000in,y=0.280409in,,top]{\color{textcolor}\sffamily\fontsize{10.000000}{12.000000}\selectfont n}%
\end{pgfscope}%
\begin{pgfscope}%
\pgfsetbuttcap%
\pgfsetroundjoin%
\definecolor{currentfill}{rgb}{0.000000,0.000000,0.000000}%
\pgfsetfillcolor{currentfill}%
\pgfsetlinewidth{0.803000pt}%
\definecolor{currentstroke}{rgb}{0.000000,0.000000,0.000000}%
\pgfsetstrokecolor{currentstroke}%
\pgfsetdash{}{0pt}%
\pgfsys@defobject{currentmarker}{\pgfqpoint{-0.048611in}{0.000000in}}{\pgfqpoint{-0.000000in}{0.000000in}}{%
\pgfpathmoveto{\pgfqpoint{-0.000000in}{0.000000in}}%
\pgfpathlineto{\pgfqpoint{-0.048611in}{0.000000in}}%
\pgfusepath{stroke,fill}%
}%
\begin{pgfscope}%
\pgfsys@transformshift{0.800000in}{0.984635in}%
\pgfsys@useobject{currentmarker}{}%
\end{pgfscope}%
\end{pgfscope}%
\begin{pgfscope}%
\definecolor{textcolor}{rgb}{0.000000,0.000000,0.000000}%
\pgfsetstrokecolor{textcolor}%
\pgfsetfillcolor{textcolor}%
\pgftext[x=0.373873in, y=0.931874in, left, base]{\color{textcolor}\sffamily\fontsize{10.000000}{12.000000}\selectfont \ensuremath{-}1.0}%
\end{pgfscope}%
\begin{pgfscope}%
\pgfsetbuttcap%
\pgfsetroundjoin%
\definecolor{currentfill}{rgb}{0.000000,0.000000,0.000000}%
\pgfsetfillcolor{currentfill}%
\pgfsetlinewidth{0.803000pt}%
\definecolor{currentstroke}{rgb}{0.000000,0.000000,0.000000}%
\pgfsetstrokecolor{currentstroke}%
\pgfsetdash{}{0pt}%
\pgfsys@defobject{currentmarker}{\pgfqpoint{-0.048611in}{0.000000in}}{\pgfqpoint{-0.000000in}{0.000000in}}{%
\pgfpathmoveto{\pgfqpoint{-0.000000in}{0.000000in}}%
\pgfpathlineto{\pgfqpoint{-0.048611in}{0.000000in}}%
\pgfusepath{stroke,fill}%
}%
\begin{pgfscope}%
\pgfsys@transformshift{0.800000in}{1.763952in}%
\pgfsys@useobject{currentmarker}{}%
\end{pgfscope}%
\end{pgfscope}%
\begin{pgfscope}%
\definecolor{textcolor}{rgb}{0.000000,0.000000,0.000000}%
\pgfsetstrokecolor{textcolor}%
\pgfsetfillcolor{textcolor}%
\pgftext[x=0.373873in, y=1.711190in, left, base]{\color{textcolor}\sffamily\fontsize{10.000000}{12.000000}\selectfont \ensuremath{-}0.5}%
\end{pgfscope}%
\begin{pgfscope}%
\pgfsetbuttcap%
\pgfsetroundjoin%
\definecolor{currentfill}{rgb}{0.000000,0.000000,0.000000}%
\pgfsetfillcolor{currentfill}%
\pgfsetlinewidth{0.803000pt}%
\definecolor{currentstroke}{rgb}{0.000000,0.000000,0.000000}%
\pgfsetstrokecolor{currentstroke}%
\pgfsetdash{}{0pt}%
\pgfsys@defobject{currentmarker}{\pgfqpoint{-0.048611in}{0.000000in}}{\pgfqpoint{-0.000000in}{0.000000in}}{%
\pgfpathmoveto{\pgfqpoint{-0.000000in}{0.000000in}}%
\pgfpathlineto{\pgfqpoint{-0.048611in}{0.000000in}}%
\pgfusepath{stroke,fill}%
}%
\begin{pgfscope}%
\pgfsys@transformshift{0.800000in}{2.543268in}%
\pgfsys@useobject{currentmarker}{}%
\end{pgfscope}%
\end{pgfscope}%
\begin{pgfscope}%
\definecolor{textcolor}{rgb}{0.000000,0.000000,0.000000}%
\pgfsetstrokecolor{textcolor}%
\pgfsetfillcolor{textcolor}%
\pgftext[x=0.481898in, y=2.490507in, left, base]{\color{textcolor}\sffamily\fontsize{10.000000}{12.000000}\selectfont 0.0}%
\end{pgfscope}%
\begin{pgfscope}%
\pgfsetbuttcap%
\pgfsetroundjoin%
\definecolor{currentfill}{rgb}{0.000000,0.000000,0.000000}%
\pgfsetfillcolor{currentfill}%
\pgfsetlinewidth{0.803000pt}%
\definecolor{currentstroke}{rgb}{0.000000,0.000000,0.000000}%
\pgfsetstrokecolor{currentstroke}%
\pgfsetdash{}{0pt}%
\pgfsys@defobject{currentmarker}{\pgfqpoint{-0.048611in}{0.000000in}}{\pgfqpoint{-0.000000in}{0.000000in}}{%
\pgfpathmoveto{\pgfqpoint{-0.000000in}{0.000000in}}%
\pgfpathlineto{\pgfqpoint{-0.048611in}{0.000000in}}%
\pgfusepath{stroke,fill}%
}%
\begin{pgfscope}%
\pgfsys@transformshift{0.800000in}{3.322585in}%
\pgfsys@useobject{currentmarker}{}%
\end{pgfscope}%
\end{pgfscope}%
\begin{pgfscope}%
\definecolor{textcolor}{rgb}{0.000000,0.000000,0.000000}%
\pgfsetstrokecolor{textcolor}%
\pgfsetfillcolor{textcolor}%
\pgftext[x=0.481898in, y=3.269823in, left, base]{\color{textcolor}\sffamily\fontsize{10.000000}{12.000000}\selectfont 0.5}%
\end{pgfscope}%
\begin{pgfscope}%
\pgfsetbuttcap%
\pgfsetroundjoin%
\definecolor{currentfill}{rgb}{0.000000,0.000000,0.000000}%
\pgfsetfillcolor{currentfill}%
\pgfsetlinewidth{0.803000pt}%
\definecolor{currentstroke}{rgb}{0.000000,0.000000,0.000000}%
\pgfsetstrokecolor{currentstroke}%
\pgfsetdash{}{0pt}%
\pgfsys@defobject{currentmarker}{\pgfqpoint{-0.048611in}{0.000000in}}{\pgfqpoint{-0.000000in}{0.000000in}}{%
\pgfpathmoveto{\pgfqpoint{-0.000000in}{0.000000in}}%
\pgfpathlineto{\pgfqpoint{-0.048611in}{0.000000in}}%
\pgfusepath{stroke,fill}%
}%
\begin{pgfscope}%
\pgfsys@transformshift{0.800000in}{4.101901in}%
\pgfsys@useobject{currentmarker}{}%
\end{pgfscope}%
\end{pgfscope}%
\begin{pgfscope}%
\definecolor{textcolor}{rgb}{0.000000,0.000000,0.000000}%
\pgfsetstrokecolor{textcolor}%
\pgfsetfillcolor{textcolor}%
\pgftext[x=0.481898in, y=4.049140in, left, base]{\color{textcolor}\sffamily\fontsize{10.000000}{12.000000}\selectfont 1.0}%
\end{pgfscope}%
\begin{pgfscope}%
\definecolor{textcolor}{rgb}{0.000000,0.000000,0.000000}%
\pgfsetstrokecolor{textcolor}%
\pgfsetfillcolor{textcolor}%
\pgftext[x=0.318318in,y=2.554200in,,bottom,rotate=90.000000]{\color{textcolor}\sffamily\fontsize{10.000000}{12.000000}\selectfont x[n] * h[n]}%
\end{pgfscope}%
\begin{pgfscope}%
\pgfpathrectangle{\pgfqpoint{0.800000in}{0.567600in}}{\pgfqpoint{4.960000in}{3.973200in}}%
\pgfusepath{clip}%
\pgfsetrectcap%
\pgfsetroundjoin%
\pgfsetlinewidth{1.505625pt}%
\definecolor{currentstroke}{rgb}{1.000000,0.000000,0.000000}%
\pgfsetstrokecolor{currentstroke}%
\pgfsetdash{}{0pt}%
\pgfpathmoveto{\pgfqpoint{1.025455in}{2.543268in}}%
\pgfpathlineto{\pgfqpoint{1.063346in}{2.543268in}}%
\pgfpathlineto{\pgfqpoint{1.101238in}{2.543268in}}%
\pgfpathlineto{\pgfqpoint{1.139129in}{2.543268in}}%
\pgfpathlineto{\pgfqpoint{1.177021in}{2.543268in}}%
\pgfpathlineto{\pgfqpoint{1.214912in}{0.923728in}}%
\pgfpathlineto{\pgfqpoint{1.252804in}{0.933215in}}%
\pgfpathlineto{\pgfqpoint{1.290695in}{1.210504in}}%
\pgfpathlineto{\pgfqpoint{1.328587in}{0.974202in}}%
\pgfpathlineto{\pgfqpoint{1.366478in}{1.105096in}}%
\pgfpathlineto{\pgfqpoint{1.404370in}{0.779696in}}%
\pgfpathlineto{\pgfqpoint{1.442261in}{0.748200in}}%
\pgfpathlineto{\pgfqpoint{1.480153in}{1.109404in}}%
\pgfpathlineto{\pgfqpoint{1.518044in}{1.222700in}}%
\pgfpathlineto{\pgfqpoint{1.555936in}{1.055000in}}%
\pgfpathlineto{\pgfqpoint{1.593827in}{1.153875in}}%
\pgfpathlineto{\pgfqpoint{1.631719in}{1.151761in}}%
\pgfpathlineto{\pgfqpoint{1.669610in}{0.887280in}}%
\pgfpathlineto{\pgfqpoint{1.707502in}{1.010430in}}%
\pgfpathlineto{\pgfqpoint{1.745393in}{0.938792in}}%
\pgfpathlineto{\pgfqpoint{1.783285in}{0.987561in}}%
\pgfpathlineto{\pgfqpoint{1.821176in}{1.188647in}}%
\pgfpathlineto{\pgfqpoint{1.859068in}{1.034356in}}%
\pgfpathlineto{\pgfqpoint{1.896960in}{1.119460in}}%
\pgfpathlineto{\pgfqpoint{1.934851in}{0.931169in}}%
\pgfpathlineto{\pgfqpoint{1.972743in}{4.351723in}}%
\pgfpathlineto{\pgfqpoint{2.010634in}{4.156286in}}%
\pgfpathlineto{\pgfqpoint{2.048526in}{3.895535in}}%
\pgfpathlineto{\pgfqpoint{2.086417in}{3.893695in}}%
\pgfpathlineto{\pgfqpoint{2.124309in}{4.000043in}}%
\pgfpathlineto{\pgfqpoint{2.162200in}{3.843055in}}%
\pgfpathlineto{\pgfqpoint{2.200092in}{4.311861in}}%
\pgfpathlineto{\pgfqpoint{2.237983in}{4.057629in}}%
\pgfpathlineto{\pgfqpoint{2.275875in}{4.315384in}}%
\pgfpathlineto{\pgfqpoint{2.313766in}{4.163039in}}%
\pgfpathlineto{\pgfqpoint{2.351658in}{3.857515in}}%
\pgfpathlineto{\pgfqpoint{2.389549in}{4.279932in}}%
\pgfpathlineto{\pgfqpoint{2.427441in}{4.075184in}}%
\pgfpathlineto{\pgfqpoint{2.465332in}{4.117517in}}%
\pgfpathlineto{\pgfqpoint{2.503224in}{3.842873in}}%
\pgfpathlineto{\pgfqpoint{2.541115in}{3.860126in}}%
\pgfpathlineto{\pgfqpoint{2.579007in}{4.090946in}}%
\pgfpathlineto{\pgfqpoint{2.616898in}{4.360200in}}%
\pgfpathlineto{\pgfqpoint{2.654790in}{4.137199in}}%
\pgfpathlineto{\pgfqpoint{2.692681in}{4.074337in}}%
\pgfpathlineto{\pgfqpoint{2.730573in}{3.914266in}}%
\pgfpathlineto{\pgfqpoint{2.768464in}{4.253384in}}%
\pgfpathlineto{\pgfqpoint{2.806356in}{4.023771in}}%
\pgfpathlineto{\pgfqpoint{2.844248in}{3.850639in}}%
\pgfpathlineto{\pgfqpoint{2.882139in}{3.863134in}}%
\pgfpathlineto{\pgfqpoint{2.920031in}{1.173387in}}%
\pgfpathlineto{\pgfqpoint{2.957922in}{0.885201in}}%
\pgfpathlineto{\pgfqpoint{2.995814in}{1.065632in}}%
\pgfpathlineto{\pgfqpoint{3.033705in}{1.242743in}}%
\pgfpathlineto{\pgfqpoint{3.071597in}{0.991142in}}%
\pgfpathlineto{\pgfqpoint{3.109488in}{0.828683in}}%
\pgfpathlineto{\pgfqpoint{3.147380in}{0.765142in}}%
\pgfpathlineto{\pgfqpoint{3.185271in}{1.107284in}}%
\pgfpathlineto{\pgfqpoint{3.223163in}{1.225867in}}%
\pgfpathlineto{\pgfqpoint{3.261054in}{1.153907in}}%
\pgfpathlineto{\pgfqpoint{3.298946in}{0.866342in}}%
\pgfpathlineto{\pgfqpoint{3.336837in}{0.899201in}}%
\pgfpathlineto{\pgfqpoint{3.374729in}{1.176446in}}%
\pgfpathlineto{\pgfqpoint{3.412620in}{0.974689in}}%
\pgfpathlineto{\pgfqpoint{3.450512in}{0.984122in}}%
\pgfpathlineto{\pgfqpoint{3.488403in}{0.946527in}}%
\pgfpathlineto{\pgfqpoint{3.526295in}{0.947323in}}%
\pgfpathlineto{\pgfqpoint{3.564186in}{0.977167in}}%
\pgfpathlineto{\pgfqpoint{3.602078in}{1.222977in}}%
\pgfpathlineto{\pgfqpoint{3.639969in}{1.000891in}}%
\pgfpathlineto{\pgfqpoint{3.677861in}{1.176011in}}%
\pgfpathlineto{\pgfqpoint{3.715752in}{0.752058in}}%
\pgfpathlineto{\pgfqpoint{3.753644in}{1.094829in}}%
\pgfpathlineto{\pgfqpoint{3.791536in}{0.886999in}}%
\pgfpathlineto{\pgfqpoint{3.829427in}{1.241542in}}%
\pgfpathlineto{\pgfqpoint{3.867319in}{4.082469in}}%
\pgfpathlineto{\pgfqpoint{3.905210in}{4.038236in}}%
\pgfpathlineto{\pgfqpoint{3.943102in}{3.987793in}}%
\pgfpathlineto{\pgfqpoint{3.980993in}{3.870820in}}%
\pgfpathlineto{\pgfqpoint{4.018885in}{4.186508in}}%
\pgfpathlineto{\pgfqpoint{4.056776in}{4.298735in}}%
\pgfpathlineto{\pgfqpoint{4.094668in}{4.254628in}}%
\pgfpathlineto{\pgfqpoint{4.132559in}{3.879858in}}%
\pgfpathlineto{\pgfqpoint{4.170451in}{4.237282in}}%
\pgfpathlineto{\pgfqpoint{4.208342in}{4.074222in}}%
\pgfpathlineto{\pgfqpoint{4.246234in}{4.128549in}}%
\pgfpathlineto{\pgfqpoint{4.284125in}{4.241124in}}%
\pgfpathlineto{\pgfqpoint{4.322017in}{4.200036in}}%
\pgfpathlineto{\pgfqpoint{4.359908in}{4.197891in}}%
\pgfpathlineto{\pgfqpoint{4.397800in}{4.240003in}}%
\pgfpathlineto{\pgfqpoint{4.435691in}{3.966526in}}%
\pgfpathlineto{\pgfqpoint{4.473583in}{4.086312in}}%
\pgfpathlineto{\pgfqpoint{4.511474in}{4.019116in}}%
\pgfpathlineto{\pgfqpoint{4.549366in}{3.965585in}}%
\pgfpathlineto{\pgfqpoint{4.587257in}{3.843736in}}%
\pgfpathlineto{\pgfqpoint{4.625149in}{3.983229in}}%
\pgfpathlineto{\pgfqpoint{4.663040in}{4.359855in}}%
\pgfpathlineto{\pgfqpoint{4.700932in}{4.323508in}}%
\pgfpathlineto{\pgfqpoint{4.738824in}{4.046876in}}%
\pgfpathlineto{\pgfqpoint{4.776715in}{4.161452in}}%
\pgfpathlineto{\pgfqpoint{4.814607in}{0.809348in}}%
\pgfpathlineto{\pgfqpoint{4.852498in}{1.219819in}}%
\pgfpathlineto{\pgfqpoint{4.890390in}{1.224359in}}%
\pgfpathlineto{\pgfqpoint{4.928281in}{0.788218in}}%
\pgfpathlineto{\pgfqpoint{4.966173in}{1.201120in}}%
\pgfpathlineto{\pgfqpoint{5.004064in}{1.222742in}}%
\pgfpathlineto{\pgfqpoint{5.041956in}{1.129014in}}%
\pgfpathlineto{\pgfqpoint{5.079847in}{0.955864in}}%
\pgfpathlineto{\pgfqpoint{5.117739in}{1.004070in}}%
\pgfpathlineto{\pgfqpoint{5.155630in}{0.901599in}}%
\pgfpathlineto{\pgfqpoint{5.193522in}{1.188729in}}%
\pgfpathlineto{\pgfqpoint{5.231413in}{1.104838in}}%
\pgfpathlineto{\pgfqpoint{5.269305in}{0.860112in}}%
\pgfpathlineto{\pgfqpoint{5.307196in}{0.797190in}}%
\pgfpathlineto{\pgfqpoint{5.345088in}{0.962181in}}%
\pgfpathlineto{\pgfqpoint{5.382979in}{1.081985in}}%
\pgfpathlineto{\pgfqpoint{5.420871in}{1.209680in}}%
\pgfpathlineto{\pgfqpoint{5.458762in}{1.136782in}}%
\pgfpathlineto{\pgfqpoint{5.496654in}{0.912625in}}%
\pgfpathlineto{\pgfqpoint{5.534545in}{0.892047in}}%
\pgfusepath{stroke}%
\end{pgfscope}%
\begin{pgfscope}%
\pgfpathrectangle{\pgfqpoint{0.800000in}{0.567600in}}{\pgfqpoint{4.960000in}{3.973200in}}%
\pgfusepath{clip}%
\pgfsetrectcap%
\pgfsetroundjoin%
\pgfsetlinewidth{1.505625pt}%
\definecolor{currentstroke}{rgb}{0.000000,0.000000,1.000000}%
\pgfsetstrokecolor{currentstroke}%
\pgfsetdash{}{0pt}%
\pgfpathmoveto{\pgfqpoint{1.025455in}{0.923728in}}%
\pgfpathlineto{\pgfqpoint{1.063346in}{0.933215in}}%
\pgfpathlineto{\pgfqpoint{1.101238in}{1.210504in}}%
\pgfpathlineto{\pgfqpoint{1.139129in}{0.974202in}}%
\pgfpathlineto{\pgfqpoint{1.177021in}{1.105096in}}%
\pgfpathlineto{\pgfqpoint{1.214912in}{0.779696in}}%
\pgfpathlineto{\pgfqpoint{1.252804in}{0.748200in}}%
\pgfpathlineto{\pgfqpoint{1.290695in}{1.109404in}}%
\pgfpathlineto{\pgfqpoint{1.328587in}{1.222700in}}%
\pgfpathlineto{\pgfqpoint{1.366478in}{1.055000in}}%
\pgfpathlineto{\pgfqpoint{1.404370in}{1.153875in}}%
\pgfpathlineto{\pgfqpoint{1.442261in}{1.151761in}}%
\pgfpathlineto{\pgfqpoint{1.480153in}{0.887280in}}%
\pgfpathlineto{\pgfqpoint{1.518044in}{1.010430in}}%
\pgfpathlineto{\pgfqpoint{1.555936in}{0.938792in}}%
\pgfpathlineto{\pgfqpoint{1.593827in}{0.987561in}}%
\pgfpathlineto{\pgfqpoint{1.631719in}{1.188647in}}%
\pgfpathlineto{\pgfqpoint{1.669610in}{1.034356in}}%
\pgfpathlineto{\pgfqpoint{1.707502in}{1.119460in}}%
\pgfpathlineto{\pgfqpoint{1.745393in}{0.931169in}}%
\pgfpathlineto{\pgfqpoint{1.783285in}{4.351723in}}%
\pgfpathlineto{\pgfqpoint{1.821176in}{4.156286in}}%
\pgfpathlineto{\pgfqpoint{1.859068in}{3.895535in}}%
\pgfpathlineto{\pgfqpoint{1.896960in}{3.893695in}}%
\pgfpathlineto{\pgfqpoint{1.934851in}{4.000043in}}%
\pgfpathlineto{\pgfqpoint{1.972743in}{3.843055in}}%
\pgfpathlineto{\pgfqpoint{2.010634in}{4.311861in}}%
\pgfpathlineto{\pgfqpoint{2.048526in}{4.057629in}}%
\pgfpathlineto{\pgfqpoint{2.086417in}{4.315384in}}%
\pgfpathlineto{\pgfqpoint{2.124309in}{4.163039in}}%
\pgfpathlineto{\pgfqpoint{2.162200in}{3.857515in}}%
\pgfpathlineto{\pgfqpoint{2.200092in}{4.279932in}}%
\pgfpathlineto{\pgfqpoint{2.237983in}{4.075184in}}%
\pgfpathlineto{\pgfqpoint{2.275875in}{4.117517in}}%
\pgfpathlineto{\pgfqpoint{2.313766in}{3.842873in}}%
\pgfpathlineto{\pgfqpoint{2.351658in}{3.860126in}}%
\pgfpathlineto{\pgfqpoint{2.389549in}{4.090946in}}%
\pgfpathlineto{\pgfqpoint{2.427441in}{4.360200in}}%
\pgfpathlineto{\pgfqpoint{2.465332in}{4.137199in}}%
\pgfpathlineto{\pgfqpoint{2.503224in}{4.074337in}}%
\pgfpathlineto{\pgfqpoint{2.541115in}{3.914266in}}%
\pgfpathlineto{\pgfqpoint{2.579007in}{4.253384in}}%
\pgfpathlineto{\pgfqpoint{2.616898in}{4.023771in}}%
\pgfpathlineto{\pgfqpoint{2.654790in}{3.850639in}}%
\pgfpathlineto{\pgfqpoint{2.692681in}{3.863134in}}%
\pgfpathlineto{\pgfqpoint{2.730573in}{1.173387in}}%
\pgfpathlineto{\pgfqpoint{2.768464in}{0.885201in}}%
\pgfpathlineto{\pgfqpoint{2.806356in}{1.065632in}}%
\pgfpathlineto{\pgfqpoint{2.844248in}{1.242743in}}%
\pgfpathlineto{\pgfqpoint{2.882139in}{0.991142in}}%
\pgfpathlineto{\pgfqpoint{2.920031in}{0.828683in}}%
\pgfpathlineto{\pgfqpoint{2.957922in}{0.765142in}}%
\pgfpathlineto{\pgfqpoint{2.995814in}{1.107284in}}%
\pgfpathlineto{\pgfqpoint{3.033705in}{1.225867in}}%
\pgfpathlineto{\pgfqpoint{3.071597in}{1.153907in}}%
\pgfpathlineto{\pgfqpoint{3.109488in}{0.866342in}}%
\pgfpathlineto{\pgfqpoint{3.147380in}{0.899201in}}%
\pgfpathlineto{\pgfqpoint{3.185271in}{1.176446in}}%
\pgfpathlineto{\pgfqpoint{3.223163in}{0.974689in}}%
\pgfpathlineto{\pgfqpoint{3.261054in}{0.984122in}}%
\pgfpathlineto{\pgfqpoint{3.298946in}{0.946527in}}%
\pgfpathlineto{\pgfqpoint{3.336837in}{0.947323in}}%
\pgfpathlineto{\pgfqpoint{3.374729in}{0.977167in}}%
\pgfpathlineto{\pgfqpoint{3.412620in}{1.222977in}}%
\pgfpathlineto{\pgfqpoint{3.450512in}{1.000891in}}%
\pgfpathlineto{\pgfqpoint{3.488403in}{1.176011in}}%
\pgfpathlineto{\pgfqpoint{3.526295in}{0.752058in}}%
\pgfpathlineto{\pgfqpoint{3.564186in}{1.094829in}}%
\pgfpathlineto{\pgfqpoint{3.602078in}{0.886999in}}%
\pgfpathlineto{\pgfqpoint{3.639969in}{1.241542in}}%
\pgfpathlineto{\pgfqpoint{3.677861in}{4.082469in}}%
\pgfpathlineto{\pgfqpoint{3.715752in}{4.038236in}}%
\pgfpathlineto{\pgfqpoint{3.753644in}{3.987793in}}%
\pgfpathlineto{\pgfqpoint{3.791536in}{3.870820in}}%
\pgfpathlineto{\pgfqpoint{3.829427in}{4.186508in}}%
\pgfpathlineto{\pgfqpoint{3.867319in}{4.298735in}}%
\pgfpathlineto{\pgfqpoint{3.905210in}{4.254628in}}%
\pgfpathlineto{\pgfqpoint{3.943102in}{3.879858in}}%
\pgfpathlineto{\pgfqpoint{3.980993in}{4.237282in}}%
\pgfpathlineto{\pgfqpoint{4.018885in}{4.074222in}}%
\pgfpathlineto{\pgfqpoint{4.056776in}{4.128549in}}%
\pgfpathlineto{\pgfqpoint{4.094668in}{4.241124in}}%
\pgfpathlineto{\pgfqpoint{4.132559in}{4.200036in}}%
\pgfpathlineto{\pgfqpoint{4.170451in}{4.197891in}}%
\pgfpathlineto{\pgfqpoint{4.208342in}{4.240003in}}%
\pgfpathlineto{\pgfqpoint{4.246234in}{3.966526in}}%
\pgfpathlineto{\pgfqpoint{4.284125in}{4.086312in}}%
\pgfpathlineto{\pgfqpoint{4.322017in}{4.019116in}}%
\pgfpathlineto{\pgfqpoint{4.359908in}{3.965585in}}%
\pgfpathlineto{\pgfqpoint{4.397800in}{3.843736in}}%
\pgfpathlineto{\pgfqpoint{4.435691in}{3.983229in}}%
\pgfpathlineto{\pgfqpoint{4.473583in}{4.359855in}}%
\pgfpathlineto{\pgfqpoint{4.511474in}{4.323508in}}%
\pgfpathlineto{\pgfqpoint{4.549366in}{4.046876in}}%
\pgfpathlineto{\pgfqpoint{4.587257in}{4.161452in}}%
\pgfpathlineto{\pgfqpoint{4.625149in}{0.809348in}}%
\pgfpathlineto{\pgfqpoint{4.663040in}{1.219819in}}%
\pgfpathlineto{\pgfqpoint{4.700932in}{1.224359in}}%
\pgfpathlineto{\pgfqpoint{4.738824in}{0.788218in}}%
\pgfpathlineto{\pgfqpoint{4.776715in}{1.201120in}}%
\pgfpathlineto{\pgfqpoint{4.814607in}{1.222742in}}%
\pgfpathlineto{\pgfqpoint{4.852498in}{1.129014in}}%
\pgfpathlineto{\pgfqpoint{4.890390in}{0.955864in}}%
\pgfpathlineto{\pgfqpoint{4.928281in}{1.004070in}}%
\pgfpathlineto{\pgfqpoint{4.966173in}{0.901599in}}%
\pgfpathlineto{\pgfqpoint{5.004064in}{1.188729in}}%
\pgfpathlineto{\pgfqpoint{5.041956in}{1.104838in}}%
\pgfpathlineto{\pgfqpoint{5.079847in}{0.860112in}}%
\pgfpathlineto{\pgfqpoint{5.117739in}{0.797190in}}%
\pgfpathlineto{\pgfqpoint{5.155630in}{0.962181in}}%
\pgfpathlineto{\pgfqpoint{5.193522in}{1.081985in}}%
\pgfpathlineto{\pgfqpoint{5.231413in}{1.209680in}}%
\pgfpathlineto{\pgfqpoint{5.269305in}{1.136782in}}%
\pgfpathlineto{\pgfqpoint{5.307196in}{0.912625in}}%
\pgfpathlineto{\pgfqpoint{5.345088in}{0.892047in}}%
\pgfpathlineto{\pgfqpoint{5.382979in}{0.877719in}}%
\pgfpathlineto{\pgfqpoint{5.420871in}{0.943551in}}%
\pgfpathlineto{\pgfqpoint{5.458762in}{0.937170in}}%
\pgfpathlineto{\pgfqpoint{5.496654in}{0.932074in}}%
\pgfpathlineto{\pgfqpoint{5.534545in}{1.190245in}}%
\pgfusepath{stroke}%
\end{pgfscope}%
\begin{pgfscope}%
\pgfsetrectcap%
\pgfsetmiterjoin%
\pgfsetlinewidth{0.803000pt}%
\definecolor{currentstroke}{rgb}{0.000000,0.000000,0.000000}%
\pgfsetstrokecolor{currentstroke}%
\pgfsetdash{}{0pt}%
\pgfpathmoveto{\pgfqpoint{0.800000in}{0.567600in}}%
\pgfpathlineto{\pgfqpoint{0.800000in}{4.540800in}}%
\pgfusepath{stroke}%
\end{pgfscope}%
\begin{pgfscope}%
\pgfsetrectcap%
\pgfsetmiterjoin%
\pgfsetlinewidth{0.803000pt}%
\definecolor{currentstroke}{rgb}{0.000000,0.000000,0.000000}%
\pgfsetstrokecolor{currentstroke}%
\pgfsetdash{}{0pt}%
\pgfpathmoveto{\pgfqpoint{5.760000in}{0.567600in}}%
\pgfpathlineto{\pgfqpoint{5.760000in}{4.540800in}}%
\pgfusepath{stroke}%
\end{pgfscope}%
\begin{pgfscope}%
\pgfsetrectcap%
\pgfsetmiterjoin%
\pgfsetlinewidth{0.803000pt}%
\definecolor{currentstroke}{rgb}{0.000000,0.000000,0.000000}%
\pgfsetstrokecolor{currentstroke}%
\pgfsetdash{}{0pt}%
\pgfpathmoveto{\pgfqpoint{0.800000in}{0.567600in}}%
\pgfpathlineto{\pgfqpoint{5.760000in}{0.567600in}}%
\pgfusepath{stroke}%
\end{pgfscope}%
\begin{pgfscope}%
\pgfsetrectcap%
\pgfsetmiterjoin%
\pgfsetlinewidth{0.803000pt}%
\definecolor{currentstroke}{rgb}{0.000000,0.000000,0.000000}%
\pgfsetstrokecolor{currentstroke}%
\pgfsetdash{}{0pt}%
\pgfpathmoveto{\pgfqpoint{0.800000in}{4.540800in}}%
\pgfpathlineto{\pgfqpoint{5.760000in}{4.540800in}}%
\pgfusepath{stroke}%
\end{pgfscope}%
\begin{pgfscope}%
\definecolor{textcolor}{rgb}{0.000000,0.000000,0.000000}%
\pgfsetstrokecolor{textcolor}%
\pgfsetfillcolor{textcolor}%
\pgftext[x=3.280000in,y=4.624133in,,base]{\color{textcolor}\sffamily\fontsize{12.000000}{14.400000}\selectfont n vs x[n] * h[n]}%
\end{pgfscope}%
\end{pgfpicture}%
\makeatother%
\endgroup%

            \centering
        \end{figure}
        

        Convolving with the $\delta[n-5]$ shifted the signal $5$ units to the right.
    \item %write the solution of q7b
        %% Creator: Matplotlib, PGF backend
%%
%% To include the figure in your LaTeX document, write
%%   \input{<filename>.pgf}
%%
%% Make sure the required packages are loaded in your preamble
%%   \usepackage{pgf}
%%
%% Also ensure that all the required font packages are loaded; for instance,
%% the lmodern package is sometimes necessary when using math font.
%%   \usepackage{lmodern}
%%
%% Figures using additional raster images can only be included by \input if
%% they are in the same directory as the main LaTeX file. For loading figures
%% from other directories you can use the `import` package
%%   \usepackage{import}
%%
%% and then include the figures with
%%   \import{<path to file>}{<filename>.pgf}
%%
%% Matplotlib used the following preamble
%%   
%%   \usepackage{fontspec}
%%   \setmainfont{DejaVuSerif.ttf}[Path=\detokenize{/home/aeren/.local/lib/python3.10/site-packages/matplotlib/mpl-data/fonts/ttf/}]
%%   \setsansfont{DejaVuSans.ttf}[Path=\detokenize{/home/aeren/.local/lib/python3.10/site-packages/matplotlib/mpl-data/fonts/ttf/}]
%%   \setmonofont{DejaVuSansMono.ttf}[Path=\detokenize{/home/aeren/.local/lib/python3.10/site-packages/matplotlib/mpl-data/fonts/ttf/}]
%%   \makeatletter\@ifpackageloaded{underscore}{}{\usepackage[strings]{underscore}}\makeatother
%%
\begingroup%
\makeatletter%
\begin{pgfpicture}%
\pgfpathrectangle{\pgfpointorigin}{\pgfqpoint{6.400000in}{5.160000in}}%
\pgfusepath{use as bounding box, clip}%
\begin{pgfscope}%
\pgfsetbuttcap%
\pgfsetmiterjoin%
\definecolor{currentfill}{rgb}{1.000000,1.000000,1.000000}%
\pgfsetfillcolor{currentfill}%
\pgfsetlinewidth{0.000000pt}%
\definecolor{currentstroke}{rgb}{1.000000,1.000000,1.000000}%
\pgfsetstrokecolor{currentstroke}%
\pgfsetdash{}{0pt}%
\pgfpathmoveto{\pgfqpoint{0.000000in}{0.000000in}}%
\pgfpathlineto{\pgfqpoint{6.400000in}{0.000000in}}%
\pgfpathlineto{\pgfqpoint{6.400000in}{5.160000in}}%
\pgfpathlineto{\pgfqpoint{0.000000in}{5.160000in}}%
\pgfpathlineto{\pgfqpoint{0.000000in}{0.000000in}}%
\pgfpathclose%
\pgfusepath{fill}%
\end{pgfscope}%
\begin{pgfscope}%
\pgfsetbuttcap%
\pgfsetmiterjoin%
\definecolor{currentfill}{rgb}{1.000000,1.000000,1.000000}%
\pgfsetfillcolor{currentfill}%
\pgfsetlinewidth{0.000000pt}%
\definecolor{currentstroke}{rgb}{0.000000,0.000000,0.000000}%
\pgfsetstrokecolor{currentstroke}%
\pgfsetstrokeopacity{0.000000}%
\pgfsetdash{}{0pt}%
\pgfpathmoveto{\pgfqpoint{0.800000in}{0.567600in}}%
\pgfpathlineto{\pgfqpoint{5.760000in}{0.567600in}}%
\pgfpathlineto{\pgfqpoint{5.760000in}{4.540800in}}%
\pgfpathlineto{\pgfqpoint{0.800000in}{4.540800in}}%
\pgfpathlineto{\pgfqpoint{0.800000in}{0.567600in}}%
\pgfpathclose%
\pgfusepath{fill}%
\end{pgfscope}%
\begin{pgfscope}%
\pgfsetbuttcap%
\pgfsetroundjoin%
\definecolor{currentfill}{rgb}{0.000000,0.000000,0.000000}%
\pgfsetfillcolor{currentfill}%
\pgfsetlinewidth{0.803000pt}%
\definecolor{currentstroke}{rgb}{0.000000,0.000000,0.000000}%
\pgfsetstrokecolor{currentstroke}%
\pgfsetdash{}{0pt}%
\pgfsys@defobject{currentmarker}{\pgfqpoint{0.000000in}{-0.048611in}}{\pgfqpoint{0.000000in}{0.000000in}}{%
\pgfpathmoveto{\pgfqpoint{0.000000in}{0.000000in}}%
\pgfpathlineto{\pgfqpoint{0.000000in}{-0.048611in}}%
\pgfusepath{stroke,fill}%
}%
\begin{pgfscope}%
\pgfsys@transformshift{1.025455in}{0.567600in}%
\pgfsys@useobject{currentmarker}{}%
\end{pgfscope}%
\end{pgfscope}%
\begin{pgfscope}%
\definecolor{textcolor}{rgb}{0.000000,0.000000,0.000000}%
\pgfsetstrokecolor{textcolor}%
\pgfsetfillcolor{textcolor}%
\pgftext[x=1.025455in,y=0.470378in,,top]{\color{textcolor}\sffamily\fontsize{10.000000}{12.000000}\selectfont \ensuremath{-}20}%
\end{pgfscope}%
\begin{pgfscope}%
\pgfsetbuttcap%
\pgfsetroundjoin%
\definecolor{currentfill}{rgb}{0.000000,0.000000,0.000000}%
\pgfsetfillcolor{currentfill}%
\pgfsetlinewidth{0.803000pt}%
\definecolor{currentstroke}{rgb}{0.000000,0.000000,0.000000}%
\pgfsetstrokecolor{currentstroke}%
\pgfsetdash{}{0pt}%
\pgfsys@defobject{currentmarker}{\pgfqpoint{0.000000in}{-0.048611in}}{\pgfqpoint{0.000000in}{0.000000in}}{%
\pgfpathmoveto{\pgfqpoint{0.000000in}{0.000000in}}%
\pgfpathlineto{\pgfqpoint{0.000000in}{-0.048611in}}%
\pgfusepath{stroke,fill}%
}%
\begin{pgfscope}%
\pgfsys@transformshift{1.783285in}{0.567600in}%
\pgfsys@useobject{currentmarker}{}%
\end{pgfscope}%
\end{pgfscope}%
\begin{pgfscope}%
\definecolor{textcolor}{rgb}{0.000000,0.000000,0.000000}%
\pgfsetstrokecolor{textcolor}%
\pgfsetfillcolor{textcolor}%
\pgftext[x=1.783285in,y=0.470378in,,top]{\color{textcolor}\sffamily\fontsize{10.000000}{12.000000}\selectfont 0}%
\end{pgfscope}%
\begin{pgfscope}%
\pgfsetbuttcap%
\pgfsetroundjoin%
\definecolor{currentfill}{rgb}{0.000000,0.000000,0.000000}%
\pgfsetfillcolor{currentfill}%
\pgfsetlinewidth{0.803000pt}%
\definecolor{currentstroke}{rgb}{0.000000,0.000000,0.000000}%
\pgfsetstrokecolor{currentstroke}%
\pgfsetdash{}{0pt}%
\pgfsys@defobject{currentmarker}{\pgfqpoint{0.000000in}{-0.048611in}}{\pgfqpoint{0.000000in}{0.000000in}}{%
\pgfpathmoveto{\pgfqpoint{0.000000in}{0.000000in}}%
\pgfpathlineto{\pgfqpoint{0.000000in}{-0.048611in}}%
\pgfusepath{stroke,fill}%
}%
\begin{pgfscope}%
\pgfsys@transformshift{2.541115in}{0.567600in}%
\pgfsys@useobject{currentmarker}{}%
\end{pgfscope}%
\end{pgfscope}%
\begin{pgfscope}%
\definecolor{textcolor}{rgb}{0.000000,0.000000,0.000000}%
\pgfsetstrokecolor{textcolor}%
\pgfsetfillcolor{textcolor}%
\pgftext[x=2.541115in,y=0.470378in,,top]{\color{textcolor}\sffamily\fontsize{10.000000}{12.000000}\selectfont 20}%
\end{pgfscope}%
\begin{pgfscope}%
\pgfsetbuttcap%
\pgfsetroundjoin%
\definecolor{currentfill}{rgb}{0.000000,0.000000,0.000000}%
\pgfsetfillcolor{currentfill}%
\pgfsetlinewidth{0.803000pt}%
\definecolor{currentstroke}{rgb}{0.000000,0.000000,0.000000}%
\pgfsetstrokecolor{currentstroke}%
\pgfsetdash{}{0pt}%
\pgfsys@defobject{currentmarker}{\pgfqpoint{0.000000in}{-0.048611in}}{\pgfqpoint{0.000000in}{0.000000in}}{%
\pgfpathmoveto{\pgfqpoint{0.000000in}{0.000000in}}%
\pgfpathlineto{\pgfqpoint{0.000000in}{-0.048611in}}%
\pgfusepath{stroke,fill}%
}%
\begin{pgfscope}%
\pgfsys@transformshift{3.298946in}{0.567600in}%
\pgfsys@useobject{currentmarker}{}%
\end{pgfscope}%
\end{pgfscope}%
\begin{pgfscope}%
\definecolor{textcolor}{rgb}{0.000000,0.000000,0.000000}%
\pgfsetstrokecolor{textcolor}%
\pgfsetfillcolor{textcolor}%
\pgftext[x=3.298946in,y=0.470378in,,top]{\color{textcolor}\sffamily\fontsize{10.000000}{12.000000}\selectfont 40}%
\end{pgfscope}%
\begin{pgfscope}%
\pgfsetbuttcap%
\pgfsetroundjoin%
\definecolor{currentfill}{rgb}{0.000000,0.000000,0.000000}%
\pgfsetfillcolor{currentfill}%
\pgfsetlinewidth{0.803000pt}%
\definecolor{currentstroke}{rgb}{0.000000,0.000000,0.000000}%
\pgfsetstrokecolor{currentstroke}%
\pgfsetdash{}{0pt}%
\pgfsys@defobject{currentmarker}{\pgfqpoint{0.000000in}{-0.048611in}}{\pgfqpoint{0.000000in}{0.000000in}}{%
\pgfpathmoveto{\pgfqpoint{0.000000in}{0.000000in}}%
\pgfpathlineto{\pgfqpoint{0.000000in}{-0.048611in}}%
\pgfusepath{stroke,fill}%
}%
\begin{pgfscope}%
\pgfsys@transformshift{4.056776in}{0.567600in}%
\pgfsys@useobject{currentmarker}{}%
\end{pgfscope}%
\end{pgfscope}%
\begin{pgfscope}%
\definecolor{textcolor}{rgb}{0.000000,0.000000,0.000000}%
\pgfsetstrokecolor{textcolor}%
\pgfsetfillcolor{textcolor}%
\pgftext[x=4.056776in,y=0.470378in,,top]{\color{textcolor}\sffamily\fontsize{10.000000}{12.000000}\selectfont 60}%
\end{pgfscope}%
\begin{pgfscope}%
\pgfsetbuttcap%
\pgfsetroundjoin%
\definecolor{currentfill}{rgb}{0.000000,0.000000,0.000000}%
\pgfsetfillcolor{currentfill}%
\pgfsetlinewidth{0.803000pt}%
\definecolor{currentstroke}{rgb}{0.000000,0.000000,0.000000}%
\pgfsetstrokecolor{currentstroke}%
\pgfsetdash{}{0pt}%
\pgfsys@defobject{currentmarker}{\pgfqpoint{0.000000in}{-0.048611in}}{\pgfqpoint{0.000000in}{0.000000in}}{%
\pgfpathmoveto{\pgfqpoint{0.000000in}{0.000000in}}%
\pgfpathlineto{\pgfqpoint{0.000000in}{-0.048611in}}%
\pgfusepath{stroke,fill}%
}%
\begin{pgfscope}%
\pgfsys@transformshift{4.814607in}{0.567600in}%
\pgfsys@useobject{currentmarker}{}%
\end{pgfscope}%
\end{pgfscope}%
\begin{pgfscope}%
\definecolor{textcolor}{rgb}{0.000000,0.000000,0.000000}%
\pgfsetstrokecolor{textcolor}%
\pgfsetfillcolor{textcolor}%
\pgftext[x=4.814607in,y=0.470378in,,top]{\color{textcolor}\sffamily\fontsize{10.000000}{12.000000}\selectfont 80}%
\end{pgfscope}%
\begin{pgfscope}%
\pgfsetbuttcap%
\pgfsetroundjoin%
\definecolor{currentfill}{rgb}{0.000000,0.000000,0.000000}%
\pgfsetfillcolor{currentfill}%
\pgfsetlinewidth{0.803000pt}%
\definecolor{currentstroke}{rgb}{0.000000,0.000000,0.000000}%
\pgfsetstrokecolor{currentstroke}%
\pgfsetdash{}{0pt}%
\pgfsys@defobject{currentmarker}{\pgfqpoint{0.000000in}{-0.048611in}}{\pgfqpoint{0.000000in}{0.000000in}}{%
\pgfpathmoveto{\pgfqpoint{0.000000in}{0.000000in}}%
\pgfpathlineto{\pgfqpoint{0.000000in}{-0.048611in}}%
\pgfusepath{stroke,fill}%
}%
\begin{pgfscope}%
\pgfsys@transformshift{5.572437in}{0.567600in}%
\pgfsys@useobject{currentmarker}{}%
\end{pgfscope}%
\end{pgfscope}%
\begin{pgfscope}%
\definecolor{textcolor}{rgb}{0.000000,0.000000,0.000000}%
\pgfsetstrokecolor{textcolor}%
\pgfsetfillcolor{textcolor}%
\pgftext[x=5.572437in,y=0.470378in,,top]{\color{textcolor}\sffamily\fontsize{10.000000}{12.000000}\selectfont 100}%
\end{pgfscope}%
\begin{pgfscope}%
\definecolor{textcolor}{rgb}{0.000000,0.000000,0.000000}%
\pgfsetstrokecolor{textcolor}%
\pgfsetfillcolor{textcolor}%
\pgftext[x=3.280000in,y=0.280409in,,top]{\color{textcolor}\sffamily\fontsize{10.000000}{12.000000}\selectfont n}%
\end{pgfscope}%
\begin{pgfscope}%
\pgfsetbuttcap%
\pgfsetroundjoin%
\definecolor{currentfill}{rgb}{0.000000,0.000000,0.000000}%
\pgfsetfillcolor{currentfill}%
\pgfsetlinewidth{0.803000pt}%
\definecolor{currentstroke}{rgb}{0.000000,0.000000,0.000000}%
\pgfsetstrokecolor{currentstroke}%
\pgfsetdash{}{0pt}%
\pgfsys@defobject{currentmarker}{\pgfqpoint{-0.048611in}{0.000000in}}{\pgfqpoint{-0.000000in}{0.000000in}}{%
\pgfpathmoveto{\pgfqpoint{-0.000000in}{0.000000in}}%
\pgfpathlineto{\pgfqpoint{-0.048611in}{0.000000in}}%
\pgfusepath{stroke,fill}%
}%
\begin{pgfscope}%
\pgfsys@transformshift{0.800000in}{0.879428in}%
\pgfsys@useobject{currentmarker}{}%
\end{pgfscope}%
\end{pgfscope}%
\begin{pgfscope}%
\definecolor{textcolor}{rgb}{0.000000,0.000000,0.000000}%
\pgfsetstrokecolor{textcolor}%
\pgfsetfillcolor{textcolor}%
\pgftext[x=0.373873in, y=0.826666in, left, base]{\color{textcolor}\sffamily\fontsize{10.000000}{12.000000}\selectfont \ensuremath{-}1.0}%
\end{pgfscope}%
\begin{pgfscope}%
\pgfsetbuttcap%
\pgfsetroundjoin%
\definecolor{currentfill}{rgb}{0.000000,0.000000,0.000000}%
\pgfsetfillcolor{currentfill}%
\pgfsetlinewidth{0.803000pt}%
\definecolor{currentstroke}{rgb}{0.000000,0.000000,0.000000}%
\pgfsetstrokecolor{currentstroke}%
\pgfsetdash{}{0pt}%
\pgfsys@defobject{currentmarker}{\pgfqpoint{-0.048611in}{0.000000in}}{\pgfqpoint{-0.000000in}{0.000000in}}{%
\pgfpathmoveto{\pgfqpoint{-0.000000in}{0.000000in}}%
\pgfpathlineto{\pgfqpoint{-0.048611in}{0.000000in}}%
\pgfusepath{stroke,fill}%
}%
\begin{pgfscope}%
\pgfsys@transformshift{0.800000in}{1.711014in}%
\pgfsys@useobject{currentmarker}{}%
\end{pgfscope}%
\end{pgfscope}%
\begin{pgfscope}%
\definecolor{textcolor}{rgb}{0.000000,0.000000,0.000000}%
\pgfsetstrokecolor{textcolor}%
\pgfsetfillcolor{textcolor}%
\pgftext[x=0.373873in, y=1.658252in, left, base]{\color{textcolor}\sffamily\fontsize{10.000000}{12.000000}\selectfont \ensuremath{-}0.5}%
\end{pgfscope}%
\begin{pgfscope}%
\pgfsetbuttcap%
\pgfsetroundjoin%
\definecolor{currentfill}{rgb}{0.000000,0.000000,0.000000}%
\pgfsetfillcolor{currentfill}%
\pgfsetlinewidth{0.803000pt}%
\definecolor{currentstroke}{rgb}{0.000000,0.000000,0.000000}%
\pgfsetstrokecolor{currentstroke}%
\pgfsetdash{}{0pt}%
\pgfsys@defobject{currentmarker}{\pgfqpoint{-0.048611in}{0.000000in}}{\pgfqpoint{-0.000000in}{0.000000in}}{%
\pgfpathmoveto{\pgfqpoint{-0.000000in}{0.000000in}}%
\pgfpathlineto{\pgfqpoint{-0.048611in}{0.000000in}}%
\pgfusepath{stroke,fill}%
}%
\begin{pgfscope}%
\pgfsys@transformshift{0.800000in}{2.542600in}%
\pgfsys@useobject{currentmarker}{}%
\end{pgfscope}%
\end{pgfscope}%
\begin{pgfscope}%
\definecolor{textcolor}{rgb}{0.000000,0.000000,0.000000}%
\pgfsetstrokecolor{textcolor}%
\pgfsetfillcolor{textcolor}%
\pgftext[x=0.481898in, y=2.489838in, left, base]{\color{textcolor}\sffamily\fontsize{10.000000}{12.000000}\selectfont 0.0}%
\end{pgfscope}%
\begin{pgfscope}%
\pgfsetbuttcap%
\pgfsetroundjoin%
\definecolor{currentfill}{rgb}{0.000000,0.000000,0.000000}%
\pgfsetfillcolor{currentfill}%
\pgfsetlinewidth{0.803000pt}%
\definecolor{currentstroke}{rgb}{0.000000,0.000000,0.000000}%
\pgfsetstrokecolor{currentstroke}%
\pgfsetdash{}{0pt}%
\pgfsys@defobject{currentmarker}{\pgfqpoint{-0.048611in}{0.000000in}}{\pgfqpoint{-0.000000in}{0.000000in}}{%
\pgfpathmoveto{\pgfqpoint{-0.000000in}{0.000000in}}%
\pgfpathlineto{\pgfqpoint{-0.048611in}{0.000000in}}%
\pgfusepath{stroke,fill}%
}%
\begin{pgfscope}%
\pgfsys@transformshift{0.800000in}{3.374185in}%
\pgfsys@useobject{currentmarker}{}%
\end{pgfscope}%
\end{pgfscope}%
\begin{pgfscope}%
\definecolor{textcolor}{rgb}{0.000000,0.000000,0.000000}%
\pgfsetstrokecolor{textcolor}%
\pgfsetfillcolor{textcolor}%
\pgftext[x=0.481898in, y=3.321424in, left, base]{\color{textcolor}\sffamily\fontsize{10.000000}{12.000000}\selectfont 0.5}%
\end{pgfscope}%
\begin{pgfscope}%
\pgfsetbuttcap%
\pgfsetroundjoin%
\definecolor{currentfill}{rgb}{0.000000,0.000000,0.000000}%
\pgfsetfillcolor{currentfill}%
\pgfsetlinewidth{0.803000pt}%
\definecolor{currentstroke}{rgb}{0.000000,0.000000,0.000000}%
\pgfsetstrokecolor{currentstroke}%
\pgfsetdash{}{0pt}%
\pgfsys@defobject{currentmarker}{\pgfqpoint{-0.048611in}{0.000000in}}{\pgfqpoint{-0.000000in}{0.000000in}}{%
\pgfpathmoveto{\pgfqpoint{-0.000000in}{0.000000in}}%
\pgfpathlineto{\pgfqpoint{-0.048611in}{0.000000in}}%
\pgfusepath{stroke,fill}%
}%
\begin{pgfscope}%
\pgfsys@transformshift{0.800000in}{4.205771in}%
\pgfsys@useobject{currentmarker}{}%
\end{pgfscope}%
\end{pgfscope}%
\begin{pgfscope}%
\definecolor{textcolor}{rgb}{0.000000,0.000000,0.000000}%
\pgfsetstrokecolor{textcolor}%
\pgfsetfillcolor{textcolor}%
\pgftext[x=0.481898in, y=4.153010in, left, base]{\color{textcolor}\sffamily\fontsize{10.000000}{12.000000}\selectfont 1.0}%
\end{pgfscope}%
\begin{pgfscope}%
\definecolor{textcolor}{rgb}{0.000000,0.000000,0.000000}%
\pgfsetstrokecolor{textcolor}%
\pgfsetfillcolor{textcolor}%
\pgftext[x=0.318318in,y=2.554200in,,bottom,rotate=90.000000]{\color{textcolor}\sffamily\fontsize{10.000000}{12.000000}\selectfont x[n] * m[n]}%
\end{pgfscope}%
\begin{pgfscope}%
\pgfpathrectangle{\pgfqpoint{0.800000in}{0.567600in}}{\pgfqpoint{4.960000in}{3.973200in}}%
\pgfusepath{clip}%
\pgfsetrectcap%
\pgfsetroundjoin%
\pgfsetlinewidth{1.505625pt}%
\definecolor{currentstroke}{rgb}{0.121569,0.466667,0.705882}%
\pgfsetstrokecolor{currentstroke}%
\pgfsetdash{}{0pt}%
\pgfpathmoveto{\pgfqpoint{1.025455in}{1.966545in}}%
\pgfpathlineto{\pgfqpoint{1.063346in}{1.393864in}}%
\pgfpathlineto{\pgfqpoint{1.101238in}{0.919813in}}%
\pgfpathlineto{\pgfqpoint{1.139129in}{0.937766in}}%
\pgfpathlineto{\pgfqpoint{1.177021in}{0.998903in}}%
\pgfpathlineto{\pgfqpoint{1.214912in}{0.845669in}}%
\pgfpathlineto{\pgfqpoint{1.252804in}{0.765282in}}%
\pgfpathlineto{\pgfqpoint{1.290695in}{0.766814in}}%
\pgfpathlineto{\pgfqpoint{1.328587in}{0.924387in}}%
\pgfpathlineto{\pgfqpoint{1.366478in}{1.033512in}}%
\pgfpathlineto{\pgfqpoint{1.404370in}{1.049330in}}%
\pgfpathlineto{\pgfqpoint{1.442261in}{1.024098in}}%
\pgfpathlineto{\pgfqpoint{1.480153in}{0.964441in}}%
\pgfpathlineto{\pgfqpoint{1.518044in}{0.913419in}}%
\pgfpathlineto{\pgfqpoint{1.555936in}{0.837669in}}%
\pgfpathlineto{\pgfqpoint{1.593827in}{0.873338in}}%
\pgfpathlineto{\pgfqpoint{1.631719in}{0.936728in}}%
\pgfpathlineto{\pgfqpoint{1.669610in}{0.970719in}}%
\pgfpathlineto{\pgfqpoint{1.707502in}{1.017634in}}%
\pgfpathlineto{\pgfqpoint{1.745393in}{0.926052in}}%
\pgfpathlineto{\pgfqpoint{1.783285in}{2.106006in}}%
\pgfpathlineto{\pgfqpoint{1.821176in}{3.186176in}}%
\pgfpathlineto{\pgfqpoint{1.859068in}{4.240572in}}%
\pgfpathlineto{\pgfqpoint{1.896960in}{4.077656in}}%
\pgfpathlineto{\pgfqpoint{1.934851in}{4.022082in}}%
\pgfpathlineto{\pgfqpoint{1.972743in}{4.003415in}}%
\pgfpathlineto{\pgfqpoint{2.010634in}{4.152152in}}%
\pgfpathlineto{\pgfqpoint{2.048526in}{4.172635in}}%
\pgfpathlineto{\pgfqpoint{2.086417in}{4.340638in}}%
\pgfpathlineto{\pgfqpoint{2.124309in}{4.287704in}}%
\pgfpathlineto{\pgfqpoint{2.162200in}{4.216525in}}%
\pgfpathlineto{\pgfqpoint{2.200092in}{4.203915in}}%
\pgfpathlineto{\pgfqpoint{2.237983in}{4.172666in}}%
\pgfpathlineto{\pgfqpoint{2.275875in}{4.265146in}}%
\pgfpathlineto{\pgfqpoint{2.313766in}{4.109689in}}%
\pgfpathlineto{\pgfqpoint{2.351658in}{4.033195in}}%
\pgfpathlineto{\pgfqpoint{2.389549in}{4.023744in}}%
\pgfpathlineto{\pgfqpoint{2.427441in}{4.207752in}}%
\pgfpathlineto{\pgfqpoint{2.465332in}{4.306304in}}%
\pgfpathlineto{\pgfqpoint{2.503224in}{4.300396in}}%
\pgfpathlineto{\pgfqpoint{2.541115in}{4.141782in}}%
\pgfpathlineto{\pgfqpoint{2.579007in}{4.183108in}}%
\pgfpathlineto{\pgfqpoint{2.616898in}{4.165122in}}%
\pgfpathlineto{\pgfqpoint{2.654790in}{4.142490in}}%
\pgfpathlineto{\pgfqpoint{2.692681in}{4.003682in}}%
\pgfpathlineto{\pgfqpoint{2.730573in}{2.989828in}}%
\pgfpathlineto{\pgfqpoint{2.768464in}{1.935051in}}%
\pgfpathlineto{\pgfqpoint{2.806356in}{0.940007in}}%
\pgfpathlineto{\pgfqpoint{2.844248in}{0.964676in}}%
\pgfpathlineto{\pgfqpoint{2.882139in}{1.002358in}}%
\pgfpathlineto{\pgfqpoint{2.920031in}{0.918078in}}%
\pgfpathlineto{\pgfqpoint{2.957922in}{0.748200in}}%
\pgfpathlineto{\pgfqpoint{2.995814in}{0.789511in}}%
\pgfpathlineto{\pgfqpoint{3.033705in}{0.930785in}}%
\pgfpathlineto{\pgfqpoint{3.071597in}{1.069065in}}%
\pgfpathlineto{\pgfqpoint{3.109488in}{0.983364in}}%
\pgfpathlineto{\pgfqpoint{3.147380in}{0.867172in}}%
\pgfpathlineto{\pgfqpoint{3.185271in}{0.875189in}}%
\pgfpathlineto{\pgfqpoint{3.223163in}{0.913727in}}%
\pgfpathlineto{\pgfqpoint{3.261054in}{0.943933in}}%
\pgfpathlineto{\pgfqpoint{3.298946in}{0.862153in}}%
\pgfpathlineto{\pgfqpoint{3.336837in}{0.852419in}}%
\pgfpathlineto{\pgfqpoint{3.374729in}{0.849945in}}%
\pgfpathlineto{\pgfqpoint{3.412620in}{0.948276in}}%
\pgfpathlineto{\pgfqpoint{3.450512in}{0.967329in}}%
\pgfpathlineto{\pgfqpoint{3.488403in}{1.038056in}}%
\pgfpathlineto{\pgfqpoint{3.526295in}{0.870555in}}%
\pgfpathlineto{\pgfqpoint{3.564186in}{0.903968in}}%
\pgfpathlineto{\pgfqpoint{3.602078in}{0.801169in}}%
\pgfpathlineto{\pgfqpoint{3.639969in}{0.975274in}}%
\pgfpathlineto{\pgfqpoint{3.677861in}{2.037948in}}%
\pgfpathlineto{\pgfqpoint{3.715752in}{3.158812in}}%
\pgfpathlineto{\pgfqpoint{3.753644in}{4.135627in}}%
\pgfpathlineto{\pgfqpoint{3.791536in}{4.060346in}}%
\pgfpathlineto{\pgfqpoint{3.829427in}{4.113085in}}%
\pgfpathlineto{\pgfqpoint{3.867319in}{4.223683in}}%
\pgfpathlineto{\pgfqpoint{3.905210in}{4.360200in}}%
\pgfpathlineto{\pgfqpoint{3.943102in}{4.251128in}}%
\pgfpathlineto{\pgfqpoint{3.980993in}{4.229269in}}%
\pgfpathlineto{\pgfqpoint{4.018885in}{4.165101in}}%
\pgfpathlineto{\pgfqpoint{4.056776in}{4.253558in}}%
\pgfpathlineto{\pgfqpoint{4.094668in}{4.254925in}}%
\pgfpathlineto{\pgfqpoint{4.132559in}{4.299675in}}%
\pgfpathlineto{\pgfqpoint{4.170451in}{4.324340in}}%
\pgfpathlineto{\pgfqpoint{4.208342in}{4.323941in}}%
\pgfpathlineto{\pgfqpoint{4.246234in}{4.240884in}}%
\pgfpathlineto{\pgfqpoint{4.284125in}{4.201196in}}%
\pgfpathlineto{\pgfqpoint{4.322017in}{4.122629in}}%
\pgfpathlineto{\pgfqpoint{4.359908in}{4.122294in}}%
\pgfpathlineto{\pgfqpoint{4.397800in}{4.036012in}}%
\pgfpathlineto{\pgfqpoint{4.435691in}{4.023247in}}%
\pgfpathlineto{\pgfqpoint{4.473583in}{4.163485in}}%
\pgfpathlineto{\pgfqpoint{4.511474in}{4.334136in}}%
\pgfpathlineto{\pgfqpoint{4.549366in}{4.356774in}}%
\pgfpathlineto{\pgfqpoint{4.587257in}{4.286204in}}%
\pgfpathlineto{\pgfqpoint{4.625149in}{3.036252in}}%
\pgfpathlineto{\pgfqpoint{4.663040in}{2.030695in}}%
\pgfpathlineto{\pgfqpoint{4.700932in}{0.986000in}}%
\pgfpathlineto{\pgfqpoint{4.738824in}{0.978484in}}%
\pgfpathlineto{\pgfqpoint{4.776715in}{0.971833in}}%
\pgfpathlineto{\pgfqpoint{4.814607in}{0.971258in}}%
\pgfpathlineto{\pgfqpoint{4.852498in}{1.092476in}}%
\pgfpathlineto{\pgfqpoint{4.890390in}{1.005240in}}%
\pgfpathlineto{\pgfqpoint{4.928281in}{0.927461in}}%
\pgfpathlineto{\pgfqpoint{4.966173in}{0.846572in}}%
\pgfpathlineto{\pgfqpoint{5.004064in}{0.929400in}}%
\pgfpathlineto{\pgfqpoint{5.041956in}{0.965242in}}%
\pgfpathlineto{\pgfqpoint{5.079847in}{0.950485in}}%
\pgfpathlineto{\pgfqpoint{5.117739in}{0.811219in}}%
\pgfpathlineto{\pgfqpoint{5.155630in}{0.760477in}}%
\pgfpathlineto{\pgfqpoint{5.193522in}{0.839395in}}%
\pgfpathlineto{\pgfqpoint{5.231413in}{0.986114in}}%
\pgfpathlineto{\pgfqpoint{5.269305in}{1.048218in}}%
\pgfpathlineto{\pgfqpoint{5.307196in}{0.987978in}}%
\pgfpathlineto{\pgfqpoint{5.345088in}{0.874999in}}%
\pgfpathlineto{\pgfqpoint{5.382979in}{0.782853in}}%
\pgfpathlineto{\pgfqpoint{5.420871in}{0.793853in}}%
\pgfpathlineto{\pgfqpoint{5.458762in}{0.809903in}}%
\pgfpathlineto{\pgfqpoint{5.496654in}{0.829236in}}%
\pgfpathlineto{\pgfqpoint{5.534545in}{0.916983in}}%
\pgfusepath{stroke}%
\end{pgfscope}%
\begin{pgfscope}%
\pgfsetrectcap%
\pgfsetmiterjoin%
\pgfsetlinewidth{0.803000pt}%
\definecolor{currentstroke}{rgb}{0.000000,0.000000,0.000000}%
\pgfsetstrokecolor{currentstroke}%
\pgfsetdash{}{0pt}%
\pgfpathmoveto{\pgfqpoint{0.800000in}{0.567600in}}%
\pgfpathlineto{\pgfqpoint{0.800000in}{4.540800in}}%
\pgfusepath{stroke}%
\end{pgfscope}%
\begin{pgfscope}%
\pgfsetrectcap%
\pgfsetmiterjoin%
\pgfsetlinewidth{0.803000pt}%
\definecolor{currentstroke}{rgb}{0.000000,0.000000,0.000000}%
\pgfsetstrokecolor{currentstroke}%
\pgfsetdash{}{0pt}%
\pgfpathmoveto{\pgfqpoint{5.760000in}{0.567600in}}%
\pgfpathlineto{\pgfqpoint{5.760000in}{4.540800in}}%
\pgfusepath{stroke}%
\end{pgfscope}%
\begin{pgfscope}%
\pgfsetrectcap%
\pgfsetmiterjoin%
\pgfsetlinewidth{0.803000pt}%
\definecolor{currentstroke}{rgb}{0.000000,0.000000,0.000000}%
\pgfsetstrokecolor{currentstroke}%
\pgfsetdash{}{0pt}%
\pgfpathmoveto{\pgfqpoint{0.800000in}{0.567600in}}%
\pgfpathlineto{\pgfqpoint{5.760000in}{0.567600in}}%
\pgfusepath{stroke}%
\end{pgfscope}%
\begin{pgfscope}%
\pgfsetrectcap%
\pgfsetmiterjoin%
\pgfsetlinewidth{0.803000pt}%
\definecolor{currentstroke}{rgb}{0.000000,0.000000,0.000000}%
\pgfsetstrokecolor{currentstroke}%
\pgfsetdash{}{0pt}%
\pgfpathmoveto{\pgfqpoint{0.800000in}{4.540800in}}%
\pgfpathlineto{\pgfqpoint{5.760000in}{4.540800in}}%
\pgfusepath{stroke}%
\end{pgfscope}%
\begin{pgfscope}%
\definecolor{textcolor}{rgb}{0.000000,0.000000,0.000000}%
\pgfsetstrokecolor{textcolor}%
\pgfsetfillcolor{textcolor}%
\pgftext[x=3.280000in,y=4.624133in,,base]{\color{textcolor}\sffamily\fontsize{12.000000}{14.400000}\selectfont n vs x[n] * m[n] (N=3)}%
\end{pgfscope}%
\end{pgfpicture}%
\makeatother%
\endgroup%

        \newline
        %% Creator: Matplotlib, PGF backend
%%
%% To include the figure in your LaTeX document, write
%%   \input{<filename>.pgf}
%%
%% Make sure the required packages are loaded in your preamble
%%   \usepackage{pgf}
%%
%% Also ensure that all the required font packages are loaded; for instance,
%% the lmodern package is sometimes necessary when using math font.
%%   \usepackage{lmodern}
%%
%% Figures using additional raster images can only be included by \input if
%% they are in the same directory as the main LaTeX file. For loading figures
%% from other directories you can use the `import` package
%%   \usepackage{import}
%%
%% and then include the figures with
%%   \import{<path to file>}{<filename>.pgf}
%%
%% Matplotlib used the following preamble
%%   
%%   \usepackage{fontspec}
%%   \setmainfont{DejaVuSerif.ttf}[Path=\detokenize{/home/aeren/.local/lib/python3.10/site-packages/matplotlib/mpl-data/fonts/ttf/}]
%%   \setsansfont{DejaVuSans.ttf}[Path=\detokenize{/home/aeren/.local/lib/python3.10/site-packages/matplotlib/mpl-data/fonts/ttf/}]
%%   \setmonofont{DejaVuSansMono.ttf}[Path=\detokenize{/home/aeren/.local/lib/python3.10/site-packages/matplotlib/mpl-data/fonts/ttf/}]
%%   \makeatletter\@ifpackageloaded{underscore}{}{\usepackage[strings]{underscore}}\makeatother
%%
\begingroup%
\makeatletter%
\begin{pgfpicture}%
\pgfpathrectangle{\pgfpointorigin}{\pgfqpoint{6.400000in}{5.160000in}}%
\pgfusepath{use as bounding box, clip}%
\begin{pgfscope}%
\pgfsetbuttcap%
\pgfsetmiterjoin%
\definecolor{currentfill}{rgb}{1.000000,1.000000,1.000000}%
\pgfsetfillcolor{currentfill}%
\pgfsetlinewidth{0.000000pt}%
\definecolor{currentstroke}{rgb}{1.000000,1.000000,1.000000}%
\pgfsetstrokecolor{currentstroke}%
\pgfsetdash{}{0pt}%
\pgfpathmoveto{\pgfqpoint{0.000000in}{0.000000in}}%
\pgfpathlineto{\pgfqpoint{6.400000in}{0.000000in}}%
\pgfpathlineto{\pgfqpoint{6.400000in}{5.160000in}}%
\pgfpathlineto{\pgfqpoint{0.000000in}{5.160000in}}%
\pgfpathlineto{\pgfqpoint{0.000000in}{0.000000in}}%
\pgfpathclose%
\pgfusepath{fill}%
\end{pgfscope}%
\begin{pgfscope}%
\pgfsetbuttcap%
\pgfsetmiterjoin%
\definecolor{currentfill}{rgb}{1.000000,1.000000,1.000000}%
\pgfsetfillcolor{currentfill}%
\pgfsetlinewidth{0.000000pt}%
\definecolor{currentstroke}{rgb}{0.000000,0.000000,0.000000}%
\pgfsetstrokecolor{currentstroke}%
\pgfsetstrokeopacity{0.000000}%
\pgfsetdash{}{0pt}%
\pgfpathmoveto{\pgfqpoint{0.800000in}{0.567600in}}%
\pgfpathlineto{\pgfqpoint{5.760000in}{0.567600in}}%
\pgfpathlineto{\pgfqpoint{5.760000in}{4.540800in}}%
\pgfpathlineto{\pgfqpoint{0.800000in}{4.540800in}}%
\pgfpathlineto{\pgfqpoint{0.800000in}{0.567600in}}%
\pgfpathclose%
\pgfusepath{fill}%
\end{pgfscope}%
\begin{pgfscope}%
\pgfsetbuttcap%
\pgfsetroundjoin%
\definecolor{currentfill}{rgb}{0.000000,0.000000,0.000000}%
\pgfsetfillcolor{currentfill}%
\pgfsetlinewidth{0.803000pt}%
\definecolor{currentstroke}{rgb}{0.000000,0.000000,0.000000}%
\pgfsetstrokecolor{currentstroke}%
\pgfsetdash{}{0pt}%
\pgfsys@defobject{currentmarker}{\pgfqpoint{0.000000in}{-0.048611in}}{\pgfqpoint{0.000000in}{0.000000in}}{%
\pgfpathmoveto{\pgfqpoint{0.000000in}{0.000000in}}%
\pgfpathlineto{\pgfqpoint{0.000000in}{-0.048611in}}%
\pgfusepath{stroke,fill}%
}%
\begin{pgfscope}%
\pgfsys@transformshift{1.025455in}{0.567600in}%
\pgfsys@useobject{currentmarker}{}%
\end{pgfscope}%
\end{pgfscope}%
\begin{pgfscope}%
\definecolor{textcolor}{rgb}{0.000000,0.000000,0.000000}%
\pgfsetstrokecolor{textcolor}%
\pgfsetfillcolor{textcolor}%
\pgftext[x=1.025455in,y=0.470378in,,top]{\color{textcolor}\sffamily\fontsize{10.000000}{12.000000}\selectfont \ensuremath{-}20}%
\end{pgfscope}%
\begin{pgfscope}%
\pgfsetbuttcap%
\pgfsetroundjoin%
\definecolor{currentfill}{rgb}{0.000000,0.000000,0.000000}%
\pgfsetfillcolor{currentfill}%
\pgfsetlinewidth{0.803000pt}%
\definecolor{currentstroke}{rgb}{0.000000,0.000000,0.000000}%
\pgfsetstrokecolor{currentstroke}%
\pgfsetdash{}{0pt}%
\pgfsys@defobject{currentmarker}{\pgfqpoint{0.000000in}{-0.048611in}}{\pgfqpoint{0.000000in}{0.000000in}}{%
\pgfpathmoveto{\pgfqpoint{0.000000in}{0.000000in}}%
\pgfpathlineto{\pgfqpoint{0.000000in}{-0.048611in}}%
\pgfusepath{stroke,fill}%
}%
\begin{pgfscope}%
\pgfsys@transformshift{1.783285in}{0.567600in}%
\pgfsys@useobject{currentmarker}{}%
\end{pgfscope}%
\end{pgfscope}%
\begin{pgfscope}%
\definecolor{textcolor}{rgb}{0.000000,0.000000,0.000000}%
\pgfsetstrokecolor{textcolor}%
\pgfsetfillcolor{textcolor}%
\pgftext[x=1.783285in,y=0.470378in,,top]{\color{textcolor}\sffamily\fontsize{10.000000}{12.000000}\selectfont 0}%
\end{pgfscope}%
\begin{pgfscope}%
\pgfsetbuttcap%
\pgfsetroundjoin%
\definecolor{currentfill}{rgb}{0.000000,0.000000,0.000000}%
\pgfsetfillcolor{currentfill}%
\pgfsetlinewidth{0.803000pt}%
\definecolor{currentstroke}{rgb}{0.000000,0.000000,0.000000}%
\pgfsetstrokecolor{currentstroke}%
\pgfsetdash{}{0pt}%
\pgfsys@defobject{currentmarker}{\pgfqpoint{0.000000in}{-0.048611in}}{\pgfqpoint{0.000000in}{0.000000in}}{%
\pgfpathmoveto{\pgfqpoint{0.000000in}{0.000000in}}%
\pgfpathlineto{\pgfqpoint{0.000000in}{-0.048611in}}%
\pgfusepath{stroke,fill}%
}%
\begin{pgfscope}%
\pgfsys@transformshift{2.541115in}{0.567600in}%
\pgfsys@useobject{currentmarker}{}%
\end{pgfscope}%
\end{pgfscope}%
\begin{pgfscope}%
\definecolor{textcolor}{rgb}{0.000000,0.000000,0.000000}%
\pgfsetstrokecolor{textcolor}%
\pgfsetfillcolor{textcolor}%
\pgftext[x=2.541115in,y=0.470378in,,top]{\color{textcolor}\sffamily\fontsize{10.000000}{12.000000}\selectfont 20}%
\end{pgfscope}%
\begin{pgfscope}%
\pgfsetbuttcap%
\pgfsetroundjoin%
\definecolor{currentfill}{rgb}{0.000000,0.000000,0.000000}%
\pgfsetfillcolor{currentfill}%
\pgfsetlinewidth{0.803000pt}%
\definecolor{currentstroke}{rgb}{0.000000,0.000000,0.000000}%
\pgfsetstrokecolor{currentstroke}%
\pgfsetdash{}{0pt}%
\pgfsys@defobject{currentmarker}{\pgfqpoint{0.000000in}{-0.048611in}}{\pgfqpoint{0.000000in}{0.000000in}}{%
\pgfpathmoveto{\pgfqpoint{0.000000in}{0.000000in}}%
\pgfpathlineto{\pgfqpoint{0.000000in}{-0.048611in}}%
\pgfusepath{stroke,fill}%
}%
\begin{pgfscope}%
\pgfsys@transformshift{3.298946in}{0.567600in}%
\pgfsys@useobject{currentmarker}{}%
\end{pgfscope}%
\end{pgfscope}%
\begin{pgfscope}%
\definecolor{textcolor}{rgb}{0.000000,0.000000,0.000000}%
\pgfsetstrokecolor{textcolor}%
\pgfsetfillcolor{textcolor}%
\pgftext[x=3.298946in,y=0.470378in,,top]{\color{textcolor}\sffamily\fontsize{10.000000}{12.000000}\selectfont 40}%
\end{pgfscope}%
\begin{pgfscope}%
\pgfsetbuttcap%
\pgfsetroundjoin%
\definecolor{currentfill}{rgb}{0.000000,0.000000,0.000000}%
\pgfsetfillcolor{currentfill}%
\pgfsetlinewidth{0.803000pt}%
\definecolor{currentstroke}{rgb}{0.000000,0.000000,0.000000}%
\pgfsetstrokecolor{currentstroke}%
\pgfsetdash{}{0pt}%
\pgfsys@defobject{currentmarker}{\pgfqpoint{0.000000in}{-0.048611in}}{\pgfqpoint{0.000000in}{0.000000in}}{%
\pgfpathmoveto{\pgfqpoint{0.000000in}{0.000000in}}%
\pgfpathlineto{\pgfqpoint{0.000000in}{-0.048611in}}%
\pgfusepath{stroke,fill}%
}%
\begin{pgfscope}%
\pgfsys@transformshift{4.056776in}{0.567600in}%
\pgfsys@useobject{currentmarker}{}%
\end{pgfscope}%
\end{pgfscope}%
\begin{pgfscope}%
\definecolor{textcolor}{rgb}{0.000000,0.000000,0.000000}%
\pgfsetstrokecolor{textcolor}%
\pgfsetfillcolor{textcolor}%
\pgftext[x=4.056776in,y=0.470378in,,top]{\color{textcolor}\sffamily\fontsize{10.000000}{12.000000}\selectfont 60}%
\end{pgfscope}%
\begin{pgfscope}%
\pgfsetbuttcap%
\pgfsetroundjoin%
\definecolor{currentfill}{rgb}{0.000000,0.000000,0.000000}%
\pgfsetfillcolor{currentfill}%
\pgfsetlinewidth{0.803000pt}%
\definecolor{currentstroke}{rgb}{0.000000,0.000000,0.000000}%
\pgfsetstrokecolor{currentstroke}%
\pgfsetdash{}{0pt}%
\pgfsys@defobject{currentmarker}{\pgfqpoint{0.000000in}{-0.048611in}}{\pgfqpoint{0.000000in}{0.000000in}}{%
\pgfpathmoveto{\pgfqpoint{0.000000in}{0.000000in}}%
\pgfpathlineto{\pgfqpoint{0.000000in}{-0.048611in}}%
\pgfusepath{stroke,fill}%
}%
\begin{pgfscope}%
\pgfsys@transformshift{4.814607in}{0.567600in}%
\pgfsys@useobject{currentmarker}{}%
\end{pgfscope}%
\end{pgfscope}%
\begin{pgfscope}%
\definecolor{textcolor}{rgb}{0.000000,0.000000,0.000000}%
\pgfsetstrokecolor{textcolor}%
\pgfsetfillcolor{textcolor}%
\pgftext[x=4.814607in,y=0.470378in,,top]{\color{textcolor}\sffamily\fontsize{10.000000}{12.000000}\selectfont 80}%
\end{pgfscope}%
\begin{pgfscope}%
\pgfsetbuttcap%
\pgfsetroundjoin%
\definecolor{currentfill}{rgb}{0.000000,0.000000,0.000000}%
\pgfsetfillcolor{currentfill}%
\pgfsetlinewidth{0.803000pt}%
\definecolor{currentstroke}{rgb}{0.000000,0.000000,0.000000}%
\pgfsetstrokecolor{currentstroke}%
\pgfsetdash{}{0pt}%
\pgfsys@defobject{currentmarker}{\pgfqpoint{0.000000in}{-0.048611in}}{\pgfqpoint{0.000000in}{0.000000in}}{%
\pgfpathmoveto{\pgfqpoint{0.000000in}{0.000000in}}%
\pgfpathlineto{\pgfqpoint{0.000000in}{-0.048611in}}%
\pgfusepath{stroke,fill}%
}%
\begin{pgfscope}%
\pgfsys@transformshift{5.572437in}{0.567600in}%
\pgfsys@useobject{currentmarker}{}%
\end{pgfscope}%
\end{pgfscope}%
\begin{pgfscope}%
\definecolor{textcolor}{rgb}{0.000000,0.000000,0.000000}%
\pgfsetstrokecolor{textcolor}%
\pgfsetfillcolor{textcolor}%
\pgftext[x=5.572437in,y=0.470378in,,top]{\color{textcolor}\sffamily\fontsize{10.000000}{12.000000}\selectfont 100}%
\end{pgfscope}%
\begin{pgfscope}%
\definecolor{textcolor}{rgb}{0.000000,0.000000,0.000000}%
\pgfsetstrokecolor{textcolor}%
\pgfsetfillcolor{textcolor}%
\pgftext[x=3.280000in,y=0.280409in,,top]{\color{textcolor}\sffamily\fontsize{10.000000}{12.000000}\selectfont n}%
\end{pgfscope}%
\begin{pgfscope}%
\pgfsetbuttcap%
\pgfsetroundjoin%
\definecolor{currentfill}{rgb}{0.000000,0.000000,0.000000}%
\pgfsetfillcolor{currentfill}%
\pgfsetlinewidth{0.803000pt}%
\definecolor{currentstroke}{rgb}{0.000000,0.000000,0.000000}%
\pgfsetstrokecolor{currentstroke}%
\pgfsetdash{}{0pt}%
\pgfsys@defobject{currentmarker}{\pgfqpoint{-0.048611in}{0.000000in}}{\pgfqpoint{-0.000000in}{0.000000in}}{%
\pgfpathmoveto{\pgfqpoint{-0.000000in}{0.000000in}}%
\pgfpathlineto{\pgfqpoint{-0.048611in}{0.000000in}}%
\pgfusepath{stroke,fill}%
}%
\begin{pgfscope}%
\pgfsys@transformshift{0.800000in}{0.827287in}%
\pgfsys@useobject{currentmarker}{}%
\end{pgfscope}%
\end{pgfscope}%
\begin{pgfscope}%
\definecolor{textcolor}{rgb}{0.000000,0.000000,0.000000}%
\pgfsetstrokecolor{textcolor}%
\pgfsetfillcolor{textcolor}%
\pgftext[x=0.373873in, y=0.774526in, left, base]{\color{textcolor}\sffamily\fontsize{10.000000}{12.000000}\selectfont \ensuremath{-}1.0}%
\end{pgfscope}%
\begin{pgfscope}%
\pgfsetbuttcap%
\pgfsetroundjoin%
\definecolor{currentfill}{rgb}{0.000000,0.000000,0.000000}%
\pgfsetfillcolor{currentfill}%
\pgfsetlinewidth{0.803000pt}%
\definecolor{currentstroke}{rgb}{0.000000,0.000000,0.000000}%
\pgfsetstrokecolor{currentstroke}%
\pgfsetdash{}{0pt}%
\pgfsys@defobject{currentmarker}{\pgfqpoint{-0.048611in}{0.000000in}}{\pgfqpoint{-0.000000in}{0.000000in}}{%
\pgfpathmoveto{\pgfqpoint{-0.000000in}{0.000000in}}%
\pgfpathlineto{\pgfqpoint{-0.048611in}{0.000000in}}%
\pgfusepath{stroke,fill}%
}%
\begin{pgfscope}%
\pgfsys@transformshift{0.800000in}{1.683164in}%
\pgfsys@useobject{currentmarker}{}%
\end{pgfscope}%
\end{pgfscope}%
\begin{pgfscope}%
\definecolor{textcolor}{rgb}{0.000000,0.000000,0.000000}%
\pgfsetstrokecolor{textcolor}%
\pgfsetfillcolor{textcolor}%
\pgftext[x=0.373873in, y=1.630402in, left, base]{\color{textcolor}\sffamily\fontsize{10.000000}{12.000000}\selectfont \ensuremath{-}0.5}%
\end{pgfscope}%
\begin{pgfscope}%
\pgfsetbuttcap%
\pgfsetroundjoin%
\definecolor{currentfill}{rgb}{0.000000,0.000000,0.000000}%
\pgfsetfillcolor{currentfill}%
\pgfsetlinewidth{0.803000pt}%
\definecolor{currentstroke}{rgb}{0.000000,0.000000,0.000000}%
\pgfsetstrokecolor{currentstroke}%
\pgfsetdash{}{0pt}%
\pgfsys@defobject{currentmarker}{\pgfqpoint{-0.048611in}{0.000000in}}{\pgfqpoint{-0.000000in}{0.000000in}}{%
\pgfpathmoveto{\pgfqpoint{-0.000000in}{0.000000in}}%
\pgfpathlineto{\pgfqpoint{-0.048611in}{0.000000in}}%
\pgfusepath{stroke,fill}%
}%
\begin{pgfscope}%
\pgfsys@transformshift{0.800000in}{2.539041in}%
\pgfsys@useobject{currentmarker}{}%
\end{pgfscope}%
\end{pgfscope}%
\begin{pgfscope}%
\definecolor{textcolor}{rgb}{0.000000,0.000000,0.000000}%
\pgfsetstrokecolor{textcolor}%
\pgfsetfillcolor{textcolor}%
\pgftext[x=0.481898in, y=2.486279in, left, base]{\color{textcolor}\sffamily\fontsize{10.000000}{12.000000}\selectfont 0.0}%
\end{pgfscope}%
\begin{pgfscope}%
\pgfsetbuttcap%
\pgfsetroundjoin%
\definecolor{currentfill}{rgb}{0.000000,0.000000,0.000000}%
\pgfsetfillcolor{currentfill}%
\pgfsetlinewidth{0.803000pt}%
\definecolor{currentstroke}{rgb}{0.000000,0.000000,0.000000}%
\pgfsetstrokecolor{currentstroke}%
\pgfsetdash{}{0pt}%
\pgfsys@defobject{currentmarker}{\pgfqpoint{-0.048611in}{0.000000in}}{\pgfqpoint{-0.000000in}{0.000000in}}{%
\pgfpathmoveto{\pgfqpoint{-0.000000in}{0.000000in}}%
\pgfpathlineto{\pgfqpoint{-0.048611in}{0.000000in}}%
\pgfusepath{stroke,fill}%
}%
\begin{pgfscope}%
\pgfsys@transformshift{0.800000in}{3.394918in}%
\pgfsys@useobject{currentmarker}{}%
\end{pgfscope}%
\end{pgfscope}%
\begin{pgfscope}%
\definecolor{textcolor}{rgb}{0.000000,0.000000,0.000000}%
\pgfsetstrokecolor{textcolor}%
\pgfsetfillcolor{textcolor}%
\pgftext[x=0.481898in, y=3.342156in, left, base]{\color{textcolor}\sffamily\fontsize{10.000000}{12.000000}\selectfont 0.5}%
\end{pgfscope}%
\begin{pgfscope}%
\pgfsetbuttcap%
\pgfsetroundjoin%
\definecolor{currentfill}{rgb}{0.000000,0.000000,0.000000}%
\pgfsetfillcolor{currentfill}%
\pgfsetlinewidth{0.803000pt}%
\definecolor{currentstroke}{rgb}{0.000000,0.000000,0.000000}%
\pgfsetstrokecolor{currentstroke}%
\pgfsetdash{}{0pt}%
\pgfsys@defobject{currentmarker}{\pgfqpoint{-0.048611in}{0.000000in}}{\pgfqpoint{-0.000000in}{0.000000in}}{%
\pgfpathmoveto{\pgfqpoint{-0.000000in}{0.000000in}}%
\pgfpathlineto{\pgfqpoint{-0.048611in}{0.000000in}}%
\pgfusepath{stroke,fill}%
}%
\begin{pgfscope}%
\pgfsys@transformshift{0.800000in}{4.250794in}%
\pgfsys@useobject{currentmarker}{}%
\end{pgfscope}%
\end{pgfscope}%
\begin{pgfscope}%
\definecolor{textcolor}{rgb}{0.000000,0.000000,0.000000}%
\pgfsetstrokecolor{textcolor}%
\pgfsetfillcolor{textcolor}%
\pgftext[x=0.481898in, y=4.198033in, left, base]{\color{textcolor}\sffamily\fontsize{10.000000}{12.000000}\selectfont 1.0}%
\end{pgfscope}%
\begin{pgfscope}%
\definecolor{textcolor}{rgb}{0.000000,0.000000,0.000000}%
\pgfsetstrokecolor{textcolor}%
\pgfsetfillcolor{textcolor}%
\pgftext[x=0.318318in,y=2.554200in,,bottom,rotate=90.000000]{\color{textcolor}\sffamily\fontsize{10.000000}{12.000000}\selectfont x[n] * m[n]}%
\end{pgfscope}%
\begin{pgfscope}%
\pgfpathrectangle{\pgfqpoint{0.800000in}{0.567600in}}{\pgfqpoint{4.960000in}{3.973200in}}%
\pgfusepath{clip}%
\pgfsetrectcap%
\pgfsetroundjoin%
\pgfsetlinewidth{1.505625pt}%
\definecolor{currentstroke}{rgb}{0.121569,0.466667,0.705882}%
\pgfsetstrokecolor{currentstroke}%
\pgfsetdash{}{0pt}%
\pgfpathmoveto{\pgfqpoint{1.025455in}{2.183312in}}%
\pgfpathlineto{\pgfqpoint{1.063346in}{1.829667in}}%
\pgfpathlineto{\pgfqpoint{1.101238in}{1.536928in}}%
\pgfpathlineto{\pgfqpoint{1.139129in}{1.192285in}}%
\pgfpathlineto{\pgfqpoint{1.177021in}{0.876394in}}%
\pgfpathlineto{\pgfqpoint{1.214912in}{0.844757in}}%
\pgfpathlineto{\pgfqpoint{1.252804in}{0.804119in}}%
\pgfpathlineto{\pgfqpoint{1.290695in}{0.781913in}}%
\pgfpathlineto{\pgfqpoint{1.328587in}{0.836495in}}%
\pgfpathlineto{\pgfqpoint{1.366478in}{0.825492in}}%
\pgfpathlineto{\pgfqpoint{1.404370in}{0.907679in}}%
\pgfpathlineto{\pgfqpoint{1.442261in}{0.996321in}}%
\pgfpathlineto{\pgfqpoint{1.480153in}{0.947532in}}%
\pgfpathlineto{\pgfqpoint{1.518044in}{0.900907in}}%
\pgfpathlineto{\pgfqpoint{1.555936in}{0.875382in}}%
\pgfpathlineto{\pgfqpoint{1.593827in}{0.838851in}}%
\pgfpathlineto{\pgfqpoint{1.631719in}{0.846953in}}%
\pgfpathlineto{\pgfqpoint{1.669610in}{0.879259in}}%
\pgfpathlineto{\pgfqpoint{1.707502in}{0.903207in}}%
\pgfpathlineto{\pgfqpoint{1.745393in}{0.901532in}}%
\pgfpathlineto{\pgfqpoint{1.783285in}{1.640464in}}%
\pgfpathlineto{\pgfqpoint{1.821176in}{2.292300in}}%
\pgfpathlineto{\pgfqpoint{1.859068in}{2.920752in}}%
\pgfpathlineto{\pgfqpoint{1.896960in}{3.530107in}}%
\pgfpathlineto{\pgfqpoint{1.934851in}{4.204180in}}%
\pgfpathlineto{\pgfqpoint{1.972743in}{4.092452in}}%
\pgfpathlineto{\pgfqpoint{2.010634in}{4.126623in}}%
\pgfpathlineto{\pgfqpoint{2.048526in}{4.162227in}}%
\pgfpathlineto{\pgfqpoint{2.086417in}{4.254850in}}%
\pgfpathlineto{\pgfqpoint{2.124309in}{4.290652in}}%
\pgfpathlineto{\pgfqpoint{2.162200in}{4.293828in}}%
\pgfpathlineto{\pgfqpoint{2.200092in}{4.286815in}}%
\pgfpathlineto{\pgfqpoint{2.237983in}{4.290671in}}%
\pgfpathlineto{\pgfqpoint{2.275875in}{4.247210in}}%
\pgfpathlineto{\pgfqpoint{2.313766in}{4.176886in}}%
\pgfpathlineto{\pgfqpoint{2.351658in}{4.177459in}}%
\pgfpathlineto{\pgfqpoint{2.389549in}{4.135949in}}%
\pgfpathlineto{\pgfqpoint{2.427441in}{4.198552in}}%
\pgfpathlineto{\pgfqpoint{2.465332in}{4.202875in}}%
\pgfpathlineto{\pgfqpoint{2.503224in}{4.253716in}}%
\pgfpathlineto{\pgfqpoint{2.541115in}{4.265607in}}%
\pgfpathlineto{\pgfqpoint{2.579007in}{4.301287in}}%
\pgfpathlineto{\pgfqpoint{2.616898in}{4.227391in}}%
\pgfpathlineto{\pgfqpoint{2.654790in}{4.164448in}}%
\pgfpathlineto{\pgfqpoint{2.692681in}{4.118058in}}%
\pgfpathlineto{\pgfqpoint{2.730573in}{3.516029in}}%
\pgfpathlineto{\pgfqpoint{2.768464in}{2.776215in}}%
\pgfpathlineto{\pgfqpoint{2.806356in}{2.126465in}}%
\pgfpathlineto{\pgfqpoint{2.844248in}{1.553646in}}%
\pgfpathlineto{\pgfqpoint{2.882139in}{0.922819in}}%
\pgfpathlineto{\pgfqpoint{2.920031in}{0.847105in}}%
\pgfpathlineto{\pgfqpoint{2.957922in}{0.820734in}}%
\pgfpathlineto{\pgfqpoint{2.995814in}{0.829883in}}%
\pgfpathlineto{\pgfqpoint{3.033705in}{0.826176in}}%
\pgfpathlineto{\pgfqpoint{3.071597in}{0.861928in}}%
\pgfpathlineto{\pgfqpoint{3.109488in}{0.870199in}}%
\pgfpathlineto{\pgfqpoint{3.147380in}{0.899645in}}%
\pgfpathlineto{\pgfqpoint{3.185271in}{0.914836in}}%
\pgfpathlineto{\pgfqpoint{3.223163in}{0.859666in}}%
\pgfpathlineto{\pgfqpoint{3.261054in}{0.822372in}}%
\pgfpathlineto{\pgfqpoint{3.298946in}{0.839985in}}%
\pgfpathlineto{\pgfqpoint{3.336837in}{0.850555in}}%
\pgfpathlineto{\pgfqpoint{3.374729in}{0.806784in}}%
\pgfpathlineto{\pgfqpoint{3.412620in}{0.861319in}}%
\pgfpathlineto{\pgfqpoint{3.450512in}{0.865003in}}%
\pgfpathlineto{\pgfqpoint{3.488403in}{0.915409in}}%
\pgfpathlineto{\pgfqpoint{3.526295in}{0.872519in}}%
\pgfpathlineto{\pgfqpoint{3.564186in}{0.898363in}}%
\pgfpathlineto{\pgfqpoint{3.602078in}{0.824566in}}%
\pgfpathlineto{\pgfqpoint{3.639969in}{0.877425in}}%
\pgfpathlineto{\pgfqpoint{3.677861in}{1.515823in}}%
\pgfpathlineto{\pgfqpoint{3.715752in}{2.237625in}}%
\pgfpathlineto{\pgfqpoint{3.753644in}{2.873059in}}%
\pgfpathlineto{\pgfqpoint{3.791536in}{3.528449in}}%
\pgfpathlineto{\pgfqpoint{3.829427in}{4.175305in}}%
\pgfpathlineto{\pgfqpoint{3.867319in}{4.222808in}}%
\pgfpathlineto{\pgfqpoint{3.905210in}{4.270338in}}%
\pgfpathlineto{\pgfqpoint{3.943102in}{4.246630in}}%
\pgfpathlineto{\pgfqpoint{3.980993in}{4.327123in}}%
\pgfpathlineto{\pgfqpoint{4.018885in}{4.302459in}}%
\pgfpathlineto{\pgfqpoint{4.056776in}{4.265078in}}%
\pgfpathlineto{\pgfqpoint{4.094668in}{4.262112in}}%
\pgfpathlineto{\pgfqpoint{4.132559in}{4.332439in}}%
\pgfpathlineto{\pgfqpoint{4.170451in}{4.323787in}}%
\pgfpathlineto{\pgfqpoint{4.208342in}{4.360200in}}%
\pgfpathlineto{\pgfqpoint{4.246234in}{4.324612in}}%
\pgfpathlineto{\pgfqpoint{4.284125in}{4.290608in}}%
\pgfpathlineto{\pgfqpoint{4.322017in}{4.250869in}}%
\pgfpathlineto{\pgfqpoint{4.359908in}{4.199843in}}%
\pgfpathlineto{\pgfqpoint{4.397800in}{4.112804in}}%
\pgfpathlineto{\pgfqpoint{4.435691in}{4.116473in}}%
\pgfpathlineto{\pgfqpoint{4.473583in}{4.176556in}}%
\pgfpathlineto{\pgfqpoint{4.511474in}{4.243415in}}%
\pgfpathlineto{\pgfqpoint{4.549366in}{4.261271in}}%
\pgfpathlineto{\pgfqpoint{4.587257in}{4.331057in}}%
\pgfpathlineto{\pgfqpoint{4.625149in}{3.633920in}}%
\pgfpathlineto{\pgfqpoint{4.663040in}{2.944217in}}%
\pgfpathlineto{\pgfqpoint{4.700932in}{2.263495in}}%
\pgfpathlineto{\pgfqpoint{4.738824in}{1.547737in}}%
\pgfpathlineto{\pgfqpoint{4.776715in}{0.897506in}}%
\pgfpathlineto{\pgfqpoint{4.814607in}{0.988307in}}%
\pgfpathlineto{\pgfqpoint{4.852498in}{0.968362in}}%
\pgfpathlineto{\pgfqpoint{4.890390in}{0.909388in}}%
\pgfpathlineto{\pgfqpoint{4.928281in}{0.956799in}}%
\pgfpathlineto{\pgfqpoint{4.966173in}{0.891010in}}%
\pgfpathlineto{\pgfqpoint{5.004064in}{0.883539in}}%
\pgfpathlineto{\pgfqpoint{5.041956in}{0.878229in}}%
\pgfpathlineto{\pgfqpoint{5.079847in}{0.857197in}}%
\pgfpathlineto{\pgfqpoint{5.117739in}{0.811756in}}%
\pgfpathlineto{\pgfqpoint{5.155630in}{0.825063in}}%
\pgfpathlineto{\pgfqpoint{5.193522in}{0.801617in}}%
\pgfpathlineto{\pgfqpoint{5.231413in}{0.824646in}}%
\pgfpathlineto{\pgfqpoint{5.269305in}{0.885416in}}%
\pgfpathlineto{\pgfqpoint{5.307196in}{0.910771in}}%
\pgfpathlineto{\pgfqpoint{5.345088in}{0.895366in}}%
\pgfpathlineto{\pgfqpoint{5.382979in}{0.850499in}}%
\pgfpathlineto{\pgfqpoint{5.420871in}{0.792044in}}%
\pgfpathlineto{\pgfqpoint{5.458762in}{0.748200in}}%
\pgfpathlineto{\pgfqpoint{5.496654in}{0.752472in}}%
\pgfpathlineto{\pgfqpoint{5.534545in}{0.817971in}}%
\pgfusepath{stroke}%
\end{pgfscope}%
\begin{pgfscope}%
\pgfsetrectcap%
\pgfsetmiterjoin%
\pgfsetlinewidth{0.803000pt}%
\definecolor{currentstroke}{rgb}{0.000000,0.000000,0.000000}%
\pgfsetstrokecolor{currentstroke}%
\pgfsetdash{}{0pt}%
\pgfpathmoveto{\pgfqpoint{0.800000in}{0.567600in}}%
\pgfpathlineto{\pgfqpoint{0.800000in}{4.540800in}}%
\pgfusepath{stroke}%
\end{pgfscope}%
\begin{pgfscope}%
\pgfsetrectcap%
\pgfsetmiterjoin%
\pgfsetlinewidth{0.803000pt}%
\definecolor{currentstroke}{rgb}{0.000000,0.000000,0.000000}%
\pgfsetstrokecolor{currentstroke}%
\pgfsetdash{}{0pt}%
\pgfpathmoveto{\pgfqpoint{5.760000in}{0.567600in}}%
\pgfpathlineto{\pgfqpoint{5.760000in}{4.540800in}}%
\pgfusepath{stroke}%
\end{pgfscope}%
\begin{pgfscope}%
\pgfsetrectcap%
\pgfsetmiterjoin%
\pgfsetlinewidth{0.803000pt}%
\definecolor{currentstroke}{rgb}{0.000000,0.000000,0.000000}%
\pgfsetstrokecolor{currentstroke}%
\pgfsetdash{}{0pt}%
\pgfpathmoveto{\pgfqpoint{0.800000in}{0.567600in}}%
\pgfpathlineto{\pgfqpoint{5.760000in}{0.567600in}}%
\pgfusepath{stroke}%
\end{pgfscope}%
\begin{pgfscope}%
\pgfsetrectcap%
\pgfsetmiterjoin%
\pgfsetlinewidth{0.803000pt}%
\definecolor{currentstroke}{rgb}{0.000000,0.000000,0.000000}%
\pgfsetstrokecolor{currentstroke}%
\pgfsetdash{}{0pt}%
\pgfpathmoveto{\pgfqpoint{0.800000in}{4.540800in}}%
\pgfpathlineto{\pgfqpoint{5.760000in}{4.540800in}}%
\pgfusepath{stroke}%
\end{pgfscope}%
\begin{pgfscope}%
\definecolor{textcolor}{rgb}{0.000000,0.000000,0.000000}%
\pgfsetstrokecolor{textcolor}%
\pgfsetfillcolor{textcolor}%
\pgftext[x=3.280000in,y=4.624133in,,base]{\color{textcolor}\sffamily\fontsize{12.000000}{14.400000}\selectfont n vs x[n] * m[n] (N=5)}%
\end{pgfscope}%
\end{pgfpicture}%
\makeatother%
\endgroup%

        \newline
        %% Creator: Matplotlib, PGF backend
%%
%% To include the figure in your LaTeX document, write
%%   \input{<filename>.pgf}
%%
%% Make sure the required packages are loaded in your preamble
%%   \usepackage{pgf}
%%
%% Also ensure that all the required font packages are loaded; for instance,
%% the lmodern package is sometimes necessary when using math font.
%%   \usepackage{lmodern}
%%
%% Figures using additional raster images can only be included by \input if
%% they are in the same directory as the main LaTeX file. For loading figures
%% from other directories you can use the `import` package
%%   \usepackage{import}
%%
%% and then include the figures with
%%   \import{<path to file>}{<filename>.pgf}
%%
%% Matplotlib used the following preamble
%%   
%%   \usepackage{fontspec}
%%   \setmainfont{DejaVuSerif.ttf}[Path=\detokenize{/home/aeren/.local/lib/python3.10/site-packages/matplotlib/mpl-data/fonts/ttf/}]
%%   \setsansfont{DejaVuSans.ttf}[Path=\detokenize{/home/aeren/.local/lib/python3.10/site-packages/matplotlib/mpl-data/fonts/ttf/}]
%%   \setmonofont{DejaVuSansMono.ttf}[Path=\detokenize{/home/aeren/.local/lib/python3.10/site-packages/matplotlib/mpl-data/fonts/ttf/}]
%%   \makeatletter\@ifpackageloaded{underscore}{}{\usepackage[strings]{underscore}}\makeatother
%%
\begingroup%
\makeatletter%
\begin{pgfpicture}%
\pgfpathrectangle{\pgfpointorigin}{\pgfqpoint{6.400000in}{5.160000in}}%
\pgfusepath{use as bounding box, clip}%
\begin{pgfscope}%
\pgfsetbuttcap%
\pgfsetmiterjoin%
\definecolor{currentfill}{rgb}{1.000000,1.000000,1.000000}%
\pgfsetfillcolor{currentfill}%
\pgfsetlinewidth{0.000000pt}%
\definecolor{currentstroke}{rgb}{1.000000,1.000000,1.000000}%
\pgfsetstrokecolor{currentstroke}%
\pgfsetdash{}{0pt}%
\pgfpathmoveto{\pgfqpoint{0.000000in}{0.000000in}}%
\pgfpathlineto{\pgfqpoint{6.400000in}{0.000000in}}%
\pgfpathlineto{\pgfqpoint{6.400000in}{5.160000in}}%
\pgfpathlineto{\pgfqpoint{0.000000in}{5.160000in}}%
\pgfpathlineto{\pgfqpoint{0.000000in}{0.000000in}}%
\pgfpathclose%
\pgfusepath{fill}%
\end{pgfscope}%
\begin{pgfscope}%
\pgfsetbuttcap%
\pgfsetmiterjoin%
\definecolor{currentfill}{rgb}{1.000000,1.000000,1.000000}%
\pgfsetfillcolor{currentfill}%
\pgfsetlinewidth{0.000000pt}%
\definecolor{currentstroke}{rgb}{0.000000,0.000000,0.000000}%
\pgfsetstrokecolor{currentstroke}%
\pgfsetstrokeopacity{0.000000}%
\pgfsetdash{}{0pt}%
\pgfpathmoveto{\pgfqpoint{0.800000in}{0.567600in}}%
\pgfpathlineto{\pgfqpoint{5.760000in}{0.567600in}}%
\pgfpathlineto{\pgfqpoint{5.760000in}{4.540800in}}%
\pgfpathlineto{\pgfqpoint{0.800000in}{4.540800in}}%
\pgfpathlineto{\pgfqpoint{0.800000in}{0.567600in}}%
\pgfpathclose%
\pgfusepath{fill}%
\end{pgfscope}%
\begin{pgfscope}%
\pgfsetbuttcap%
\pgfsetroundjoin%
\definecolor{currentfill}{rgb}{0.000000,0.000000,0.000000}%
\pgfsetfillcolor{currentfill}%
\pgfsetlinewidth{0.803000pt}%
\definecolor{currentstroke}{rgb}{0.000000,0.000000,0.000000}%
\pgfsetstrokecolor{currentstroke}%
\pgfsetdash{}{0pt}%
\pgfsys@defobject{currentmarker}{\pgfqpoint{0.000000in}{-0.048611in}}{\pgfqpoint{0.000000in}{0.000000in}}{%
\pgfpathmoveto{\pgfqpoint{0.000000in}{0.000000in}}%
\pgfpathlineto{\pgfqpoint{0.000000in}{-0.048611in}}%
\pgfusepath{stroke,fill}%
}%
\begin{pgfscope}%
\pgfsys@transformshift{1.025455in}{0.567600in}%
\pgfsys@useobject{currentmarker}{}%
\end{pgfscope}%
\end{pgfscope}%
\begin{pgfscope}%
\definecolor{textcolor}{rgb}{0.000000,0.000000,0.000000}%
\pgfsetstrokecolor{textcolor}%
\pgfsetfillcolor{textcolor}%
\pgftext[x=1.025455in,y=0.470378in,,top]{\color{textcolor}\sffamily\fontsize{10.000000}{12.000000}\selectfont \ensuremath{-}20}%
\end{pgfscope}%
\begin{pgfscope}%
\pgfsetbuttcap%
\pgfsetroundjoin%
\definecolor{currentfill}{rgb}{0.000000,0.000000,0.000000}%
\pgfsetfillcolor{currentfill}%
\pgfsetlinewidth{0.803000pt}%
\definecolor{currentstroke}{rgb}{0.000000,0.000000,0.000000}%
\pgfsetstrokecolor{currentstroke}%
\pgfsetdash{}{0pt}%
\pgfsys@defobject{currentmarker}{\pgfqpoint{0.000000in}{-0.048611in}}{\pgfqpoint{0.000000in}{0.000000in}}{%
\pgfpathmoveto{\pgfqpoint{0.000000in}{0.000000in}}%
\pgfpathlineto{\pgfqpoint{0.000000in}{-0.048611in}}%
\pgfusepath{stroke,fill}%
}%
\begin{pgfscope}%
\pgfsys@transformshift{1.783285in}{0.567600in}%
\pgfsys@useobject{currentmarker}{}%
\end{pgfscope}%
\end{pgfscope}%
\begin{pgfscope}%
\definecolor{textcolor}{rgb}{0.000000,0.000000,0.000000}%
\pgfsetstrokecolor{textcolor}%
\pgfsetfillcolor{textcolor}%
\pgftext[x=1.783285in,y=0.470378in,,top]{\color{textcolor}\sffamily\fontsize{10.000000}{12.000000}\selectfont 0}%
\end{pgfscope}%
\begin{pgfscope}%
\pgfsetbuttcap%
\pgfsetroundjoin%
\definecolor{currentfill}{rgb}{0.000000,0.000000,0.000000}%
\pgfsetfillcolor{currentfill}%
\pgfsetlinewidth{0.803000pt}%
\definecolor{currentstroke}{rgb}{0.000000,0.000000,0.000000}%
\pgfsetstrokecolor{currentstroke}%
\pgfsetdash{}{0pt}%
\pgfsys@defobject{currentmarker}{\pgfqpoint{0.000000in}{-0.048611in}}{\pgfqpoint{0.000000in}{0.000000in}}{%
\pgfpathmoveto{\pgfqpoint{0.000000in}{0.000000in}}%
\pgfpathlineto{\pgfqpoint{0.000000in}{-0.048611in}}%
\pgfusepath{stroke,fill}%
}%
\begin{pgfscope}%
\pgfsys@transformshift{2.541115in}{0.567600in}%
\pgfsys@useobject{currentmarker}{}%
\end{pgfscope}%
\end{pgfscope}%
\begin{pgfscope}%
\definecolor{textcolor}{rgb}{0.000000,0.000000,0.000000}%
\pgfsetstrokecolor{textcolor}%
\pgfsetfillcolor{textcolor}%
\pgftext[x=2.541115in,y=0.470378in,,top]{\color{textcolor}\sffamily\fontsize{10.000000}{12.000000}\selectfont 20}%
\end{pgfscope}%
\begin{pgfscope}%
\pgfsetbuttcap%
\pgfsetroundjoin%
\definecolor{currentfill}{rgb}{0.000000,0.000000,0.000000}%
\pgfsetfillcolor{currentfill}%
\pgfsetlinewidth{0.803000pt}%
\definecolor{currentstroke}{rgb}{0.000000,0.000000,0.000000}%
\pgfsetstrokecolor{currentstroke}%
\pgfsetdash{}{0pt}%
\pgfsys@defobject{currentmarker}{\pgfqpoint{0.000000in}{-0.048611in}}{\pgfqpoint{0.000000in}{0.000000in}}{%
\pgfpathmoveto{\pgfqpoint{0.000000in}{0.000000in}}%
\pgfpathlineto{\pgfqpoint{0.000000in}{-0.048611in}}%
\pgfusepath{stroke,fill}%
}%
\begin{pgfscope}%
\pgfsys@transformshift{3.298946in}{0.567600in}%
\pgfsys@useobject{currentmarker}{}%
\end{pgfscope}%
\end{pgfscope}%
\begin{pgfscope}%
\definecolor{textcolor}{rgb}{0.000000,0.000000,0.000000}%
\pgfsetstrokecolor{textcolor}%
\pgfsetfillcolor{textcolor}%
\pgftext[x=3.298946in,y=0.470378in,,top]{\color{textcolor}\sffamily\fontsize{10.000000}{12.000000}\selectfont 40}%
\end{pgfscope}%
\begin{pgfscope}%
\pgfsetbuttcap%
\pgfsetroundjoin%
\definecolor{currentfill}{rgb}{0.000000,0.000000,0.000000}%
\pgfsetfillcolor{currentfill}%
\pgfsetlinewidth{0.803000pt}%
\definecolor{currentstroke}{rgb}{0.000000,0.000000,0.000000}%
\pgfsetstrokecolor{currentstroke}%
\pgfsetdash{}{0pt}%
\pgfsys@defobject{currentmarker}{\pgfqpoint{0.000000in}{-0.048611in}}{\pgfqpoint{0.000000in}{0.000000in}}{%
\pgfpathmoveto{\pgfqpoint{0.000000in}{0.000000in}}%
\pgfpathlineto{\pgfqpoint{0.000000in}{-0.048611in}}%
\pgfusepath{stroke,fill}%
}%
\begin{pgfscope}%
\pgfsys@transformshift{4.056776in}{0.567600in}%
\pgfsys@useobject{currentmarker}{}%
\end{pgfscope}%
\end{pgfscope}%
\begin{pgfscope}%
\definecolor{textcolor}{rgb}{0.000000,0.000000,0.000000}%
\pgfsetstrokecolor{textcolor}%
\pgfsetfillcolor{textcolor}%
\pgftext[x=4.056776in,y=0.470378in,,top]{\color{textcolor}\sffamily\fontsize{10.000000}{12.000000}\selectfont 60}%
\end{pgfscope}%
\begin{pgfscope}%
\pgfsetbuttcap%
\pgfsetroundjoin%
\definecolor{currentfill}{rgb}{0.000000,0.000000,0.000000}%
\pgfsetfillcolor{currentfill}%
\pgfsetlinewidth{0.803000pt}%
\definecolor{currentstroke}{rgb}{0.000000,0.000000,0.000000}%
\pgfsetstrokecolor{currentstroke}%
\pgfsetdash{}{0pt}%
\pgfsys@defobject{currentmarker}{\pgfqpoint{0.000000in}{-0.048611in}}{\pgfqpoint{0.000000in}{0.000000in}}{%
\pgfpathmoveto{\pgfqpoint{0.000000in}{0.000000in}}%
\pgfpathlineto{\pgfqpoint{0.000000in}{-0.048611in}}%
\pgfusepath{stroke,fill}%
}%
\begin{pgfscope}%
\pgfsys@transformshift{4.814607in}{0.567600in}%
\pgfsys@useobject{currentmarker}{}%
\end{pgfscope}%
\end{pgfscope}%
\begin{pgfscope}%
\definecolor{textcolor}{rgb}{0.000000,0.000000,0.000000}%
\pgfsetstrokecolor{textcolor}%
\pgfsetfillcolor{textcolor}%
\pgftext[x=4.814607in,y=0.470378in,,top]{\color{textcolor}\sffamily\fontsize{10.000000}{12.000000}\selectfont 80}%
\end{pgfscope}%
\begin{pgfscope}%
\pgfsetbuttcap%
\pgfsetroundjoin%
\definecolor{currentfill}{rgb}{0.000000,0.000000,0.000000}%
\pgfsetfillcolor{currentfill}%
\pgfsetlinewidth{0.803000pt}%
\definecolor{currentstroke}{rgb}{0.000000,0.000000,0.000000}%
\pgfsetstrokecolor{currentstroke}%
\pgfsetdash{}{0pt}%
\pgfsys@defobject{currentmarker}{\pgfqpoint{0.000000in}{-0.048611in}}{\pgfqpoint{0.000000in}{0.000000in}}{%
\pgfpathmoveto{\pgfqpoint{0.000000in}{0.000000in}}%
\pgfpathlineto{\pgfqpoint{0.000000in}{-0.048611in}}%
\pgfusepath{stroke,fill}%
}%
\begin{pgfscope}%
\pgfsys@transformshift{5.572437in}{0.567600in}%
\pgfsys@useobject{currentmarker}{}%
\end{pgfscope}%
\end{pgfscope}%
\begin{pgfscope}%
\definecolor{textcolor}{rgb}{0.000000,0.000000,0.000000}%
\pgfsetstrokecolor{textcolor}%
\pgfsetfillcolor{textcolor}%
\pgftext[x=5.572437in,y=0.470378in,,top]{\color{textcolor}\sffamily\fontsize{10.000000}{12.000000}\selectfont 100}%
\end{pgfscope}%
\begin{pgfscope}%
\definecolor{textcolor}{rgb}{0.000000,0.000000,0.000000}%
\pgfsetstrokecolor{textcolor}%
\pgfsetfillcolor{textcolor}%
\pgftext[x=3.280000in,y=0.280409in,,top]{\color{textcolor}\sffamily\fontsize{10.000000}{12.000000}\selectfont n}%
\end{pgfscope}%
\begin{pgfscope}%
\pgfsetbuttcap%
\pgfsetroundjoin%
\definecolor{currentfill}{rgb}{0.000000,0.000000,0.000000}%
\pgfsetfillcolor{currentfill}%
\pgfsetlinewidth{0.803000pt}%
\definecolor{currentstroke}{rgb}{0.000000,0.000000,0.000000}%
\pgfsetstrokecolor{currentstroke}%
\pgfsetdash{}{0pt}%
\pgfsys@defobject{currentmarker}{\pgfqpoint{-0.048611in}{0.000000in}}{\pgfqpoint{-0.000000in}{0.000000in}}{%
\pgfpathmoveto{\pgfqpoint{-0.000000in}{0.000000in}}%
\pgfpathlineto{\pgfqpoint{-0.048611in}{0.000000in}}%
\pgfusepath{stroke,fill}%
}%
\begin{pgfscope}%
\pgfsys@transformshift{0.800000in}{0.827287in}%
\pgfsys@useobject{currentmarker}{}%
\end{pgfscope}%
\end{pgfscope}%
\begin{pgfscope}%
\definecolor{textcolor}{rgb}{0.000000,0.000000,0.000000}%
\pgfsetstrokecolor{textcolor}%
\pgfsetfillcolor{textcolor}%
\pgftext[x=0.373873in, y=0.774526in, left, base]{\color{textcolor}\sffamily\fontsize{10.000000}{12.000000}\selectfont \ensuremath{-}1.0}%
\end{pgfscope}%
\begin{pgfscope}%
\pgfsetbuttcap%
\pgfsetroundjoin%
\definecolor{currentfill}{rgb}{0.000000,0.000000,0.000000}%
\pgfsetfillcolor{currentfill}%
\pgfsetlinewidth{0.803000pt}%
\definecolor{currentstroke}{rgb}{0.000000,0.000000,0.000000}%
\pgfsetstrokecolor{currentstroke}%
\pgfsetdash{}{0pt}%
\pgfsys@defobject{currentmarker}{\pgfqpoint{-0.048611in}{0.000000in}}{\pgfqpoint{-0.000000in}{0.000000in}}{%
\pgfpathmoveto{\pgfqpoint{-0.000000in}{0.000000in}}%
\pgfpathlineto{\pgfqpoint{-0.048611in}{0.000000in}}%
\pgfusepath{stroke,fill}%
}%
\begin{pgfscope}%
\pgfsys@transformshift{0.800000in}{1.683164in}%
\pgfsys@useobject{currentmarker}{}%
\end{pgfscope}%
\end{pgfscope}%
\begin{pgfscope}%
\definecolor{textcolor}{rgb}{0.000000,0.000000,0.000000}%
\pgfsetstrokecolor{textcolor}%
\pgfsetfillcolor{textcolor}%
\pgftext[x=0.373873in, y=1.630402in, left, base]{\color{textcolor}\sffamily\fontsize{10.000000}{12.000000}\selectfont \ensuremath{-}0.5}%
\end{pgfscope}%
\begin{pgfscope}%
\pgfsetbuttcap%
\pgfsetroundjoin%
\definecolor{currentfill}{rgb}{0.000000,0.000000,0.000000}%
\pgfsetfillcolor{currentfill}%
\pgfsetlinewidth{0.803000pt}%
\definecolor{currentstroke}{rgb}{0.000000,0.000000,0.000000}%
\pgfsetstrokecolor{currentstroke}%
\pgfsetdash{}{0pt}%
\pgfsys@defobject{currentmarker}{\pgfqpoint{-0.048611in}{0.000000in}}{\pgfqpoint{-0.000000in}{0.000000in}}{%
\pgfpathmoveto{\pgfqpoint{-0.000000in}{0.000000in}}%
\pgfpathlineto{\pgfqpoint{-0.048611in}{0.000000in}}%
\pgfusepath{stroke,fill}%
}%
\begin{pgfscope}%
\pgfsys@transformshift{0.800000in}{2.539041in}%
\pgfsys@useobject{currentmarker}{}%
\end{pgfscope}%
\end{pgfscope}%
\begin{pgfscope}%
\definecolor{textcolor}{rgb}{0.000000,0.000000,0.000000}%
\pgfsetstrokecolor{textcolor}%
\pgfsetfillcolor{textcolor}%
\pgftext[x=0.481898in, y=2.486279in, left, base]{\color{textcolor}\sffamily\fontsize{10.000000}{12.000000}\selectfont 0.0}%
\end{pgfscope}%
\begin{pgfscope}%
\pgfsetbuttcap%
\pgfsetroundjoin%
\definecolor{currentfill}{rgb}{0.000000,0.000000,0.000000}%
\pgfsetfillcolor{currentfill}%
\pgfsetlinewidth{0.803000pt}%
\definecolor{currentstroke}{rgb}{0.000000,0.000000,0.000000}%
\pgfsetstrokecolor{currentstroke}%
\pgfsetdash{}{0pt}%
\pgfsys@defobject{currentmarker}{\pgfqpoint{-0.048611in}{0.000000in}}{\pgfqpoint{-0.000000in}{0.000000in}}{%
\pgfpathmoveto{\pgfqpoint{-0.000000in}{0.000000in}}%
\pgfpathlineto{\pgfqpoint{-0.048611in}{0.000000in}}%
\pgfusepath{stroke,fill}%
}%
\begin{pgfscope}%
\pgfsys@transformshift{0.800000in}{3.394918in}%
\pgfsys@useobject{currentmarker}{}%
\end{pgfscope}%
\end{pgfscope}%
\begin{pgfscope}%
\definecolor{textcolor}{rgb}{0.000000,0.000000,0.000000}%
\pgfsetstrokecolor{textcolor}%
\pgfsetfillcolor{textcolor}%
\pgftext[x=0.481898in, y=3.342156in, left, base]{\color{textcolor}\sffamily\fontsize{10.000000}{12.000000}\selectfont 0.5}%
\end{pgfscope}%
\begin{pgfscope}%
\pgfsetbuttcap%
\pgfsetroundjoin%
\definecolor{currentfill}{rgb}{0.000000,0.000000,0.000000}%
\pgfsetfillcolor{currentfill}%
\pgfsetlinewidth{0.803000pt}%
\definecolor{currentstroke}{rgb}{0.000000,0.000000,0.000000}%
\pgfsetstrokecolor{currentstroke}%
\pgfsetdash{}{0pt}%
\pgfsys@defobject{currentmarker}{\pgfqpoint{-0.048611in}{0.000000in}}{\pgfqpoint{-0.000000in}{0.000000in}}{%
\pgfpathmoveto{\pgfqpoint{-0.000000in}{0.000000in}}%
\pgfpathlineto{\pgfqpoint{-0.048611in}{0.000000in}}%
\pgfusepath{stroke,fill}%
}%
\begin{pgfscope}%
\pgfsys@transformshift{0.800000in}{4.250794in}%
\pgfsys@useobject{currentmarker}{}%
\end{pgfscope}%
\end{pgfscope}%
\begin{pgfscope}%
\definecolor{textcolor}{rgb}{0.000000,0.000000,0.000000}%
\pgfsetstrokecolor{textcolor}%
\pgfsetfillcolor{textcolor}%
\pgftext[x=0.481898in, y=4.198033in, left, base]{\color{textcolor}\sffamily\fontsize{10.000000}{12.000000}\selectfont 1.0}%
\end{pgfscope}%
\begin{pgfscope}%
\definecolor{textcolor}{rgb}{0.000000,0.000000,0.000000}%
\pgfsetstrokecolor{textcolor}%
\pgfsetfillcolor{textcolor}%
\pgftext[x=0.318318in,y=2.554200in,,bottom,rotate=90.000000]{\color{textcolor}\sffamily\fontsize{10.000000}{12.000000}\selectfont x[n] * m[n]}%
\end{pgfscope}%
\begin{pgfscope}%
\pgfpathrectangle{\pgfqpoint{0.800000in}{0.567600in}}{\pgfqpoint{4.960000in}{3.973200in}}%
\pgfusepath{clip}%
\pgfsetrectcap%
\pgfsetroundjoin%
\pgfsetlinewidth{1.505625pt}%
\definecolor{currentstroke}{rgb}{0.121569,0.466667,0.705882}%
\pgfsetstrokecolor{currentstroke}%
\pgfsetdash{}{0pt}%
\pgfpathmoveto{\pgfqpoint{1.025455in}{2.183312in}}%
\pgfpathlineto{\pgfqpoint{1.063346in}{1.829667in}}%
\pgfpathlineto{\pgfqpoint{1.101238in}{1.536928in}}%
\pgfpathlineto{\pgfqpoint{1.139129in}{1.192285in}}%
\pgfpathlineto{\pgfqpoint{1.177021in}{0.876394in}}%
\pgfpathlineto{\pgfqpoint{1.214912in}{0.844757in}}%
\pgfpathlineto{\pgfqpoint{1.252804in}{0.804119in}}%
\pgfpathlineto{\pgfqpoint{1.290695in}{0.781913in}}%
\pgfpathlineto{\pgfqpoint{1.328587in}{0.836495in}}%
\pgfpathlineto{\pgfqpoint{1.366478in}{0.825492in}}%
\pgfpathlineto{\pgfqpoint{1.404370in}{0.907679in}}%
\pgfpathlineto{\pgfqpoint{1.442261in}{0.996321in}}%
\pgfpathlineto{\pgfqpoint{1.480153in}{0.947532in}}%
\pgfpathlineto{\pgfqpoint{1.518044in}{0.900907in}}%
\pgfpathlineto{\pgfqpoint{1.555936in}{0.875382in}}%
\pgfpathlineto{\pgfqpoint{1.593827in}{0.838851in}}%
\pgfpathlineto{\pgfqpoint{1.631719in}{0.846953in}}%
\pgfpathlineto{\pgfqpoint{1.669610in}{0.879259in}}%
\pgfpathlineto{\pgfqpoint{1.707502in}{0.903207in}}%
\pgfpathlineto{\pgfqpoint{1.745393in}{0.901532in}}%
\pgfpathlineto{\pgfqpoint{1.783285in}{1.640464in}}%
\pgfpathlineto{\pgfqpoint{1.821176in}{2.292300in}}%
\pgfpathlineto{\pgfqpoint{1.859068in}{2.920752in}}%
\pgfpathlineto{\pgfqpoint{1.896960in}{3.530107in}}%
\pgfpathlineto{\pgfqpoint{1.934851in}{4.204180in}}%
\pgfpathlineto{\pgfqpoint{1.972743in}{4.092452in}}%
\pgfpathlineto{\pgfqpoint{2.010634in}{4.126623in}}%
\pgfpathlineto{\pgfqpoint{2.048526in}{4.162227in}}%
\pgfpathlineto{\pgfqpoint{2.086417in}{4.254850in}}%
\pgfpathlineto{\pgfqpoint{2.124309in}{4.290652in}}%
\pgfpathlineto{\pgfqpoint{2.162200in}{4.293828in}}%
\pgfpathlineto{\pgfqpoint{2.200092in}{4.286815in}}%
\pgfpathlineto{\pgfqpoint{2.237983in}{4.290671in}}%
\pgfpathlineto{\pgfqpoint{2.275875in}{4.247210in}}%
\pgfpathlineto{\pgfqpoint{2.313766in}{4.176886in}}%
\pgfpathlineto{\pgfqpoint{2.351658in}{4.177459in}}%
\pgfpathlineto{\pgfqpoint{2.389549in}{4.135949in}}%
\pgfpathlineto{\pgfqpoint{2.427441in}{4.198552in}}%
\pgfpathlineto{\pgfqpoint{2.465332in}{4.202875in}}%
\pgfpathlineto{\pgfqpoint{2.503224in}{4.253716in}}%
\pgfpathlineto{\pgfqpoint{2.541115in}{4.265607in}}%
\pgfpathlineto{\pgfqpoint{2.579007in}{4.301287in}}%
\pgfpathlineto{\pgfqpoint{2.616898in}{4.227391in}}%
\pgfpathlineto{\pgfqpoint{2.654790in}{4.164448in}}%
\pgfpathlineto{\pgfqpoint{2.692681in}{4.118058in}}%
\pgfpathlineto{\pgfqpoint{2.730573in}{3.516029in}}%
\pgfpathlineto{\pgfqpoint{2.768464in}{2.776215in}}%
\pgfpathlineto{\pgfqpoint{2.806356in}{2.126465in}}%
\pgfpathlineto{\pgfqpoint{2.844248in}{1.553646in}}%
\pgfpathlineto{\pgfqpoint{2.882139in}{0.922819in}}%
\pgfpathlineto{\pgfqpoint{2.920031in}{0.847105in}}%
\pgfpathlineto{\pgfqpoint{2.957922in}{0.820734in}}%
\pgfpathlineto{\pgfqpoint{2.995814in}{0.829883in}}%
\pgfpathlineto{\pgfqpoint{3.033705in}{0.826176in}}%
\pgfpathlineto{\pgfqpoint{3.071597in}{0.861928in}}%
\pgfpathlineto{\pgfqpoint{3.109488in}{0.870199in}}%
\pgfpathlineto{\pgfqpoint{3.147380in}{0.899645in}}%
\pgfpathlineto{\pgfqpoint{3.185271in}{0.914836in}}%
\pgfpathlineto{\pgfqpoint{3.223163in}{0.859666in}}%
\pgfpathlineto{\pgfqpoint{3.261054in}{0.822372in}}%
\pgfpathlineto{\pgfqpoint{3.298946in}{0.839985in}}%
\pgfpathlineto{\pgfqpoint{3.336837in}{0.850555in}}%
\pgfpathlineto{\pgfqpoint{3.374729in}{0.806784in}}%
\pgfpathlineto{\pgfqpoint{3.412620in}{0.861319in}}%
\pgfpathlineto{\pgfqpoint{3.450512in}{0.865003in}}%
\pgfpathlineto{\pgfqpoint{3.488403in}{0.915409in}}%
\pgfpathlineto{\pgfqpoint{3.526295in}{0.872519in}}%
\pgfpathlineto{\pgfqpoint{3.564186in}{0.898363in}}%
\pgfpathlineto{\pgfqpoint{3.602078in}{0.824566in}}%
\pgfpathlineto{\pgfqpoint{3.639969in}{0.877425in}}%
\pgfpathlineto{\pgfqpoint{3.677861in}{1.515823in}}%
\pgfpathlineto{\pgfqpoint{3.715752in}{2.237625in}}%
\pgfpathlineto{\pgfqpoint{3.753644in}{2.873059in}}%
\pgfpathlineto{\pgfqpoint{3.791536in}{3.528449in}}%
\pgfpathlineto{\pgfqpoint{3.829427in}{4.175305in}}%
\pgfpathlineto{\pgfqpoint{3.867319in}{4.222808in}}%
\pgfpathlineto{\pgfqpoint{3.905210in}{4.270338in}}%
\pgfpathlineto{\pgfqpoint{3.943102in}{4.246630in}}%
\pgfpathlineto{\pgfqpoint{3.980993in}{4.327123in}}%
\pgfpathlineto{\pgfqpoint{4.018885in}{4.302459in}}%
\pgfpathlineto{\pgfqpoint{4.056776in}{4.265078in}}%
\pgfpathlineto{\pgfqpoint{4.094668in}{4.262112in}}%
\pgfpathlineto{\pgfqpoint{4.132559in}{4.332439in}}%
\pgfpathlineto{\pgfqpoint{4.170451in}{4.323787in}}%
\pgfpathlineto{\pgfqpoint{4.208342in}{4.360200in}}%
\pgfpathlineto{\pgfqpoint{4.246234in}{4.324612in}}%
\pgfpathlineto{\pgfqpoint{4.284125in}{4.290608in}}%
\pgfpathlineto{\pgfqpoint{4.322017in}{4.250869in}}%
\pgfpathlineto{\pgfqpoint{4.359908in}{4.199843in}}%
\pgfpathlineto{\pgfqpoint{4.397800in}{4.112804in}}%
\pgfpathlineto{\pgfqpoint{4.435691in}{4.116473in}}%
\pgfpathlineto{\pgfqpoint{4.473583in}{4.176556in}}%
\pgfpathlineto{\pgfqpoint{4.511474in}{4.243415in}}%
\pgfpathlineto{\pgfqpoint{4.549366in}{4.261271in}}%
\pgfpathlineto{\pgfqpoint{4.587257in}{4.331057in}}%
\pgfpathlineto{\pgfqpoint{4.625149in}{3.633920in}}%
\pgfpathlineto{\pgfqpoint{4.663040in}{2.944217in}}%
\pgfpathlineto{\pgfqpoint{4.700932in}{2.263495in}}%
\pgfpathlineto{\pgfqpoint{4.738824in}{1.547737in}}%
\pgfpathlineto{\pgfqpoint{4.776715in}{0.897506in}}%
\pgfpathlineto{\pgfqpoint{4.814607in}{0.988307in}}%
\pgfpathlineto{\pgfqpoint{4.852498in}{0.968362in}}%
\pgfpathlineto{\pgfqpoint{4.890390in}{0.909388in}}%
\pgfpathlineto{\pgfqpoint{4.928281in}{0.956799in}}%
\pgfpathlineto{\pgfqpoint{4.966173in}{0.891010in}}%
\pgfpathlineto{\pgfqpoint{5.004064in}{0.883539in}}%
\pgfpathlineto{\pgfqpoint{5.041956in}{0.878229in}}%
\pgfpathlineto{\pgfqpoint{5.079847in}{0.857197in}}%
\pgfpathlineto{\pgfqpoint{5.117739in}{0.811756in}}%
\pgfpathlineto{\pgfqpoint{5.155630in}{0.825063in}}%
\pgfpathlineto{\pgfqpoint{5.193522in}{0.801617in}}%
\pgfpathlineto{\pgfqpoint{5.231413in}{0.824646in}}%
\pgfpathlineto{\pgfqpoint{5.269305in}{0.885416in}}%
\pgfpathlineto{\pgfqpoint{5.307196in}{0.910771in}}%
\pgfpathlineto{\pgfqpoint{5.345088in}{0.895366in}}%
\pgfpathlineto{\pgfqpoint{5.382979in}{0.850499in}}%
\pgfpathlineto{\pgfqpoint{5.420871in}{0.792044in}}%
\pgfpathlineto{\pgfqpoint{5.458762in}{0.748200in}}%
\pgfpathlineto{\pgfqpoint{5.496654in}{0.752472in}}%
\pgfpathlineto{\pgfqpoint{5.534545in}{0.817971in}}%
\pgfusepath{stroke}%
\end{pgfscope}%
\begin{pgfscope}%
\pgfsetrectcap%
\pgfsetmiterjoin%
\pgfsetlinewidth{0.803000pt}%
\definecolor{currentstroke}{rgb}{0.000000,0.000000,0.000000}%
\pgfsetstrokecolor{currentstroke}%
\pgfsetdash{}{0pt}%
\pgfpathmoveto{\pgfqpoint{0.800000in}{0.567600in}}%
\pgfpathlineto{\pgfqpoint{0.800000in}{4.540800in}}%
\pgfusepath{stroke}%
\end{pgfscope}%
\begin{pgfscope}%
\pgfsetrectcap%
\pgfsetmiterjoin%
\pgfsetlinewidth{0.803000pt}%
\definecolor{currentstroke}{rgb}{0.000000,0.000000,0.000000}%
\pgfsetstrokecolor{currentstroke}%
\pgfsetdash{}{0pt}%
\pgfpathmoveto{\pgfqpoint{5.760000in}{0.567600in}}%
\pgfpathlineto{\pgfqpoint{5.760000in}{4.540800in}}%
\pgfusepath{stroke}%
\end{pgfscope}%
\begin{pgfscope}%
\pgfsetrectcap%
\pgfsetmiterjoin%
\pgfsetlinewidth{0.803000pt}%
\definecolor{currentstroke}{rgb}{0.000000,0.000000,0.000000}%
\pgfsetstrokecolor{currentstroke}%
\pgfsetdash{}{0pt}%
\pgfpathmoveto{\pgfqpoint{0.800000in}{0.567600in}}%
\pgfpathlineto{\pgfqpoint{5.760000in}{0.567600in}}%
\pgfusepath{stroke}%
\end{pgfscope}%
\begin{pgfscope}%
\pgfsetrectcap%
\pgfsetmiterjoin%
\pgfsetlinewidth{0.803000pt}%
\definecolor{currentstroke}{rgb}{0.000000,0.000000,0.000000}%
\pgfsetstrokecolor{currentstroke}%
\pgfsetdash{}{0pt}%
\pgfpathmoveto{\pgfqpoint{0.800000in}{4.540800in}}%
\pgfpathlineto{\pgfqpoint{5.760000in}{4.540800in}}%
\pgfusepath{stroke}%
\end{pgfscope}%
\begin{pgfscope}%
\definecolor{textcolor}{rgb}{0.000000,0.000000,0.000000}%
\pgfsetstrokecolor{textcolor}%
\pgfsetfillcolor{textcolor}%
\pgftext[x=3.280000in,y=4.624133in,,base]{\color{textcolor}\sffamily\fontsize{12.000000}{14.400000}\selectfont n vs x[n] * m[n] (N=5)}%
\end{pgfscope}%
\end{pgfpicture}%
\makeatother%
\endgroup%

        \newline
        %% Creator: Matplotlib, PGF backend
%%
%% To include the figure in your LaTeX document, write
%%   \input{<filename>.pgf}
%%
%% Make sure the required packages are loaded in your preamble
%%   \usepackage{pgf}
%%
%% Also ensure that all the required font packages are loaded; for instance,
%% the lmodern package is sometimes necessary when using math font.
%%   \usepackage{lmodern}
%%
%% Figures using additional raster images can only be included by \input if
%% they are in the same directory as the main LaTeX file. For loading figures
%% from other directories you can use the `import` package
%%   \usepackage{import}
%%
%% and then include the figures with
%%   \import{<path to file>}{<filename>.pgf}
%%
%% Matplotlib used the following preamble
%%   
%%   \usepackage{fontspec}
%%   \setmainfont{DejaVuSerif.ttf}[Path=\detokenize{/home/aeren/.local/lib/python3.10/site-packages/matplotlib/mpl-data/fonts/ttf/}]
%%   \setsansfont{DejaVuSans.ttf}[Path=\detokenize{/home/aeren/.local/lib/python3.10/site-packages/matplotlib/mpl-data/fonts/ttf/}]
%%   \setmonofont{DejaVuSansMono.ttf}[Path=\detokenize{/home/aeren/.local/lib/python3.10/site-packages/matplotlib/mpl-data/fonts/ttf/}]
%%   \makeatletter\@ifpackageloaded{underscore}{}{\usepackage[strings]{underscore}}\makeatother
%%
\begingroup%
\makeatletter%
\begin{pgfpicture}%
\pgfpathrectangle{\pgfpointorigin}{\pgfqpoint{6.400000in}{5.160000in}}%
\pgfusepath{use as bounding box, clip}%
\begin{pgfscope}%
\pgfsetbuttcap%
\pgfsetmiterjoin%
\definecolor{currentfill}{rgb}{1.000000,1.000000,1.000000}%
\pgfsetfillcolor{currentfill}%
\pgfsetlinewidth{0.000000pt}%
\definecolor{currentstroke}{rgb}{1.000000,1.000000,1.000000}%
\pgfsetstrokecolor{currentstroke}%
\pgfsetdash{}{0pt}%
\pgfpathmoveto{\pgfqpoint{0.000000in}{0.000000in}}%
\pgfpathlineto{\pgfqpoint{6.400000in}{0.000000in}}%
\pgfpathlineto{\pgfqpoint{6.400000in}{5.160000in}}%
\pgfpathlineto{\pgfqpoint{0.000000in}{5.160000in}}%
\pgfpathlineto{\pgfqpoint{0.000000in}{0.000000in}}%
\pgfpathclose%
\pgfusepath{fill}%
\end{pgfscope}%
\begin{pgfscope}%
\pgfsetbuttcap%
\pgfsetmiterjoin%
\definecolor{currentfill}{rgb}{1.000000,1.000000,1.000000}%
\pgfsetfillcolor{currentfill}%
\pgfsetlinewidth{0.000000pt}%
\definecolor{currentstroke}{rgb}{0.000000,0.000000,0.000000}%
\pgfsetstrokecolor{currentstroke}%
\pgfsetstrokeopacity{0.000000}%
\pgfsetdash{}{0pt}%
\pgfpathmoveto{\pgfqpoint{0.800000in}{0.567600in}}%
\pgfpathlineto{\pgfqpoint{5.760000in}{0.567600in}}%
\pgfpathlineto{\pgfqpoint{5.760000in}{4.540800in}}%
\pgfpathlineto{\pgfqpoint{0.800000in}{4.540800in}}%
\pgfpathlineto{\pgfqpoint{0.800000in}{0.567600in}}%
\pgfpathclose%
\pgfusepath{fill}%
\end{pgfscope}%
\begin{pgfscope}%
\pgfsetbuttcap%
\pgfsetroundjoin%
\definecolor{currentfill}{rgb}{0.000000,0.000000,0.000000}%
\pgfsetfillcolor{currentfill}%
\pgfsetlinewidth{0.803000pt}%
\definecolor{currentstroke}{rgb}{0.000000,0.000000,0.000000}%
\pgfsetstrokecolor{currentstroke}%
\pgfsetdash{}{0pt}%
\pgfsys@defobject{currentmarker}{\pgfqpoint{0.000000in}{-0.048611in}}{\pgfqpoint{0.000000in}{0.000000in}}{%
\pgfpathmoveto{\pgfqpoint{0.000000in}{0.000000in}}%
\pgfpathlineto{\pgfqpoint{0.000000in}{-0.048611in}}%
\pgfusepath{stroke,fill}%
}%
\begin{pgfscope}%
\pgfsys@transformshift{1.025455in}{0.567600in}%
\pgfsys@useobject{currentmarker}{}%
\end{pgfscope}%
\end{pgfscope}%
\begin{pgfscope}%
\definecolor{textcolor}{rgb}{0.000000,0.000000,0.000000}%
\pgfsetstrokecolor{textcolor}%
\pgfsetfillcolor{textcolor}%
\pgftext[x=1.025455in,y=0.470378in,,top]{\color{textcolor}\sffamily\fontsize{10.000000}{12.000000}\selectfont \ensuremath{-}20}%
\end{pgfscope}%
\begin{pgfscope}%
\pgfsetbuttcap%
\pgfsetroundjoin%
\definecolor{currentfill}{rgb}{0.000000,0.000000,0.000000}%
\pgfsetfillcolor{currentfill}%
\pgfsetlinewidth{0.803000pt}%
\definecolor{currentstroke}{rgb}{0.000000,0.000000,0.000000}%
\pgfsetstrokecolor{currentstroke}%
\pgfsetdash{}{0pt}%
\pgfsys@defobject{currentmarker}{\pgfqpoint{0.000000in}{-0.048611in}}{\pgfqpoint{0.000000in}{0.000000in}}{%
\pgfpathmoveto{\pgfqpoint{0.000000in}{0.000000in}}%
\pgfpathlineto{\pgfqpoint{0.000000in}{-0.048611in}}%
\pgfusepath{stroke,fill}%
}%
\begin{pgfscope}%
\pgfsys@transformshift{1.783285in}{0.567600in}%
\pgfsys@useobject{currentmarker}{}%
\end{pgfscope}%
\end{pgfscope}%
\begin{pgfscope}%
\definecolor{textcolor}{rgb}{0.000000,0.000000,0.000000}%
\pgfsetstrokecolor{textcolor}%
\pgfsetfillcolor{textcolor}%
\pgftext[x=1.783285in,y=0.470378in,,top]{\color{textcolor}\sffamily\fontsize{10.000000}{12.000000}\selectfont 0}%
\end{pgfscope}%
\begin{pgfscope}%
\pgfsetbuttcap%
\pgfsetroundjoin%
\definecolor{currentfill}{rgb}{0.000000,0.000000,0.000000}%
\pgfsetfillcolor{currentfill}%
\pgfsetlinewidth{0.803000pt}%
\definecolor{currentstroke}{rgb}{0.000000,0.000000,0.000000}%
\pgfsetstrokecolor{currentstroke}%
\pgfsetdash{}{0pt}%
\pgfsys@defobject{currentmarker}{\pgfqpoint{0.000000in}{-0.048611in}}{\pgfqpoint{0.000000in}{0.000000in}}{%
\pgfpathmoveto{\pgfqpoint{0.000000in}{0.000000in}}%
\pgfpathlineto{\pgfqpoint{0.000000in}{-0.048611in}}%
\pgfusepath{stroke,fill}%
}%
\begin{pgfscope}%
\pgfsys@transformshift{2.541115in}{0.567600in}%
\pgfsys@useobject{currentmarker}{}%
\end{pgfscope}%
\end{pgfscope}%
\begin{pgfscope}%
\definecolor{textcolor}{rgb}{0.000000,0.000000,0.000000}%
\pgfsetstrokecolor{textcolor}%
\pgfsetfillcolor{textcolor}%
\pgftext[x=2.541115in,y=0.470378in,,top]{\color{textcolor}\sffamily\fontsize{10.000000}{12.000000}\selectfont 20}%
\end{pgfscope}%
\begin{pgfscope}%
\pgfsetbuttcap%
\pgfsetroundjoin%
\definecolor{currentfill}{rgb}{0.000000,0.000000,0.000000}%
\pgfsetfillcolor{currentfill}%
\pgfsetlinewidth{0.803000pt}%
\definecolor{currentstroke}{rgb}{0.000000,0.000000,0.000000}%
\pgfsetstrokecolor{currentstroke}%
\pgfsetdash{}{0pt}%
\pgfsys@defobject{currentmarker}{\pgfqpoint{0.000000in}{-0.048611in}}{\pgfqpoint{0.000000in}{0.000000in}}{%
\pgfpathmoveto{\pgfqpoint{0.000000in}{0.000000in}}%
\pgfpathlineto{\pgfqpoint{0.000000in}{-0.048611in}}%
\pgfusepath{stroke,fill}%
}%
\begin{pgfscope}%
\pgfsys@transformshift{3.298946in}{0.567600in}%
\pgfsys@useobject{currentmarker}{}%
\end{pgfscope}%
\end{pgfscope}%
\begin{pgfscope}%
\definecolor{textcolor}{rgb}{0.000000,0.000000,0.000000}%
\pgfsetstrokecolor{textcolor}%
\pgfsetfillcolor{textcolor}%
\pgftext[x=3.298946in,y=0.470378in,,top]{\color{textcolor}\sffamily\fontsize{10.000000}{12.000000}\selectfont 40}%
\end{pgfscope}%
\begin{pgfscope}%
\pgfsetbuttcap%
\pgfsetroundjoin%
\definecolor{currentfill}{rgb}{0.000000,0.000000,0.000000}%
\pgfsetfillcolor{currentfill}%
\pgfsetlinewidth{0.803000pt}%
\definecolor{currentstroke}{rgb}{0.000000,0.000000,0.000000}%
\pgfsetstrokecolor{currentstroke}%
\pgfsetdash{}{0pt}%
\pgfsys@defobject{currentmarker}{\pgfqpoint{0.000000in}{-0.048611in}}{\pgfqpoint{0.000000in}{0.000000in}}{%
\pgfpathmoveto{\pgfqpoint{0.000000in}{0.000000in}}%
\pgfpathlineto{\pgfqpoint{0.000000in}{-0.048611in}}%
\pgfusepath{stroke,fill}%
}%
\begin{pgfscope}%
\pgfsys@transformshift{4.056776in}{0.567600in}%
\pgfsys@useobject{currentmarker}{}%
\end{pgfscope}%
\end{pgfscope}%
\begin{pgfscope}%
\definecolor{textcolor}{rgb}{0.000000,0.000000,0.000000}%
\pgfsetstrokecolor{textcolor}%
\pgfsetfillcolor{textcolor}%
\pgftext[x=4.056776in,y=0.470378in,,top]{\color{textcolor}\sffamily\fontsize{10.000000}{12.000000}\selectfont 60}%
\end{pgfscope}%
\begin{pgfscope}%
\pgfsetbuttcap%
\pgfsetroundjoin%
\definecolor{currentfill}{rgb}{0.000000,0.000000,0.000000}%
\pgfsetfillcolor{currentfill}%
\pgfsetlinewidth{0.803000pt}%
\definecolor{currentstroke}{rgb}{0.000000,0.000000,0.000000}%
\pgfsetstrokecolor{currentstroke}%
\pgfsetdash{}{0pt}%
\pgfsys@defobject{currentmarker}{\pgfqpoint{0.000000in}{-0.048611in}}{\pgfqpoint{0.000000in}{0.000000in}}{%
\pgfpathmoveto{\pgfqpoint{0.000000in}{0.000000in}}%
\pgfpathlineto{\pgfqpoint{0.000000in}{-0.048611in}}%
\pgfusepath{stroke,fill}%
}%
\begin{pgfscope}%
\pgfsys@transformshift{4.814607in}{0.567600in}%
\pgfsys@useobject{currentmarker}{}%
\end{pgfscope}%
\end{pgfscope}%
\begin{pgfscope}%
\definecolor{textcolor}{rgb}{0.000000,0.000000,0.000000}%
\pgfsetstrokecolor{textcolor}%
\pgfsetfillcolor{textcolor}%
\pgftext[x=4.814607in,y=0.470378in,,top]{\color{textcolor}\sffamily\fontsize{10.000000}{12.000000}\selectfont 80}%
\end{pgfscope}%
\begin{pgfscope}%
\pgfsetbuttcap%
\pgfsetroundjoin%
\definecolor{currentfill}{rgb}{0.000000,0.000000,0.000000}%
\pgfsetfillcolor{currentfill}%
\pgfsetlinewidth{0.803000pt}%
\definecolor{currentstroke}{rgb}{0.000000,0.000000,0.000000}%
\pgfsetstrokecolor{currentstroke}%
\pgfsetdash{}{0pt}%
\pgfsys@defobject{currentmarker}{\pgfqpoint{0.000000in}{-0.048611in}}{\pgfqpoint{0.000000in}{0.000000in}}{%
\pgfpathmoveto{\pgfqpoint{0.000000in}{0.000000in}}%
\pgfpathlineto{\pgfqpoint{0.000000in}{-0.048611in}}%
\pgfusepath{stroke,fill}%
}%
\begin{pgfscope}%
\pgfsys@transformshift{5.572437in}{0.567600in}%
\pgfsys@useobject{currentmarker}{}%
\end{pgfscope}%
\end{pgfscope}%
\begin{pgfscope}%
\definecolor{textcolor}{rgb}{0.000000,0.000000,0.000000}%
\pgfsetstrokecolor{textcolor}%
\pgfsetfillcolor{textcolor}%
\pgftext[x=5.572437in,y=0.470378in,,top]{\color{textcolor}\sffamily\fontsize{10.000000}{12.000000}\selectfont 100}%
\end{pgfscope}%
\begin{pgfscope}%
\definecolor{textcolor}{rgb}{0.000000,0.000000,0.000000}%
\pgfsetstrokecolor{textcolor}%
\pgfsetfillcolor{textcolor}%
\pgftext[x=3.280000in,y=0.280409in,,top]{\color{textcolor}\sffamily\fontsize{10.000000}{12.000000}\selectfont n}%
\end{pgfscope}%
\begin{pgfscope}%
\pgfsetbuttcap%
\pgfsetroundjoin%
\definecolor{currentfill}{rgb}{0.000000,0.000000,0.000000}%
\pgfsetfillcolor{currentfill}%
\pgfsetlinewidth{0.803000pt}%
\definecolor{currentstroke}{rgb}{0.000000,0.000000,0.000000}%
\pgfsetstrokecolor{currentstroke}%
\pgfsetdash{}{0pt}%
\pgfsys@defobject{currentmarker}{\pgfqpoint{-0.048611in}{0.000000in}}{\pgfqpoint{-0.000000in}{0.000000in}}{%
\pgfpathmoveto{\pgfqpoint{-0.000000in}{0.000000in}}%
\pgfpathlineto{\pgfqpoint{-0.048611in}{0.000000in}}%
\pgfusepath{stroke,fill}%
}%
\begin{pgfscope}%
\pgfsys@transformshift{0.800000in}{0.731749in}%
\pgfsys@useobject{currentmarker}{}%
\end{pgfscope}%
\end{pgfscope}%
\begin{pgfscope}%
\definecolor{textcolor}{rgb}{0.000000,0.000000,0.000000}%
\pgfsetstrokecolor{textcolor}%
\pgfsetfillcolor{textcolor}%
\pgftext[x=0.285508in, y=0.678988in, left, base]{\color{textcolor}\sffamily\fontsize{10.000000}{12.000000}\selectfont \ensuremath{-}1.00}%
\end{pgfscope}%
\begin{pgfscope}%
\pgfsetbuttcap%
\pgfsetroundjoin%
\definecolor{currentfill}{rgb}{0.000000,0.000000,0.000000}%
\pgfsetfillcolor{currentfill}%
\pgfsetlinewidth{0.803000pt}%
\definecolor{currentstroke}{rgb}{0.000000,0.000000,0.000000}%
\pgfsetstrokecolor{currentstroke}%
\pgfsetdash{}{0pt}%
\pgfsys@defobject{currentmarker}{\pgfqpoint{-0.048611in}{0.000000in}}{\pgfqpoint{-0.000000in}{0.000000in}}{%
\pgfpathmoveto{\pgfqpoint{-0.000000in}{0.000000in}}%
\pgfpathlineto{\pgfqpoint{-0.048611in}{0.000000in}}%
\pgfusepath{stroke,fill}%
}%
\begin{pgfscope}%
\pgfsys@transformshift{0.800000in}{1.181729in}%
\pgfsys@useobject{currentmarker}{}%
\end{pgfscope}%
\end{pgfscope}%
\begin{pgfscope}%
\definecolor{textcolor}{rgb}{0.000000,0.000000,0.000000}%
\pgfsetstrokecolor{textcolor}%
\pgfsetfillcolor{textcolor}%
\pgftext[x=0.285508in, y=1.128968in, left, base]{\color{textcolor}\sffamily\fontsize{10.000000}{12.000000}\selectfont \ensuremath{-}0.75}%
\end{pgfscope}%
\begin{pgfscope}%
\pgfsetbuttcap%
\pgfsetroundjoin%
\definecolor{currentfill}{rgb}{0.000000,0.000000,0.000000}%
\pgfsetfillcolor{currentfill}%
\pgfsetlinewidth{0.803000pt}%
\definecolor{currentstroke}{rgb}{0.000000,0.000000,0.000000}%
\pgfsetstrokecolor{currentstroke}%
\pgfsetdash{}{0pt}%
\pgfsys@defobject{currentmarker}{\pgfqpoint{-0.048611in}{0.000000in}}{\pgfqpoint{-0.000000in}{0.000000in}}{%
\pgfpathmoveto{\pgfqpoint{-0.000000in}{0.000000in}}%
\pgfpathlineto{\pgfqpoint{-0.048611in}{0.000000in}}%
\pgfusepath{stroke,fill}%
}%
\begin{pgfscope}%
\pgfsys@transformshift{0.800000in}{1.631709in}%
\pgfsys@useobject{currentmarker}{}%
\end{pgfscope}%
\end{pgfscope}%
\begin{pgfscope}%
\definecolor{textcolor}{rgb}{0.000000,0.000000,0.000000}%
\pgfsetstrokecolor{textcolor}%
\pgfsetfillcolor{textcolor}%
\pgftext[x=0.285508in, y=1.578947in, left, base]{\color{textcolor}\sffamily\fontsize{10.000000}{12.000000}\selectfont \ensuremath{-}0.50}%
\end{pgfscope}%
\begin{pgfscope}%
\pgfsetbuttcap%
\pgfsetroundjoin%
\definecolor{currentfill}{rgb}{0.000000,0.000000,0.000000}%
\pgfsetfillcolor{currentfill}%
\pgfsetlinewidth{0.803000pt}%
\definecolor{currentstroke}{rgb}{0.000000,0.000000,0.000000}%
\pgfsetstrokecolor{currentstroke}%
\pgfsetdash{}{0pt}%
\pgfsys@defobject{currentmarker}{\pgfqpoint{-0.048611in}{0.000000in}}{\pgfqpoint{-0.000000in}{0.000000in}}{%
\pgfpathmoveto{\pgfqpoint{-0.000000in}{0.000000in}}%
\pgfpathlineto{\pgfqpoint{-0.048611in}{0.000000in}}%
\pgfusepath{stroke,fill}%
}%
\begin{pgfscope}%
\pgfsys@transformshift{0.800000in}{2.081688in}%
\pgfsys@useobject{currentmarker}{}%
\end{pgfscope}%
\end{pgfscope}%
\begin{pgfscope}%
\definecolor{textcolor}{rgb}{0.000000,0.000000,0.000000}%
\pgfsetstrokecolor{textcolor}%
\pgfsetfillcolor{textcolor}%
\pgftext[x=0.285508in, y=2.028927in, left, base]{\color{textcolor}\sffamily\fontsize{10.000000}{12.000000}\selectfont \ensuremath{-}0.25}%
\end{pgfscope}%
\begin{pgfscope}%
\pgfsetbuttcap%
\pgfsetroundjoin%
\definecolor{currentfill}{rgb}{0.000000,0.000000,0.000000}%
\pgfsetfillcolor{currentfill}%
\pgfsetlinewidth{0.803000pt}%
\definecolor{currentstroke}{rgb}{0.000000,0.000000,0.000000}%
\pgfsetstrokecolor{currentstroke}%
\pgfsetdash{}{0pt}%
\pgfsys@defobject{currentmarker}{\pgfqpoint{-0.048611in}{0.000000in}}{\pgfqpoint{-0.000000in}{0.000000in}}{%
\pgfpathmoveto{\pgfqpoint{-0.000000in}{0.000000in}}%
\pgfpathlineto{\pgfqpoint{-0.048611in}{0.000000in}}%
\pgfusepath{stroke,fill}%
}%
\begin{pgfscope}%
\pgfsys@transformshift{0.800000in}{2.531668in}%
\pgfsys@useobject{currentmarker}{}%
\end{pgfscope}%
\end{pgfscope}%
\begin{pgfscope}%
\definecolor{textcolor}{rgb}{0.000000,0.000000,0.000000}%
\pgfsetstrokecolor{textcolor}%
\pgfsetfillcolor{textcolor}%
\pgftext[x=0.393533in, y=2.478907in, left, base]{\color{textcolor}\sffamily\fontsize{10.000000}{12.000000}\selectfont 0.00}%
\end{pgfscope}%
\begin{pgfscope}%
\pgfsetbuttcap%
\pgfsetroundjoin%
\definecolor{currentfill}{rgb}{0.000000,0.000000,0.000000}%
\pgfsetfillcolor{currentfill}%
\pgfsetlinewidth{0.803000pt}%
\definecolor{currentstroke}{rgb}{0.000000,0.000000,0.000000}%
\pgfsetstrokecolor{currentstroke}%
\pgfsetdash{}{0pt}%
\pgfsys@defobject{currentmarker}{\pgfqpoint{-0.048611in}{0.000000in}}{\pgfqpoint{-0.000000in}{0.000000in}}{%
\pgfpathmoveto{\pgfqpoint{-0.000000in}{0.000000in}}%
\pgfpathlineto{\pgfqpoint{-0.048611in}{0.000000in}}%
\pgfusepath{stroke,fill}%
}%
\begin{pgfscope}%
\pgfsys@transformshift{0.800000in}{2.981648in}%
\pgfsys@useobject{currentmarker}{}%
\end{pgfscope}%
\end{pgfscope}%
\begin{pgfscope}%
\definecolor{textcolor}{rgb}{0.000000,0.000000,0.000000}%
\pgfsetstrokecolor{textcolor}%
\pgfsetfillcolor{textcolor}%
\pgftext[x=0.393533in, y=2.928886in, left, base]{\color{textcolor}\sffamily\fontsize{10.000000}{12.000000}\selectfont 0.25}%
\end{pgfscope}%
\begin{pgfscope}%
\pgfsetbuttcap%
\pgfsetroundjoin%
\definecolor{currentfill}{rgb}{0.000000,0.000000,0.000000}%
\pgfsetfillcolor{currentfill}%
\pgfsetlinewidth{0.803000pt}%
\definecolor{currentstroke}{rgb}{0.000000,0.000000,0.000000}%
\pgfsetstrokecolor{currentstroke}%
\pgfsetdash{}{0pt}%
\pgfsys@defobject{currentmarker}{\pgfqpoint{-0.048611in}{0.000000in}}{\pgfqpoint{-0.000000in}{0.000000in}}{%
\pgfpathmoveto{\pgfqpoint{-0.000000in}{0.000000in}}%
\pgfpathlineto{\pgfqpoint{-0.048611in}{0.000000in}}%
\pgfusepath{stroke,fill}%
}%
\begin{pgfscope}%
\pgfsys@transformshift{0.800000in}{3.431628in}%
\pgfsys@useobject{currentmarker}{}%
\end{pgfscope}%
\end{pgfscope}%
\begin{pgfscope}%
\definecolor{textcolor}{rgb}{0.000000,0.000000,0.000000}%
\pgfsetstrokecolor{textcolor}%
\pgfsetfillcolor{textcolor}%
\pgftext[x=0.393533in, y=3.378866in, left, base]{\color{textcolor}\sffamily\fontsize{10.000000}{12.000000}\selectfont 0.50}%
\end{pgfscope}%
\begin{pgfscope}%
\pgfsetbuttcap%
\pgfsetroundjoin%
\definecolor{currentfill}{rgb}{0.000000,0.000000,0.000000}%
\pgfsetfillcolor{currentfill}%
\pgfsetlinewidth{0.803000pt}%
\definecolor{currentstroke}{rgb}{0.000000,0.000000,0.000000}%
\pgfsetstrokecolor{currentstroke}%
\pgfsetdash{}{0pt}%
\pgfsys@defobject{currentmarker}{\pgfqpoint{-0.048611in}{0.000000in}}{\pgfqpoint{-0.000000in}{0.000000in}}{%
\pgfpathmoveto{\pgfqpoint{-0.000000in}{0.000000in}}%
\pgfpathlineto{\pgfqpoint{-0.048611in}{0.000000in}}%
\pgfusepath{stroke,fill}%
}%
\begin{pgfscope}%
\pgfsys@transformshift{0.800000in}{3.881607in}%
\pgfsys@useobject{currentmarker}{}%
\end{pgfscope}%
\end{pgfscope}%
\begin{pgfscope}%
\definecolor{textcolor}{rgb}{0.000000,0.000000,0.000000}%
\pgfsetstrokecolor{textcolor}%
\pgfsetfillcolor{textcolor}%
\pgftext[x=0.393533in, y=3.828846in, left, base]{\color{textcolor}\sffamily\fontsize{10.000000}{12.000000}\selectfont 0.75}%
\end{pgfscope}%
\begin{pgfscope}%
\pgfsetbuttcap%
\pgfsetroundjoin%
\definecolor{currentfill}{rgb}{0.000000,0.000000,0.000000}%
\pgfsetfillcolor{currentfill}%
\pgfsetlinewidth{0.803000pt}%
\definecolor{currentstroke}{rgb}{0.000000,0.000000,0.000000}%
\pgfsetstrokecolor{currentstroke}%
\pgfsetdash{}{0pt}%
\pgfsys@defobject{currentmarker}{\pgfqpoint{-0.048611in}{0.000000in}}{\pgfqpoint{-0.000000in}{0.000000in}}{%
\pgfpathmoveto{\pgfqpoint{-0.000000in}{0.000000in}}%
\pgfpathlineto{\pgfqpoint{-0.048611in}{0.000000in}}%
\pgfusepath{stroke,fill}%
}%
\begin{pgfscope}%
\pgfsys@transformshift{0.800000in}{4.331587in}%
\pgfsys@useobject{currentmarker}{}%
\end{pgfscope}%
\end{pgfscope}%
\begin{pgfscope}%
\definecolor{textcolor}{rgb}{0.000000,0.000000,0.000000}%
\pgfsetstrokecolor{textcolor}%
\pgfsetfillcolor{textcolor}%
\pgftext[x=0.393533in, y=4.278825in, left, base]{\color{textcolor}\sffamily\fontsize{10.000000}{12.000000}\selectfont 1.00}%
\end{pgfscope}%
\begin{pgfscope}%
\definecolor{textcolor}{rgb}{0.000000,0.000000,0.000000}%
\pgfsetstrokecolor{textcolor}%
\pgfsetfillcolor{textcolor}%
\pgftext[x=0.229952in,y=2.554200in,,bottom,rotate=90.000000]{\color{textcolor}\sffamily\fontsize{10.000000}{12.000000}\selectfont x[n] * m[n]}%
\end{pgfscope}%
\begin{pgfscope}%
\pgfpathrectangle{\pgfqpoint{0.800000in}{0.567600in}}{\pgfqpoint{4.960000in}{3.973200in}}%
\pgfusepath{clip}%
\pgfsetrectcap%
\pgfsetroundjoin%
\pgfsetlinewidth{1.505625pt}%
\definecolor{currentstroke}{rgb}{0.121569,0.466667,0.705882}%
\pgfsetstrokecolor{currentstroke}%
\pgfsetdash{}{0pt}%
\pgfpathmoveto{\pgfqpoint{1.025455in}{2.438155in}}%
\pgfpathlineto{\pgfqpoint{1.063346in}{2.345190in}}%
\pgfpathlineto{\pgfqpoint{1.101238in}{2.268236in}}%
\pgfpathlineto{\pgfqpoint{1.139129in}{2.177638in}}%
\pgfpathlineto{\pgfqpoint{1.177021in}{2.094597in}}%
\pgfpathlineto{\pgfqpoint{1.214912in}{1.992768in}}%
\pgfpathlineto{\pgfqpoint{1.252804in}{1.889120in}}%
\pgfpathlineto{\pgfqpoint{1.290695in}{1.806329in}}%
\pgfpathlineto{\pgfqpoint{1.328587in}{1.730079in}}%
\pgfpathlineto{\pgfqpoint{1.366478in}{1.644146in}}%
\pgfpathlineto{\pgfqpoint{1.404370in}{1.563922in}}%
\pgfpathlineto{\pgfqpoint{1.442261in}{1.483576in}}%
\pgfpathlineto{\pgfqpoint{1.480153in}{1.387958in}}%
\pgfpathlineto{\pgfqpoint{1.518044in}{1.299452in}}%
\pgfpathlineto{\pgfqpoint{1.555936in}{1.206809in}}%
\pgfpathlineto{\pgfqpoint{1.593827in}{1.116982in}}%
\pgfpathlineto{\pgfqpoint{1.631719in}{1.038766in}}%
\pgfpathlineto{\pgfqpoint{1.669610in}{0.951641in}}%
\pgfpathlineto{\pgfqpoint{1.707502in}{0.869430in}}%
\pgfpathlineto{\pgfqpoint{1.745393in}{0.776347in}}%
\pgfpathlineto{\pgfqpoint{1.783285in}{0.974280in}}%
\pgfpathlineto{\pgfqpoint{1.821176in}{1.160381in}}%
\pgfpathlineto{\pgfqpoint{1.859068in}{1.315416in}}%
\pgfpathlineto{\pgfqpoint{1.896960in}{1.483988in}}%
\pgfpathlineto{\pgfqpoint{1.934851in}{1.651143in}}%
\pgfpathlineto{\pgfqpoint{1.972743in}{1.828022in}}%
\pgfpathlineto{\pgfqpoint{2.010634in}{2.033789in}}%
\pgfpathlineto{\pgfqpoint{2.048526in}{2.204021in}}%
\pgfpathlineto{\pgfqpoint{2.086417in}{2.382593in}}%
\pgfpathlineto{\pgfqpoint{2.124309in}{2.562052in}}%
\pgfpathlineto{\pgfqpoint{2.162200in}{2.718161in}}%
\pgfpathlineto{\pgfqpoint{2.200092in}{2.898783in}}%
\pgfpathlineto{\pgfqpoint{2.237983in}{3.082853in}}%
\pgfpathlineto{\pgfqpoint{2.275875in}{3.262257in}}%
\pgfpathlineto{\pgfqpoint{2.313766in}{3.429940in}}%
\pgfpathlineto{\pgfqpoint{2.351658in}{3.595802in}}%
\pgfpathlineto{\pgfqpoint{2.389549in}{3.763382in}}%
\pgfpathlineto{\pgfqpoint{2.427441in}{3.955417in}}%
\pgfpathlineto{\pgfqpoint{2.465332in}{4.129662in}}%
\pgfpathlineto{\pgfqpoint{2.503224in}{4.311150in}}%
\pgfpathlineto{\pgfqpoint{2.541115in}{4.285891in}}%
\pgfpathlineto{\pgfqpoint{2.579007in}{4.291497in}}%
\pgfpathlineto{\pgfqpoint{2.616898in}{4.298902in}}%
\pgfpathlineto{\pgfqpoint{2.654790in}{4.296416in}}%
\pgfpathlineto{\pgfqpoint{2.692681in}{4.288510in}}%
\pgfpathlineto{\pgfqpoint{2.730573in}{4.134363in}}%
\pgfpathlineto{\pgfqpoint{2.768464in}{3.936507in}}%
\pgfpathlineto{\pgfqpoint{2.806356in}{3.763748in}}%
\pgfpathlineto{\pgfqpoint{2.844248in}{3.586333in}}%
\pgfpathlineto{\pgfqpoint{2.882139in}{3.403186in}}%
\pgfpathlineto{\pgfqpoint{2.920031in}{3.228301in}}%
\pgfpathlineto{\pgfqpoint{2.957922in}{3.025356in}}%
\pgfpathlineto{\pgfqpoint{2.995814in}{2.853988in}}%
\pgfpathlineto{\pgfqpoint{3.033705in}{2.687024in}}%
\pgfpathlineto{\pgfqpoint{3.071597in}{2.531762in}}%
\pgfpathlineto{\pgfqpoint{3.109488in}{2.358900in}}%
\pgfpathlineto{\pgfqpoint{3.147380in}{2.174608in}}%
\pgfpathlineto{\pgfqpoint{3.185271in}{1.990777in}}%
\pgfpathlineto{\pgfqpoint{3.223163in}{1.808173in}}%
\pgfpathlineto{\pgfqpoint{3.261054in}{1.629743in}}%
\pgfpathlineto{\pgfqpoint{3.298946in}{1.458385in}}%
\pgfpathlineto{\pgfqpoint{3.336837in}{1.267492in}}%
\pgfpathlineto{\pgfqpoint{3.374729in}{1.091580in}}%
\pgfpathlineto{\pgfqpoint{3.412620in}{0.939858in}}%
\pgfpathlineto{\pgfqpoint{3.450512in}{0.774591in}}%
\pgfpathlineto{\pgfqpoint{3.488403in}{0.774743in}}%
\pgfpathlineto{\pgfqpoint{3.526295in}{0.767055in}}%
\pgfpathlineto{\pgfqpoint{3.564186in}{0.768741in}}%
\pgfpathlineto{\pgfqpoint{3.602078in}{0.748200in}}%
\pgfpathlineto{\pgfqpoint{3.639969in}{0.762658in}}%
\pgfpathlineto{\pgfqpoint{3.677861in}{0.950533in}}%
\pgfpathlineto{\pgfqpoint{3.715752in}{1.139522in}}%
\pgfpathlineto{\pgfqpoint{3.753644in}{1.305844in}}%
\pgfpathlineto{\pgfqpoint{3.791536in}{1.458564in}}%
\pgfpathlineto{\pgfqpoint{3.829427in}{1.633667in}}%
\pgfpathlineto{\pgfqpoint{3.867319in}{1.831854in}}%
\pgfpathlineto{\pgfqpoint{3.905210in}{2.025598in}}%
\pgfpathlineto{\pgfqpoint{3.943102in}{2.181694in}}%
\pgfpathlineto{\pgfqpoint{3.980993in}{2.370077in}}%
\pgfpathlineto{\pgfqpoint{4.018885in}{2.548500in}}%
\pgfpathlineto{\pgfqpoint{4.056776in}{2.732231in}}%
\pgfpathlineto{\pgfqpoint{4.094668in}{2.922416in}}%
\pgfpathlineto{\pgfqpoint{4.132559in}{3.108505in}}%
\pgfpathlineto{\pgfqpoint{4.170451in}{3.280278in}}%
\pgfpathlineto{\pgfqpoint{4.208342in}{3.467305in}}%
\pgfpathlineto{\pgfqpoint{4.246234in}{3.628430in}}%
\pgfpathlineto{\pgfqpoint{4.284125in}{3.820951in}}%
\pgfpathlineto{\pgfqpoint{4.322017in}{3.989800in}}%
\pgfpathlineto{\pgfqpoint{4.359908in}{4.167559in}}%
\pgfpathlineto{\pgfqpoint{4.397800in}{4.317810in}}%
\pgfpathlineto{\pgfqpoint{4.435691in}{4.312080in}}%
\pgfpathlineto{\pgfqpoint{4.473583in}{4.330650in}}%
\pgfpathlineto{\pgfqpoint{4.511474in}{4.350034in}}%
\pgfpathlineto{\pgfqpoint{4.549366in}{4.360200in}}%
\pgfpathlineto{\pgfqpoint{4.587257in}{4.358753in}}%
\pgfpathlineto{\pgfqpoint{4.625149in}{4.157275in}}%
\pgfpathlineto{\pgfqpoint{4.663040in}{3.982044in}}%
\pgfpathlineto{\pgfqpoint{4.700932in}{3.828715in}}%
\pgfpathlineto{\pgfqpoint{4.738824in}{3.629565in}}%
\pgfpathlineto{\pgfqpoint{4.776715in}{3.463671in}}%
\pgfpathlineto{\pgfqpoint{4.814607in}{3.295889in}}%
\pgfpathlineto{\pgfqpoint{4.852498in}{3.116195in}}%
\pgfpathlineto{\pgfqpoint{4.890390in}{2.928876in}}%
\pgfpathlineto{\pgfqpoint{4.928281in}{2.744463in}}%
\pgfpathlineto{\pgfqpoint{4.966173in}{2.551703in}}%
\pgfpathlineto{\pgfqpoint{5.004064in}{2.391312in}}%
\pgfpathlineto{\pgfqpoint{5.041956in}{2.219161in}}%
\pgfpathlineto{\pgfqpoint{5.079847in}{2.036759in}}%
\pgfpathlineto{\pgfqpoint{5.117739in}{1.853815in}}%
\pgfpathlineto{\pgfqpoint{5.155630in}{1.687433in}}%
\pgfpathlineto{\pgfqpoint{5.193522in}{1.519915in}}%
\pgfpathlineto{\pgfqpoint{5.231413in}{1.338023in}}%
\pgfpathlineto{\pgfqpoint{5.269305in}{1.154020in}}%
\pgfpathlineto{\pgfqpoint{5.307196in}{0.973048in}}%
\pgfpathlineto{\pgfqpoint{5.345088in}{0.784271in}}%
\pgfpathlineto{\pgfqpoint{5.382979in}{0.788219in}}%
\pgfpathlineto{\pgfqpoint{5.420871in}{0.772267in}}%
\pgfpathlineto{\pgfqpoint{5.458762in}{0.755685in}}%
\pgfpathlineto{\pgfqpoint{5.496654in}{0.763991in}}%
\pgfpathlineto{\pgfqpoint{5.534545in}{0.763363in}}%
\pgfusepath{stroke}%
\end{pgfscope}%
\begin{pgfscope}%
\pgfsetrectcap%
\pgfsetmiterjoin%
\pgfsetlinewidth{0.803000pt}%
\definecolor{currentstroke}{rgb}{0.000000,0.000000,0.000000}%
\pgfsetstrokecolor{currentstroke}%
\pgfsetdash{}{0pt}%
\pgfpathmoveto{\pgfqpoint{0.800000in}{0.567600in}}%
\pgfpathlineto{\pgfqpoint{0.800000in}{4.540800in}}%
\pgfusepath{stroke}%
\end{pgfscope}%
\begin{pgfscope}%
\pgfsetrectcap%
\pgfsetmiterjoin%
\pgfsetlinewidth{0.803000pt}%
\definecolor{currentstroke}{rgb}{0.000000,0.000000,0.000000}%
\pgfsetstrokecolor{currentstroke}%
\pgfsetdash{}{0pt}%
\pgfpathmoveto{\pgfqpoint{5.760000in}{0.567600in}}%
\pgfpathlineto{\pgfqpoint{5.760000in}{4.540800in}}%
\pgfusepath{stroke}%
\end{pgfscope}%
\begin{pgfscope}%
\pgfsetrectcap%
\pgfsetmiterjoin%
\pgfsetlinewidth{0.803000pt}%
\definecolor{currentstroke}{rgb}{0.000000,0.000000,0.000000}%
\pgfsetstrokecolor{currentstroke}%
\pgfsetdash{}{0pt}%
\pgfpathmoveto{\pgfqpoint{0.800000in}{0.567600in}}%
\pgfpathlineto{\pgfqpoint{5.760000in}{0.567600in}}%
\pgfusepath{stroke}%
\end{pgfscope}%
\begin{pgfscope}%
\pgfsetrectcap%
\pgfsetmiterjoin%
\pgfsetlinewidth{0.803000pt}%
\definecolor{currentstroke}{rgb}{0.000000,0.000000,0.000000}%
\pgfsetstrokecolor{currentstroke}%
\pgfsetdash{}{0pt}%
\pgfpathmoveto{\pgfqpoint{0.800000in}{4.540800in}}%
\pgfpathlineto{\pgfqpoint{5.760000in}{4.540800in}}%
\pgfusepath{stroke}%
\end{pgfscope}%
\begin{pgfscope}%
\definecolor{textcolor}{rgb}{0.000000,0.000000,0.000000}%
\pgfsetstrokecolor{textcolor}%
\pgfsetfillcolor{textcolor}%
\pgftext[x=3.280000in,y=4.624133in,,base]{\color{textcolor}\sffamily\fontsize{12.000000}{14.400000}\selectfont n vs x[n] * m[n] (N=20)}%
\end{pgfscope}%
\end{pgfpicture}%
\makeatother%
\endgroup%


        Convolving with the N-point moving average filter reduces the noise of the signal. Noise is reduced more as $N$ increases.

        \begin{minted}{python}
            import numpy as np
            import matplotlib.pyplot as plt

            path = "hw2_signal.csv"
            file = open(path, "r")

            splitted = file.read().split(",")
            si = int(splitted[0])
            signal = [float(i) for i in splitted[1:]]

            def x(n):
                return signal[n - si] if n >= si and n <= si + len(signal) else 0

            def delta5(n):
                if n == 5: 
                    return 1
                return 0

            def npoint(N):
                def npointN(n):
                    return 1/N if n >= 0 and n <= (N-1) else 0
                return npointN

            def convolve(n, x, h):
                sum = 0
                for k in range(si, si+len(signal)):
                    sum = sum + x(k)*h(n-k)
                return sum

            range_arr = range(si, si+len(signal))
            x_arr = [x(n) for n in range_arr]
            delta5conv_arr = [convolve(n, x, delta5) for n in range_arr]
            npoint3conv_arr = [convolve(n, x, npoint(3)) for n in range_arr]
            npoint5conv_arr = [convolve(n, x, npoint(5)) for n in range_arr]
            npoint10conv_arr = [convolve(n, x, npoint(10)) for n in range_arr]
            npoint20conv_arr = [convolve(n, x, npoint(20)) for n in range_arr]

            plt.xlabel("n")
            plt.ylabel("x[n] * h[n]")
            plt.title("n vs x[n] * h[n]")
            plt.plot(range_arr, delta5conv_arr, "r")
            plt.plot(range_arr, x_arr, "b")
            plt.show()

            plt.xlabel("n")
            plt.ylabel("x[n] * m[n]")
            plt.title("n vs x[n] * m[n] (N=3)")
            plt.plot(range_arr, npoint3conv_arr)
            plt.show()

            plt.xlabel("n")
            plt.ylabel("x[n] * m[n]")
            plt.title("n vs x[n] * m[n] (N=5)")
            plt.plot(range_arr, npoint5conv_arr)
            plt.show()

            plt.xlabel("n")
            plt.ylabel("x[n] * m[n]")
            plt.title("n vs x[n] * m[n] (N=10)")
            plt.plot(range_arr, npoint10conv_arr)
            plt.show()

            plt.xlabel("n")
            plt.ylabel("x[n] * m[n]")
            plt.title("n vs x[n] * m[n] (N=20)")
            plt.plot(range_arr, npoint20conv_arr)
            plt.show()

        \end{minted}
    \end{enumerate}    

\end{enumerate}


\end{document}


